\section{Monitor\_nD: A general Monitor for 0D/1D/2D records}
\label{s:monitornd}

This component was contributed by Emmanuel Farhi, Institute
Laue-Langevin.

The component {\bf Monitor\_nD} is a general Monitor that can output any
set of physical parameters concerning the passing neutrons. The
generated files are either a set of 1D signals ([Intensity] {\it vs.}
[Variable]), or a single 2D signal ([Intensity] {\it vs.} [Variable 1]
{\it vs.} [Variable 1]), and possibly a simple long list of the selected
physical parameters for each neutron.

The input parameters for {\bf Monitor\_nD} are its dimensions $x_{\rm
  min}, x_{\rm max}, y_{\rm min}$, $y_{\rm max}$ (in meters) and an {\it
  options} string describing what to detect, and what to do with the
signals, in clear language. The formatting of the {\it options}
parameter is free, as long as it contains some specific keywords, that
can be sometimes followed by values. The {\it no} or {\it not} option
modifier will revert next option. The {\it all} option can also affect a
set of monitor configuration parameters (see below).

\subsubsection{The Monitor\_nD geometry}

The monitor shape can be selected among four geometries:
\begin{enumerate}
\item{({\it square}) The default geometry is flat rectangular in ($xy$)
    plane with dimensions $x_{\rm min}, x_{\rm max}, y_{\rm min}$,
    $y_{\rm max}$.}
\item{({\it disk}) When choosing this geometry, the detector is a flat
    disk in ($xy$) plane. The radius is then
    \begin{equation}
      \mbox{radius} = \max ( \mbox{abs } [ x_{\rm min}, x_{\rm max}, y_{\rm
        min}, y_{\rm max} ] ).
    \end{equation}
    }
\item{({\it sphere}) The detector is a sphere with the same radius as
    for the {\it disk} geometry.}
\item{({\it cylinder}) The detector is a cylinder with revolution axis
    along $y$ (vertical). The radius in ($xz$) plane is
    \begin{equation}
      \mbox{radius} =  \max ( \mbox{abs } [ x_{\rm min}, x_{\rm max} ] ),
    \end{equation}
    and the height along $y$ is 
    \begin{equation}
      \mbox{height} =  | y_{\rm max} - y_{\rm max} |.
    \end{equation}
    }
\end{enumerate}

By default, the monitor is flat, rectangular. Of course, you can choose
the orientation of the {\bf Monitor\_nD} in the instrument description
file with the usual \texttt{ROTATED} modifier.

For the {\it sphere} and {\it cylinder}, the incoming neutrons are
monitored by default, but you can choose to monitor outgoing neutron
with the {\it outgoing} option.

At last, the {\it slit} or {\it absorb} option will ask the component to
absorb the neutrons that do not intersect the monitor.

\subsubsection{The neutron parameters that can be monitored}

There are 25 different variables that can be monitored at the same time
and position. Some can have more than one name (e.g. \texttt{energy} or
\texttt{omega}).


\begin{verbatim}
 kx ky kz k wavevector (Angs-1) Wavevector norm or coordinates
 vx vy vz v            (m/s)    Velocity norm or coordinates
 x y z radius          (m)      Position and radius in (xy) plane
 t time                (s)      Time of Flight
 energy omega          (meV)    Neutron energy
 lambda wavelength     (Angs)   Neutron wavelength
 p intensity flux      (n/s) or (n/cm^2/s) The neutron weight
 ncounts               (1)      The number of events detected
 sx sy sz              (1)      Spin of the neutron
 vdiv                  (deg)    vertical divergence
 hdiv divergence       (deg)    horizontal divergence
 angle                 (deg)    divergence from <z> direction
 theta longitude       (deg)    longitude (x/z)
 phi   lattitude       (deg)    lattitude (y/z)
\end{verbatim}

To tell the component what you want to monitor, just add the variable
names in the {\it options} parameter. The data will be sorted into {\it
  bins} cells (default is 20), between some default {\it limits}, that
can also be set by user. The {\it auto} option will automatically
determine what limits should be used to have a good sampling of signals.

The {\it with borders} option will monitor variables that are outside
the limits. These values are then accumulated on the 'borders' of the
signal.

Each monitoring will record the flux (sum of weights $p$) versus the
given variables. The {\it per cm2} option will ask to normalize the flux
to the monitor section surface.

Some examples ?
\begin{enumerate}
\item{\texttt{options="x bins=30 limits=[-0.05 0.05] ; y"} \\
    will set the monitor to look at $x$ and $y$. For $y$, default bins
    and limits values (monitor dimensions) are used.}
\item{\texttt{options="x y, all bins=30, all limits=[-0.05 0.05]"} \\
    will do the same, but set limits and bins for $x$ and $y$.}
\item{\texttt{options="x y, auto limits"} \\
    will determine itself the required limits for $x$ and $y$ to monitor
    passing neutrons with default {\it bins}=20.}
\end{enumerate}

\subsubsection{The output files}

By default, the file names will be the component name, followed by
automatic extensions showing what was monitored (such as
\texttt{MyMonitor.x}). You can also set the filename in {\it options}
with the {\it file} keyword followed by the file name that you want. The
extension will then be added if the name does not contain a dot (.).

The output files format are standard 1D or 2D McStas detector files.
The {\it no file} option will {\it unactivate} monitor, and make it a
single 0D monitor detecting integrated flux and counts.
The {\it verbose} option will display the nature of the monitor, and the
names of the generated files.

\subsubsection{The 2D output}

When you ask the {\bf Monitor\_nD} to monitor only two variables (e.g.
{\it options} = "x y"), a single 2D file of intensity versus these two
correlated variables will be created.

\subsubsection{The 1D output}

The {\bf Monitor\_nD} can produce a set of 1D files, one for each
monitored variable, when using 1 or more than 2 variables, or when
specifying the {\it multiple} keyword option.

\subsubsection{The List output}

The {\bf Monitor\_nD} can additionally produce a {\it list} of variable
values for neutrons that pass into the monitor. This feature is additive
to the 1D or 2D output. By default only 1000 events will be recorded in
the file, but you can specify for instance "{\it list} 3000 neutrons" or
"{\it list all} neutrons". This last option might require a lot of
memory and generate huge files.

\subsubsection{Usage examples}

\begin{itemize}
\item{
\begin{verbatim}
COMPONENT MyMonitor = Monitor_nD( 
    xmin = -0.1, xmax = 0.1, 
    ymin = -0.1, ymax = 0.1, 
    options = "energy auto limits")
\end{verbatim}
will monitor the neutron energy in a single 1D file (a kind of E\_monitor)}
\item{\texttt{{\it options}="x y, all bins=50"} \\
will monitor the neutron $x$ and $y$ in a single 2D file (same as PSD\_monitor)}

\item{\texttt{{\it options}="multiple x bins=30, y limits=[-0.05 0.05]"} \\
will monitor the neutron $x$ and $y$ in two 1D files}
\item{\texttt{{\it options}="x y z kx ky kz,  auto limits"} \\
will monitor theses variables in six 1D files}
\item{\texttt{{\it options}="x y z kx ky kz, list all, auto limits"} \\
will monitor all theses neutron variables in one long list}
\item{\texttt{{\it options}="multiple x y z kx ky kz, and list 2000,  auto limits"} \\
    will monitor all theses neutron variables in one list of 2000 events
    and in six 1D files}
\end{itemize}
