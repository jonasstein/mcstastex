% Emacs settings: -*-mode: latex; TeX-master: "manual.tex"; -*-

\section{Source\_adapt: A neutron source with adaptive importance sampling}
\label{s:Source_adapt}
\label{s:source_adapt}

The {\bf Source\_adapt} component is a neutron source that uses adaptive
importance sampling to improve the efficiency of the simulations. It
works by changing on-the-fly the probability distributions from which
the initial neutron state is sampled so that samples in regions that
contribute much to the accuracy of the overall result are preferred over
samples that contribute little. The method can achive improvements of a
factor of ten or sometimes several hundred in simulations where only a
small part of the initial phase space contains useful neutrons.

The physical characteristics of the source are similar to those of
Source\_flat (see section~\ref{sourceaim}). The source is a thin
rectangle in the $X$-$Y$ plane with a flat energy spectrum in a
user-specified range. The flux per area per steradian per
{\AA}ngstr{\o}m per second is specified by the user; the total weight of
neutrons emitted from the source will then be the same irrespectively of
the number of neutron histories simulated, corresponding to one second
of measurements.

The initial neutron weight is given by (see
section~\ref{Source_flux_lambda} for details)
$$ p_0 = \frac{N_{\rm total}}{N_{\rm sim}} = 
    \frac{\Phi}{N_{\rm sim}} A \Omega \Delta\lambda $$
Here $\Delta\lambda$ is the total wavelength range of the source; since
the spectrum is flat in energy (but not in wavelength), the flux
will actually be different for different energies. A later version of
this component will probably adapt (in a backward-compatible way) a more
sensible way to specify the flux. For now, an energy or wavelength
monitor (see sections~\ref{s:e_monitor} and~\ref{s:L_monitor}) placed
just after the source will show the actual energy-dependent flux.


\subsection{The adaption algorithm}

The adaptive importance sampling works by subdividing the initial
neutron phase space into a number of equal-sized bins. The division is
done on the three dimensions of energy, horizontal position, and
horizontal divergence, using $N_{\rm eng}$, $N_{\rm pos}$, and $N_{\rm
  div}$ number of bins in each dimension, respectively. The total number
of bins is therefore
$$
N_{\rm bin} = N_{\rm eng} N_{\rm pos} N_{\rm div}
$$
Each bin $i$ is assigned a sampling weight $w_i$; the probability of
emitting a neutron within bin $i$ is
$$
P(i) = \frac{w_i}{\sum_{j=1}^{N_{\rm bin}} w_j}
$$
In order to avoid false learning, the sampling weight of a bin is
kept larger than $w_{\rm min}$, defined as
$$
w_{\rm min} = \frac{\beta}{N_{\rm bin}}\sum_{j=1}^{N_{\rm bin}}w_j,\qquad
    0 \leq \beta \leq 1
$$
This way a (small) fraction $\beta$ of the neutrons are sampled
uniformly from all bins, while the fraction $(1 - \beta)$ are sampled in an adaptive way.

Compared to a uniform sampling of the phase space (where the probability
of each bin is $1/N_{\rm bin}$), the neutron weight
must be adjusted by the amount
$$
\pi_i = \frac{1/N_{\rm bin}}{P(i)} =
    \frac{\sum_{j=1}^{N_{\rm bin}} w_j}{N_{\rm bin} w_i}
$$

In order to set the criteria for adaption, the Adapt\_check component is
used (see section~\ref{s:adapt_check}). The source attemps to sample
only from bins from which neutrons are not absorbed prior to the
position in the instrument at which the Adapt\_check component is
placed. Among those bins, the algorithm attemps to minimize the variance
of the neutron weights at the Adapt\_check position. Thus bins that
would give high weights at the Adapt\_check position are sampled more
often (lowering the weights), while those with low weights are sampled
less often.

Let $\pi = p_1/p_0$ denote the ratio between the neutron weight $p_1$ at
the Adapt\_check position and the initial weight $p_0$ just after the
source. For each bin, the component keeps track of the sum $\psi$ of
$\pi$'s as well as of the total number of neutrons $n_i$ from that
bin. The average weight at the Adapt\_source position of bin $i$ is thus
$\psi_i/n_i$.

We now distribute a total sampling weight of $\beta$ uniformly
among all the bins, and a total weight of $(1 - \beta)$ among bins in
proportion to their average weight $\psi_i/n_i$ at the Adapt\_source
position:
$$
w_i = \frac{\beta}{N_{\rm bin}} + 
    (1-\beta) \frac{\psi_i/n_i}{\sum_{j=1}^{N_{\rm bins}} \psi_j/n_j}
$$
After each neutron event originating from bin $i$, the sampling weight $w_i$
is updated.

This basic idea can be improved with a small modification. The problem
is that until the source has had the time to learn the right sampling
weights, neutrons may be emitted with high neutron weights (but low
probability). These low probability neutrons may account for a large fraction of
the total intensity in detectors, causing large variances in the
result. To avoid this, the component emits early neutrons with a lower
weight, and later neutrons with a higher weight to compensate. This way
the neutrons that are emitted with the best adaption contribute the most
to the result.

The factor with which the neutron weights are adjusted is given by a
logistic curve
\begin{equation}
  F(j) = C\frac{y_0}{y_0 + (1 - y_0) e^{-r_0 j}}
\end{equation}
where $j$ is the index of the particular neutron history, $1 \leq j
\leq N_{\rm hist}$. The constants $y_0$, $r_0$, and $C$ are given by
\begin{eqnarray}
  y_0 &=& \frac{2}{N_{\rm bin}} \\
  r_0 &=& \frac{1}{\alpha}\frac{1}{N_{\rm hist}}
     \log\left(\frac{1 - y_0}{y_0}\right) \\
  C &=& 1 + \log\left(y_0 + \frac{1 - y_0}{N_{\rm hist}}
     e^{-r_0 N_{\rm hist}}\right)
\end{eqnarray}
The number $\alpha$ is given by the user and specifies (as a fraction
between zero and one) the point at which the adaption is considered
good. The initial fraction $\alpha$ of neutron histories are emitted
with low weight; the rest are emitted with high weight:
$$ p_0(j) = 
    \frac{\Phi}{N_{\rm sim}} A \Omega \Delta\lambda 
    \frac{\sum_{j=1}^{N_{\rm bin}} w_j}{N_{\rm bin} w_i}
    F(j)
$$
The choice of the constants $y_0$, $r_0$, and $C$ ensure that
$$
\int_{t=0}^{N_{\rm hist}} F(j) = 1
$$
so that the total intensity over the whole simulation will be correct

Similarly, the adjustment of sampling weights is modified so that the
actual formula used is
$$
w_i(j) = \frac{\beta}{N_{\rm bin}} + 
    (1-\beta) \frac{y_0}{y_0 + (1 - y_0) e^{-r_0 j}}
     \frac{\psi_i/n_i}{\sum_{j=1}^{N_{\rm bins}} \psi_j/n_j}
$$

\subsection{The implementation}

The heart of the algorithm is a discrete distribution $p$. The
distribution has $N$ \emph{bins}, $1\ldots N$. Each bin has a value
$v_i$; the probability of bin $i$ is then $v_i/(\sum_{j=1}^N v_j)$.

Two basic operations are possible on the distribution. An \emph{update}
adds a number $a$ to a bin, setting $v_i^{\rm new} = v_i^{\rm old} +
a$. A \emph{search} finds, for given input $b$, the minimum $i$ such
that
$$ b \leq \sum_{j=1}^{i} v_j. $$
The search operation is used to sample from the distribution p. If $r$
is a uniformly distributed random number on the interval
$[0;\sum_{j=1}^N v_j]$ then $i = {\rm search}(r)$ is a random number
distributed according to $p$. This is seen from the inequality
$$ \sum_{j=1}^{i-1} v_j < r \leq \sum_{j=1}^{i} v_j, $$
from which $r \in [\sum_{j=1}^{i-1} v_j; v_i + \sum_{j=1}^{i-1} v_j]$
which is an interval of length $v_i$. Hence the probability of $i$ is
$v_i/(\sum_{j=1}^N v_j)$.
The update operation is used to
adapt the distribution to the problem at hand during a simulation. Both
the update and the add operation can be performed very efficiently; how
this is achieved will be described elsewhere.

The input parameters for Source\_adapt are
\textit{xmin}, \textit{xmax}, \textit{ymin}, and
\textit{ymax} in meters to set the source dimensions;
\textit{dist}, \textit{xw}, and \textit{yh}
to set the focusing as for Source\_flat (section~\ref{sourceaim}); \textit{E0} and
\textit{dE} to set the range of energies emitted, in meV (the range
will be from $\textit{E0} - \textit{dE}$ to
$\textit{E0} + \textit{dE}$); flux to set the source flux $\Phi$ in ${\rm
  cm}^{-2} {\rm st}^{-1} \textit{\AA} {\rm s}^{-1}$;
$N_{\rm eng}$, $N_{\rm pos}$, and $N_{\rm
  div}$ to set the number of bins in each dimensions; \textit{alpha} and
\textit{beta} to set the parameters $\alpha$ and $\beta$ as described
above; and \textit{filename} to give the name of a file in which to
output the final sampling destribution.

A good general-purpose value for $\alpha$ and $\beta$ is $\alpha = \beta
= 0.25$. The number of bins to choose will depend on the
application. More bins will allow better adaption of the sampling, but
will require more neutron histories to be simulated before a good
adaption is obtained. The output of the sampling distribution is only
meant for debugging, and the units on the axis are not necessarily
meaningful. Setting the filename to \verb+NULL+ disables the output of
the sampling distribution.
