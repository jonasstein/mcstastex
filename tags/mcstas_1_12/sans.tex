\section{Sans\_spheres: A sample of hard spheres for small-angle scattering}
\label{sans}
\index{Samples!Dilute colloid medium}
\index{Diffraction}
\index{Small angle scattering}

\component{Sans\_spheres}{(System); Lise Arleth, Veterinary University of Denmark}{$R$, $x_w$, $y_h$, $z_t$, $r$, $\sigma_a$, $\phi$, $\Delta \rho$, $R_{\rm det}$, $d$}{}{}

The component {\bf Sans\_spheres} models a sample of small independent
spheres of radius $R$, which are uniformly distributed
in a rectangular volume $x_w \times y_h \times z_t$ with a volume
fraction $\phi$. The absorption cross section density for the spheres
is $\sigma_a$ (in units of m$^{-1}$), specified
for neutrons at 2200 m/s. Absorption and incoherent scattering
from the medium is neglected.
The difference in scattering length density
(the contrast) between the hard spheres and the medium is called $\Delta \rho$.
$d$ denotes the distance to the (presumed circular) SANS detector of radius $R$.

A usage example of this component can be found in the \verb+Neutron site/tests/SANS+ instrument from the \verb+mcgui+.

\subsection{Small-angle scattering cross section}
The neutron intensity scattered into a solid angle $\Delta \Omega$
for a flat isotropic SANS sample in transmission geometry
is given by \cite{ILLblue}:
\begin{equation}
I_s(q) = \Psi \Delta\Omega T A z_{\rm max} \frac{d\sigma_v}{d\Omega}(q) ,
\end{equation}
where $\Psi$ is the neutron flux, $T$ is the sample transmission,
$A$ is the illuminated sample area, and $z_{\rm max}$ the length of
the neutron path through the sample.

In this component, we consider only scattering from a thin solution
of monodisperse hard spheres of radius $R$, where the volume-specific
scattering cross section is given by \cite{ILLblue}
\begin{equation}
\frac{d\sigma_v}{d\Omega}(q) =
  n (\Delta\rho)^2 V^2 f(q)  ,
\end{equation}
where $f(q) = \left( 3\frac{\sin(qR)-qR\cos(qR)}{(qR)^3} \right)^2$,
$n$ is the number density of spheres, and $V = 4 / 3 \pi R^3$ is the
sphere volume. (The density is thus $n = \phi/V$.)

Multiple scattering is ignored.

\subsection{Algorithm}
All neutrons, which hit the sample volume, are scattered.
(Hence, no direct beam is simulated.)
For scattered neutrons, the following steps are taken:
\begin{enumerate}
\item Choose a value of $q$ uniformly in the interval $[0;q_{\rm max}]$.
\item Choose a polar angle, $\alpha$,
  for the {\bf q}-vector uniformly in $[0;\pi]$.
\item Scatter the neutron according to $(q,\alpha)$.
\item Calculate and apply the correct weight factor correction.
\end{enumerate}

\subsection{Calculating the weight factor}
The scattering position is found by a Monte Carlo choice uniformly
along the whole (unscattered) beam path with the sample, length $l_{\rm full}$, giving
$f_l = 1/l_{\rm full}$. The direction focusing on the detector gives
(in an small angle approximation) $f_\Omega = d^2 / (\pi R_{\rm det}^2)$.

Hence, the total weight tranformation factor becomes % (more explanation to come)
\begin{equation}
\pi_j = l_{\rm full} (\pi R_{\rm det}^2 / d^2)/(4 \pi)
  n (\Delta\rho)^2 V^2 f(q) \exp(-\mu_a l) ,
\end{equation}
where $\mu_a$ is the linear attenuation factor due to absorption
and $l$ is the total neutron path length within the sample.

This component does NOT simulate absolute intensities. This latter depends on the detector parameters. \index{Bugs}

Some alternative implementations exist as contributed components.

The \verb+SANS+ test/example instrument exists in the distribution for this component.