% Emacs settings: -*-mode: latex; TeX-master: "manual.tex"; -*-

\chapter{The \MCS\ terminology}
\label{s:terminology}

This is a short explanation of phrases and terms which have a specific
meaning within \MCS. We have tried to keep the list as short
as possible with the risk that the reader may occasionally miss
an explanation. In this case, you are more than welcome to contact
the \MCS\ core team.

\noindent
\begin{itemize}
\item{\bf Arm}  A generic \MCS\ component which defines a frame of reference
      for other components.
\item{\bf Component} One unit ({\em e.g.} optical element) in a neutron
      spectrometer. These are considered as Types of elements to be instantiated in an Instrument description.
\item{\bf Component Instance} A named Component (of a given Type) inserted in an Instrument description.
\item{\bf Definition parameter} An input parameter for a component. For
  example the radius of a sample component or the divergence of a collimator.
\item{\bf Input parameter} For a component, either a definition parameter
or a setting parameter. These parameters are supplied by the user to
define the characteristics of the particular instance of the component
definition. For an instrument, a parameter that can be changed at
simulation run-time.
\item{\bf Instrument} An assembly of \MCS\ components defining
      a neutron spectrometer.
\item{\bf Kernel} The \MCS\ language definition and the associated compiler
\item{\bf \MCS} Monte Carlo Simulation of Triple Axis Spectrometers
       (the name of this package). Pronounciation ranges from \emph{mex-tas}, to \emph{mac-stas} and \emph{m-c-stas}.
\item{\bf Output parameter} An output parameter for a component.
  For example the counts in a monitor. An output parameter may be
  accessed from the instrument in which the component is used using
  \verb`MC_GETPAR`.
\item{\bf Run-time} C code, contained in the files
  \verb+mcstas-r.c+ and \verb+mcstas-r.h+ included in the \MCS\
  distribution, that declare functions and variables used by the
  generated simulations.
\item{\bf Setting parameter} Similar to a definition parameter, but with the
  restriction that the value of the parameter must be a number.
\end{itemize}
