\subsection{Mosaic\_anisotropic: The crystal with anisotropic mosaic}

The component {\bf Mosaic\_anisotropic} is a modified version of the
Mosaic\_simple component, intended to replace the Monocromator component
from previous releases. It restricts the scattering vector to be
perpendicular to the crystal surface, but extends the Mosaic\_simple
component by allowing different mosaics in the horizontal and vertical
direction.

The code is largely similar to that for Mosaic\_simple, and the
documentation for the latter should be consulted for details. The
differences are mainly due to two reasons:
\begin{itemize}
\item Some simplifications have been done since two of the components of
  the scattering vector are known to be zero.
\item The computation of the Gaussian for the mosaic is done done using
  different mosaics for the two axes.
\end{itemize}

The input parameters for the component Mosaic\_anisotropic are
\textit{zmin}, \textit{zmax}, \textit{ymin}, and \textit{ymax} to define
the size of the crystal (in meters); \textit{mosaich} and \textit{mosaicv} to define
the mosaic (in minutes of arc); \textit{r0} to define the reflectivity
(no unit); and \textit{Q} to set the length of the scattering vector (in
$\mbox{\AA}^{-1}$).
