% Emacs settings: -*-mode: latex; TeX-master: "manual.tex"; -*-

\chapter{Source components}
\label{c:source}
\index{Sources}

\MCS\ contains a number of different source components, 
and any simulation will contain exactly one source.
The main function of a source is to determine a set of initial
parameters $({\bf r}, {\bf v}, t)$, or equivalent (${\bf r}, v, \Ombold , t $),
for each neutron. This is done by Monte Carlo choices from
suitable distributions. For example, the initial position is
always found from a uniform distribution over the source surface.
For time-of-flight sources, the choice of $t$ is being made on basis of
detailed analytical expressions.
For other sources, the initial neutron time is set to zero.

Polarization is not relevant for sources, 
and we let the neutron spin ${\bf s}=(0,0,0)$.

The shape of most sources can be chosen to be either circular or rectangular.
The initial neutron velocity is selected within an interval
of either the corresponding energy or the corresponding wavelength.

The flux of the sources deserves special attention. The total neutron
intensity is defined as the sum of weights of all emitted neutron rays 
during one simulation 
(the unit of total neutron weight is thus neutrons per second).
The flux is then defined as intensity per area of the source.
For samples that uses a realistic energy/wavelength distribution,
the neutron intensity is given as the integrated neutron intensity
inside the  

(CHECK THIS, WHAT ABOUT THE SOLID ANGLE??)

The flux~$\Phi$ is the number of neutrons emitted per second from a
one~cm$^2$ area on the source surface, with direction within a one
steradian solid angle, and with wavelength within a one {\AA}ngstr{\o}m
interval. The total number of neutrons emitted towards a given diafraghm
in one second is therefore
$$ N_{\rm total} = \Phi A \Omega \Delta\lambda $$
where $A$ is the source area, $\Omega$ is the solid angle of the
diafraghm as seen from the source surface, and $\Delta\lambda$ is the
width of the wavelength interval in which neutrons are emitted (assuming
a uniform wavelength spectrum). If $N_{\rm sim}$ denotes the number of
neutron histories to simulate, the initial neutron weight $p_0$ must be set to
$$ p_0 = \frac{N_{\rm total}}{N_{\rm sim}} = 
    \frac{\Phi}{N_{\rm sim}} A \Omega \Delta\lambda $$

The simulations are performed so that detector intensities 
are independent of the number of neutron histories simulated
(though of course more neutron histories will give better statistics).

\newpage
\section{Source\_simple: A simple continuous source
with a flat energy/wavelength spectrum}
\label{source-simple}
\index{Sources!Source\_simple}

\component{Source\_simple}{System}{ $r_{\rm s}$, $z_{\rm foc}$, $w$, $h$, $E_0$, $\Delta E$, $\Psi$}{$\lambda_0$, $d\lambda$}{Validated, position is the center of disk}

This component is
a simple source with a energy distribution which is uniform
in the range $E_0 \pm dE$
(or wavelength distribution in the range $\lambda_0 \pm d\lambda$).
This component is not used for detailed time-of-flight simulations,
so we put $t=0$ for all neutron rays.

The initial neutron ray position is chosen randomly from within a
circle of radius $r_{\rm s}$ in the $z=0$ plane.
This geometry is a fair approximation
of a cylindrical cold/thermal source with the beam going out along
the cylinder axis.

The initial neutron ray direction is focused within
a solid angle, defined by a rectangular target of width
$w$, height $h$, parallel to
the $xy$ plane placed at $(0,0,z_{\rm foc})$.

The initial weight of the created neutron ray, $p_0$, is set to the
energy-integrated flux, $\Psi$, times the source area, $\pi r_{\rm s}^2$
times a solid-angle factor, which is basically the
solid angle of the focusing rectangle.
See also the discussion on focusing in the \MCS\ Manual \ref{s:focus}.

This component replaces Source\_flux\_lambda, Source\_flat, Source\_flat\_lambda, and
Source\_flux.


%\section{Source\_div: A continuous source with specified divergence}
\label{source-div}
\index{Sources!Source\_div}

\component{Source\_div}{System}{ $w$, $h$, $\delta_h$, $\delta_v$, $E_0$, $\Delta E$}{$\lambda_0$, $\Delta\lambda$, gauss}{Validated}

{\bf Source\_div} is a rectangular source, $w \times h$ (in m), which emits a
beam of a specified divergence around the direction of the $z$ axis.
The beam intensity is uniform over
the whole of the source, and the energy (or wavelength) distribution
of the beam is uniform over the specified energy range
$E_0 \pm \Delta E$ (in meV), or alternatively
the wavelength range $\lambda_0 \pm \delta\lambda$ (in \AA ).

The source divergencies are $\delta_h$ and $\delta_v$ (FWHM in degrees).
If the \verb+gauss+ flag is set to 0 (default), 
the divergence distribution is uniform, otherwise it is Gaussian.

This component may be used as a simple model of the
beam profile at the end of a guide or at the sample position.



\newpage
\section{Source\_Maxwell\_3: A continuous source
with a Maxwellian spectrum}
\label{source-maxwell}
\index{Sources!Source\_Maxwell\_3}

\component{Source\_Maxwell\_3}{System}{ $h$, $w$, $d_{\rm foc}$, $xw$, $yh$, $\lambda_{\rm low}$, $\lambda_{\rm high}$, $I_1$, $T_1$}{$I_2$, $T_2$, $I_3$, $T_3$}{Validated}

This component is a source with a Maxwellian energy/wavelength distribution
sampled in the range $\lambda_{\rm low}$ to $\lambda_{\rm high}$.
The initial neutron ray position is chosen randomly from within a
rectangle of area $h \times w$ in the $z=0$ plane.
The initial neutron ray direction is focused within
a solid angle, defined by a rectangular target of width
$xw$, height $yh$, parallel to
the $xy$ plane placed at $(0,0,d_{\rm foc})$.
The energy distribution used is a sum of 1, 2, or 3 Maxwellians with
temperatures $T_1$ to $T_3$ and integrated intensities $I_1$ to $I_3$.

The initial weight of the created neutron ray, $\pi_1$, is
calculated in the following way for one single Maxwellian:
The intensity in a small wavelength interval $[\lambda, \lambda+d\lambda]$ is
$ I_1 M(\lambda,T_1) d\lambda $
where
$M(\lambda,T_1) = 2 \alpha^2 \exp(-\alpha/\lambda^2) / \lambda^5 $ 
is the normalized Maxwell distribution ($\alpha=949.0$~K \AA$^2/T_1$).
The number of neutrons per second through a focusing window
of solid angle $\Omega$
from a source of area $A$ within the wavelength interval $\lambda_1$ to
$\lambda_2$ is thus
\begin{equation}
I_{\rm tot} = \Omega A \int_{\lambda_1}^{\lambda_2} I_1 M(\lambda,T_1) d\lambda.
\end{equation}
In a Monte Carlo integration, the observed intensity becomes
\begin{equation}
I_{\rm MC} \approx N_{\rm MC} \int p(\lambda) \pi_1(\lambda) d\lambda ,
\end{equation}
where $N_{\rm MC}$ is the number of Monte Carlo steps.
We here choose the wavelength from a uniform distribution between the two
limits, giving $p(\lambda)=1/(\lambda_2-\lambda_1)$.
To fulfill $I_{\rm tot} = I_{\rm MC}$ we need to have
\begin{equation}
\pi_1(\lambda) = \Omega A (\lambda_2-\lambda_1) I_1 M(\lambda,T_1) / N_{\rm MC} .
\end{equation}

%This expression is strictly valid only for $\Omega \ll 1$,
%see also the discussion on focusing in section \ref{s:focus}.
The expression is easily generalized to a general number of Maxwellians.

The component {\bf Source\_gen} (see section \ref{source-gen}) 
works on the same principle, but provides more options concerning 
wavelength/energy range specifications, shape, etc.



%\section{Source\_gen: A general continuous source}
\label{source-gen}
\index{Sources!General continuous source}

\component{Source\_gen}{(System) E. Farhi, ILL}{$w$, $h$, $xw$, $yh$, $E_0$, $\Delta E$, $T_1$, $T_2$, $T_3$, $I_1$, $I_2$, $I_3$ }{$r$, $\lambda_0$, $d\lambda$, $E_{min}$, $E_{max}$, $\lambda_{min}$, $\lambda_{max}$}{Validated for Maxwellian expressions. t=0}

This component is a continuous neutron source (rectangular or circular), which aims at
a rectangular target centered at the beam.
The angular divergence is given by the dimensions of the target.
The shape may be rectangular (dimension $h$ and $w$), or a disk of radius $r$.
The wavelength/energy range to emit is specified either using center and half width, or using minimum and maximum boundaries, alternatively for energy and wavelength.
The flux spectrum is specified with the same Maxwellian parameters as in component Source\_Maxwell\_3 (refer to section \ref{source-maxwell}).

Maxwellian parameters for some continuous sources
are given in Table~\ref{t:source_gen-params}. As nobody knows exactly the characteristics of the sources (it is not easy to measure spectrum there), these figures should be used with caution.

\begin{table}
  \begin{center}
  {\let\my=\\
    \begin{tabular}{|c|cccccc|c|}
    \hline
    Source Name & $T_1$ & $I_1$ & $T_2$ & $I_2$ & $T_3$ & $I_3$ & factor \\
    \hline
    PSI cold source & 150.4  & 3.67e11   & 38.74 & 3.64e11    & 14.84& 0.95e11   & * $I_{\rm target}$~(mA)\\
    ILL VCS (H1)    & 216.8  & 1.24e13   & 33.9  & 1.02e13    & 16.7 & 3.042e12  &\\
    ILL HCS (H5)    & 413.5  & 10.22e12  & 145.8 & 3.44e13    & 40.1 & 2.78e13   &\\
    ILL Thermal(H2) & 683.7  & 5.874e12  & 257.7 & 2.51e13    & 16.7 & 1.034e12  & /2.25\\
    ILL Hot source  & 1695   & 1.74e13   & 708   & 3.9e12     &      &           &\\ \hline
    \end{tabular}
    \caption{Flux parameters for present sources used in components
             Source\_gen and Source\_Maxwell\_3.
             For some cases, a correction factor to the intensity
             should be used to reach measured data; for the PSI cold source,
             this correction factor is the beam current, $I_{\rm target}$,
             which is currently of the order 1.2~mA.
}
    \label{t:source_gen-params}
  }
  \end{center}
\end{table}

%\section{Source\_adapt: A neutron source with adaptive importance sampling}
\label{s:Source_adapt}
\label{s:source-adapt}
\index{Optimization}
\index{Sources!Adaptive source}

\component{Source\_adapt}{K. Nielsen}{$x_{min}$, $x_{max}$, $y_{min}$, $y_{max}$, $E0$, $dE$, dist, $xw$, $yh$, $\Phi$}{$\alpha$, $\beta$ (plenty, default values are ok)}{partially validated}

{\bf Source\_adapt} is a neutron source that uses adaptive
importance sampling to improve the efficiency of the simulations. It
works by changing on-the-fly the probability distributions from which
the initial neutron state is sampled so that samples in regions that
contribute much to the accuracy of the overall result are preferred over
samples that contribute little. The method can achieve improvements of a
factor of ten or sometimes several hundred in simulations where only a
small part of the initial phase space contains useful neutrons.
This component uses the correlation between neutron energy,
initial direction and initial position.

The physical characteristics of the source are similar to those of
{\bf Source\_simple} (see section~\ref{source-simple}). The source is a thin
rectangle in the $x$-$y$ plane with a flat energy spectrum in a
user-specified range. The flux, $\Phi$, per area per steradian per
{\AA}ngstr{\o}m per second is specified by the user.

The initial neutron weight is given by Eq. (\ref{proprule}) using
$\Delta\lambda$ as the total wavelength range of the source.
A later version of this component will probably include a
$\lambda$-dependence of the flux.

We use the input parameters \textit{dist}, \textit{xw}, and \textit{yh}
to set the focusing as for Source\_simple (section~\ref{source-simple}).
The energy range will be from $E_0 - dE$ to $E_0 + dE$.
\textit{filename} is used to give the name of a file in which to
output the final sampling destribution, see below.
$N_{\rm eng}$, $N_{\rm pos}$, and $N_{\rm div}$
are used to set the number of bins in each dimensions.
Good general-purpose values for the optimization parameters are
$\alpha = \beta = 0.25$. The number of bins to choose will depend on the
application. More bins will allow better adaption of the sampling, but
will require more neutron histories to be simulated before a good
adaption is obtained. The output of the sampling distribution is only
meant for debugging, and the units on the axis are not necessarily
meaningful. Setting the filename to \verb+NULL+ disables the output of
the sampling distribution.

\subsection{Optimization disclaimer}

A warning is in place here regarding potentially wrong results
using optimization techniques.
It is highly recommanded in any case to benchmark 'optimized' simulations
against non-optimized ones, checking that obtained results are the same,
but hopefully with a much improved statistics.

\subsection{The adaption algorithm}

The adaptive importance sampling works by subdividing the initial
neutron phase space into a number of equal-sized bins. The division is
done on the three dimensions of energy, horizontal position, and
horizontal divergence, using $N_{\rm eng}$, $N_{\rm pos}$, and $N_{\rm
  div}$ number of bins in each dimension, respectively. The total number
of bins is therefore
\begin{equation}
N_{\rm bin} = N_{\rm eng} N_{\rm pos} N_{\rm div}
\end{equation}
Each bin $i$ is assigned a sampling weight $w_i$; the probability of
emitting a neutron within bin $i$ is
\begin{equation}
P(i) = \frac{w_i}{\sum_{j=1}^{N_{\rm bin}} w_j}
\end{equation}
In order to avoid false learning, the sampling weight of a bin is
kept larger than $w_{\rm min}$, defined as
\begin{equation}
w_{\rm min} = \frac{\beta}{N_{\rm bin}}\sum_{j=1}^{N_{\rm bin}}w_j,\qquad
    0 \leq \beta \leq 1
\end{equation}
This way a (small) fraction $\beta$ of the neutrons are sampled
uniformly from all bins, while the fraction $(1 - \beta)$ are sampled in an adaptive way.

Compared to a uniform sampling of the phase space (where the probability
of each bin is $1/N_{\rm bin}$), the neutron weight
must be adjusted as given by (\ref{probrule})
\begin{equation}
\pi_1 = \frac{P_1}{f_{\rm MC,1}} =\frac{1/N_{\rm bin}}{P(i)} =
    \frac{\sum_{j=1}^{N_{\rm bin}} w_j}{N_{\rm bin} w_i} ,
\end{equation}
where $P_1$ is understood by the "natural" uniform sampling.

In order to set the criteria for adaption, the {\bf Adapt\_check} component is
used (see section~\ref{s:adapt_check}). The source attemps to sample
only from bins from which neutrons are not absorbed prior to the
position in the instrument at which {\bf Adapt\_check} is
placed. Among those bins, the algorithm attemps to minimize the variance
of the neutron weights at the {\bf Adapt\_check} position. Thus bins that
would give high weights at the {\bf Adapt\_check} position are sampled more
often (lowering the weights), while those with low weights are sampled
less often.

Let $\pi = p_{\rm ac}/p_0$ denote the ratio between the neutron weight $p_1$ at
the {\bf Adapt\_check} position and the initial weight $p_0$ just after the
source. For each bin, the component keeps track of the sum $\Sigma$ of
$\pi$'s as well as of the total number of neutrons $n_i$ from that
bin. The average weight at the {\bf Adapt\_source} position of bin $i$ is thus
$\Sigma_i/n_i$.

We now distribute a total sampling weight of $\beta$ uniformly
among all the bins, and a total weight of $(1 - \beta)$ among bins in
proportion to their average weight $\Sigma_i/n_i$ at the {\bf Adapt\_source}
position:
\begin{equation}
w_i = \frac{\beta}{N_{\rm bin}} +
    (1-\beta) \frac{\Sigma_i/n_i}{\sum_{j=1}^{N_{\rm bins}} \Sigma_j/n_j}
\end{equation}
After each neutron event originating from bin $i$, the sampling weight $w_i$
is updated.

This basic idea can be improved with a small modification. The problem
is that until the source has had the time to learn the right sampling
weights, neutrons may be emitted with high neutron weights (but low
probability). These low probability neutrons may account for a large fraction of
the total intensity in detectors, causing large variances in the
result. To avoid this, the component emits early neutrons with a lower
weight, and later neutrons with a higher weight to compensate. This way
the neutrons that are emitted with the best adaption contribute the most
to the result.

The factor with which the neutron weights are adjusted is given by a
logistic curve
\begin{equation}
  F(j) = C\frac{y_0}{y_0 + (1 - y_0) e^{-r_0 j}}
\end{equation}
where $j$ is the index of the particular neutron history, $1 \leq j
\leq N_{\rm hist}$. The constants $y_0$, $r_0$, and $C$ are given by
\begin{eqnarray}
  y_0 &=& \frac{2}{N_{\rm bin}} \\
  r_0 &=& \frac{1}{\alpha}\frac{1}{N_{\rm hist}}
     \log\left(\frac{1 - y_0}{y_0}\right) \\
  C &=& 1 + \log\left(y_0 + \frac{1 - y_0}{N_{\rm hist}}
     e^{-r_0 N_{\rm hist}}\right)
\end{eqnarray}
The number $\alpha$ is given by the user and specifies (as a fraction
between zero and one) the point at which the adaption is considered
good. The initial fraction $\alpha$ of neutron histories are emitted
with low weight; the rest are emitted with high weight:
\begin{equation}
  p_0(j) =
    \frac{\Phi}{N_{\rm sim}} A \Omega \Delta\lambda
    \frac{\sum_{j=1}^{N_{\rm bin}} w_j}{N_{\rm bin} w_i}
    F(j)
\end{equation}
The choice of the constants $y_0$, $r_0$, and $C$ ensure that
\begin{equation}
\int_{t=0}^{N_{\rm hist}} F(j) = 1
\end{equation}
so that the total intensity over the whole simulation will be correct

Similarly, the adjustment of sampling weights is modified so that the
actual formula used is
\begin{equation}
w_i(j) = \frac{\beta}{N_{\rm bin}} +
    (1-\beta) \frac{y_0}{y_0 + (1 - y_0) e^{-r_0 j}}
     \frac{\psi_i/n_i}{\sum_{j=1}^{N_{\rm bins}} \psi_j/n_j}
\end{equation}

\subsection{The implementation}

The heart of the algorithm is a discrete distribution $p$. The
distribution has $N$ \emph{bins}, $1\ldots N$. Each bin has a value
$v_i$; the probability of bin $i$ is then $v_i/(\sum_{j=1}^N v_j)$.

Two basic operations are possible on the distribution. An \emph{update}
adds a number $a$ to a bin, setting $v_i^{\rm new} = v_i^{\rm old} +
a$. A \emph{search} finds, for given input $b$, the minimum $i$ such
that
\begin{equation}
 b \leq \sum_{j=1}^{i} v_j.
\end{equation}
The search operation is used to sample from the distribution p. If $r$
is a uniformly distributed random number on the interval
$[0;\sum_{j=1}^N v_j]$ then $i = {\rm search}(r)$ is a random number
distributed according to $p$. This is seen from the inequality
\begin{equation}
\sum_{j=1}^{i-1} v_j < r \leq \sum_{j=1}^{i} v_j,
\end{equation}
from which $r \in [\sum_{j=1}^{i-1} v_j; v_i + \sum_{j=1}^{i-1} v_j]$
which is an interval of length $v_i$. Hence the probability of $i$ is
$v_i/(\sum_{j=1}^N v_j)$.
The update operation is used to
adapt the distribution to the problem at hand during a simulation. Both
the update and the add operation can be performed very efficiently.

As an alternative, you may use the {\bf Source\_Optimizer} component
(see section \ref{source-optimizer}).


%% Emacs settings: -*-mode: latex; TeX-master: "manual.tex"; -*-

\section{Source\_Optimizer: A general Optimizer for McStas}
\label{s:sourceoptimizer}

This component was contributed by Emmanuel Farhi, Institute
Laue-Langevin.

The component {\bf Source\_Optimizer} optimizes the whole neutron flux
in order to achieve better statistics at each {\bf Monitor\_Optimizer}
location(s) (see section~\ref{s:monitoroptimizer} for this latter
component). It can act on any incoming neutron beam (from any source
type), and more than one optimization criteria location can be placed
along the instrument.

The usage of the optimizer is very simple, and usually does not require
any configuration parameter. Anyway the user can still customize the
optimization {\it via} various {\it options}.

The optimizer efficiency makes it easy to increase the number of events
at optimization criteria locations by a factor of 20, and thus decreases
the signal error bars by a factor 4.5. Higher factors can often be
achieved in practise. Of course, the overall flux remains the same as
without optimizer.

\subsection{The optimization algorithm}

When a neutron reaches the {\bf Monitor\_Optimizer} location(s), the
component records its position ($x$, $y$) and speed ($v_x,
v_y, v_z$) when it passed in the {\bf Source\_Optimizer}. Some
distribution tables of {\it good} neutrons characteristics are then
built.

When a {\it bad} neutron comes to the {\bf Source\_Optimizer} (it would
then have few chances to reach {\bf Monitor\_Optimizer}), it is changed
into a better one. That means that its position and velocity coordinates
are translated to better values according to the {\it good} neutrons
distribution tables. Anyway, the neutron energy ($\surd v_x^2 + v_y^2 +
v_z^2$) is kept as far as possible.

The {\bf Source\_Optimizer} works as follow:
\begin{enumerate}
\item{First of all, the {\bf Source\_Optimizer} determines some limits
    ({\it min} and {\it max}) for variables $x, y, v_x, v_y, v_z$.}
\item{Then the component records the non-optimized flux distributions in
    arrays with {\it bins} cells (default is 10 cells). This constitutes
    the {\it Reference } source.}
\item{\label {SourceOptimizer:step3}The {\bf Monitor\_Optimizer} records
    the {\it good} neutrons (that reach it) and communicate an {\it
      Optimized} source to the {\bf Source\_Optimizer}. However, '{\it
      keep}' percent of the original {\it Reference} source is sent
    unmodified (default is 10 \%). The {\it Optimized} source is thus:

    \begin{center}
      \begin{tabular}{rcl}
        {\it Optimized} & = & {\it keep} * {\it Reference} \\
        & + & (1 - {\it keep}) [Neutrons that will reach monitor].
      \end{tabular}
    \end{center}
    }
\item{The {\bf Source\_Optimizer} transforms the {\it bad} neutrons into
    {\it good} ones from the {\it Optimized} source. The resulting
    optimised flux is normalised to the non-optimized one:
    \begin{equation}
      p_{optimized} = p_{initial} \frac{\mbox{Reference}}{\mbox{Optimized}},
    \end{equation}
    and thus the overall flux at {\bf Monitor\_Optimizer} location is
    the same as without the optimizer. Usually, the process sends more
    {\it good} neutrons from the {\it Optimized} source than than in the
    {\it Reference} one.
    The energy (and velocity) spectra of neutron beam is also kept, as
    far as possible. For instance, an optimization of $v_z$ will induce
    a modification of $v_x$ or $v_y$ to try to keep $|\vec{v}|$
    constant.
    }
\item{When the {\it continuous} optimization option is activated (by
    default), the process loops to Step (\ref{SourceOptimizer:step3})
    every '{\it step}' percent of the simulation. This parameter is
    computed automatically (usually around 10 \%) in {\it auto} mode,
    but can also be set by user.}
\end{enumerate}

During steps (1) and (2), some non-optimized neutrons with original
weight $p_{initial}$ may lead to spikes on detector signals. This is
greatly improved by lowering the weight $p$ during these steps, with the
{\it smooth} option.
The component optimizes the neutron parameters on the basis of
independant variables. Howver, it usually does work fine when these
variables are correlated (which is often the case in the course of the
instrument simulation).
The memory requirements of the component are very low, as no big
$n$-dimensional array is needed.

\subsection{Using the Source\_Optimizer}

To use this component, just install the {\bf Source\_Optimizer} after a
source (but any location is possible in principle), and use the {\bf
  Monitor\_Optimizer} at a location where you want to have better
statistics.

\begin{verbatim}
    /* where to act on neutrons */
    COMPONENT optim_s = Source_Optimizer(options="") 
    ...
    /* where to have better statistics */
    COMPONENT optim_m = Monitor_Optimizer( 
    xmin = -0.05, xmax = 0.05, 
    ymin = -0.05, ymax = 0.05,
    optim_comp = optim_s) 
    ...
    /* using more than one Monitor_Optimizer is possible */
\end{verbatim}

The input parameter for {\bf Source\_Optimizer} is a single {\it
  options} string that can contain some specific optimizer configuration
settings in clear language. The formatting of the {\it options}
parameter is free, as long as it contains some specific keywords, that
can be sometimes followed by values.

The default configuration (equivalent to {\it options} = "") is
\begin{center}
\begin{tabular}{rcl}
  {\it options} & = & "{\it continuous} optimization,
  {\it auto} setting, {\it keep} = 10, {\it bins} = 10, \\
  & & {\it smooth} spikes, and do {\it not free} energy during optimization".
\end{tabular}
\end{center}
The keyword modifiers {\it no} or {\it not} revert the next option.
Other options not shown here are:
\begin{verbatim}
verbose         displays optimization process (debug purpose).
unactivate      to unactivate the Optimizer.
file=[name]     Filename where to save optimized source distributions
\end{verbatim}
The {\it file} option will save the source distributions at the end of
the optimization. If no name is given the component name will be used,
and a '.src' extension will be added. By default, no file is generated.
The file format is in a McStas 2D record style.

