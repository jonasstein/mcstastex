% Emacs settings: -*-mode: latex; TeX-master: "manual.tex"; -*-

\chapter{Source components}
\label{c:source}
\index{Sources}
\index{Library!Components!sources}

\MCS\ contains a number of different source components,
and any simulation will contain exactly one source.
The main function of a source is to determine a set of initial
parameters $({\bf r}, {\bf v}, t)$, or equivalent (${\bf r}, v, \Ombold , t $),
for each neutron. This is done by Monte Carlo choices from
suitable distributions. For example, the initial position is
always found from a uniform distribution over the source surface.
For time-of-flight sources, the choice of $t$ is being made on basis of
detailed analytical expressions.
For other sources, the initial neutron time is set to zero (default). In the case you would like to use a 'conventional' source (e.g. steady state source) with time-of-flight settings, it is then \emph{important} to set the time of each neutron using a random number from a distribution. This may be achieved thanks to the \verb+EXTEND+ keyword in the instrument description source:\index{Keyword!EXTEND}

\begin{verbatim}
  TRACE

  COMPONENT MySource=Source_gen(...) AT (...)
  EXTEND
  %{
    t = 1e-3*randpm1(); /* set time to +/- 1 ms */
  %}
\end{verbatim}

Polarization is not relevant for sources,
and we let the neutron spin ${\bf s}=(0,0,0)$.

The shape of most sources can be chosen to be either circular or rectangular.
The initial neutron velocity is selected within an interval
of either the corresponding energy or the corresponding wavelength.

The flux of the sources deserves special attention. The total neutron
intensity is defined as the sum of weights of all emitted neutron rays
during one simulation
(the unit of total neutron weight is thus neutrons per second).
The flux is then defined as intensity per area of the source.
For samples that uses a realistic energy/wavelength distribution,
the neutron intensity is given as the integrated neutron intensity
inside the energy/wavelength range. Similarly, as most sources can focus neutrons to an input window (usually the guide entrance), the intensity is scaled to account for the corresponding solid angle.

The flux~$\Phi$ is the number of neutrons emitted per second from a
one~cm$^2$ area on the source surface, with direction within a one
steradian solid angle, and with wavelength within a one {\AA}ngstr{\o}m
interval. The total number of neutrons emitted towards a given diaphragm
in one second is therefore
$$ N_{\rm total} = \Phi A \Omega \Delta\lambda $$
where $A$ is the source area, $\Omega$ is the solid angle of the
diaphragm as seen from the source surface, and $\Delta\lambda$ is the
width of the wavelength interval in which neutrons are emitted (assuming
a uniform wavelength spectrum). If $N_{\rm sim}$ denotes the number of
neutron histories to simulate, the initial neutron weight $p_0$ must be set to
$$ p_0 = \frac{N_{\rm total}}{N_{\rm sim}} =
    \frac{\Phi}{N_{\rm sim}} A \Omega \Delta\lambda $$

The simulations are performed so that detector intensities
are independent of the number of neutron histories simulated
(though of course more neutron histories will give better statistics).

As a start, we recommand new \MCS\ users to use the Source\_simple component, then use the Source\_Maxwell\_3 or the Source\_gen.

Optimizers can dramatically improve the statistics, but may occasionally give wrong results, due to misleaded optimization. You should always check such simulations with non-optimized ones.

Other ways to speed-up simulations are to read events from a file. See section \ref{sources-seealso} for possible file formats.

\begin{figure}
  \begin{center}
    \includegraphics[width=0.9\textwidth]{figures/sources.eps}
  \end{center}
\caption{A circular source component (at z=0) emitting neutron events randomly, either from a model, or from a data file.}
\label{f:source}
\end{figure}

\newpage
\section{Source\_simple: A simple continuous source
with a flat energy/wavelength spectrum}
\label{source-simple}
\index{Sources!Source\_simple}

\component{Source\_simple}{System}{ $r_{\rm s}$, $z_{\rm foc}$, $w$, $h$, $E_0$, $\Delta E$, $\Psi$}{$\lambda_0$, $d\lambda$}{Validated, position is the center of disk}

This component is
a simple source with a energy distribution which is uniform
in the range $E_0 \pm dE$
(or wavelength distribution in the range $\lambda_0 \pm d\lambda$).
This component is not used for detailed time-of-flight simulations,
so we put $t=0$ for all neutron rays.

The initial neutron ray position is chosen randomly from within a
circle of radius $r_{\rm s}$ in the $z=0$ plane.
This geometry is a fair approximation
of a cylindrical cold/thermal source with the beam going out along
the cylinder axis.

The initial neutron ray direction is focused within
a solid angle, defined by a rectangular target of width
$w$, height $h$, parallel to
the $xy$ plane placed at $(0,0,z_{\rm foc})$.

The initial weight of the created neutron ray, $p_0$, is set to the
energy-integrated flux, $\Psi$, times the source area, $\pi r_{\rm s}^2$
times a solid-angle factor, which is basically the
solid angle of the focusing rectangle.
See also the discussion on focusing in the \MCS\ Manual \ref{s:focus}.

This component replaces Source\_flat, Source\_flat\_lambda,
Source\_flux and Source\_flux\_lambda.


\newpage
\section{Source\_div: A divergent source}

{\bf Source\_div} is a rectangular source which emits a
beam of a certain divergence around the main exit direction
(the direction of the $z$ axis).
The beam intensity and divergence are uniform over
the whole of the source, and the energy distribution
of the beam is uniform.

This component may be used as a simple model of the
beam profile at the end of a guide or at the sample
position.

The input parameters for Source\_div are the source dimensions
$w$ and $h$ (in m), the divergencies $\delta_h$ and $\delta_v$ (FWHM in degrees), 
and the mean energy $E_0$ and the energy spread $dE$ (both in meV).
The neutron energy range is $(E_0-dE; E_0+dE)$. 



\newpage
\section{Source\_Maxwell\_3: A simple continuous source 
with a Maxwellian spectrum}
\label{source-maxwell}
\index{Sources!Source\_Maxwell\_3}

\component{Source\_Maxwell\_3}{System}{ $h$, $w$, $d_{\rm foc}$, $xw$, $yh$, $\lambda_{\rm low}$, $\lambda_{\rm high}$, $I_1$, $T_1$}{$I_2$, $T_2$, $I_3$, $T_3$}

This component is a source with a Maxwellian energy/wavelength distribution 
sampled in the range $\lambda_{\rm low}$ to $\lambda_{\rm high}$.
This component is not used for detailed time-of-flight simulations,
so $t=0$ for all neutron rays.

The initial neutron ray position is chosen randomly from within a
rectangle of area $h \times w$ in the $z=0$ plane. 

The initial neutron ray direction is focused within
a solid angle, defined by a rectangular target of width
$xw$, height $yh$, parallel to 
the $xy$ plane placed at $(0,0,d_{\rm foc})$. 

The energy distribution used is a sum of 1-3 Maxwellians with
temperatures $T_1$ to $T_3$ and integrated intensities $I_1$ to $I_3$.

The initial weight of the created neutron ray, $\pi_1$, is 
calculated in the following way for one single Maxwellian:
The intensity in a small wavelength interval $(\lambda, \lambda+d\lambda)$ is
$ I_1 M(\lambda,T_1) d\lambda $
where 
$M(\lambda,T_1) = 2 \alpha^2 \exp(-\alpha/\lambda^2) / \lambda^5 $ is the normalized Maxwell distribution ($\alpha=949.0 \rm{K \AA}^2/T_1$).
The number of neutrons per second through a focusing window 
of solid angle $\Omega$
from a source of area $A$ within the wavelength interval $\lambda_1$ to
$\lambda_2$ is thus
\begin{equation}
I_{\rm tot} = \Omega A \int_{\lambda_1}^{\lambda_2} I_1 M(\lambda,T_1) d\lambda.
\end{equation}
In a Monte Carlo integration, the observed intensity becomes
\begin{equation}
I_{\rm MC} \approx N_{\rm MC} \int p(\lambda) \pi_1(\lambda) ,
\end{equation}
where $N_{\rm MC}$ is the number of Monte Carlo steps. 
We here choose the wavelength from a uniform distribution between the two
limits, giving $p(\lambda)=1/(\lambda_{\rm high}-\lambda_{\rm low})$.
To fulfill $I_{\rm tot} = I_{\rm MC}$ we need to have
\begin{equation}
\pi_1(\lambda) = \Omega A \delta\lambda I_1 M(\lambda,T_1) / N_{\rm MC} .
\end{equation}

This expression is strictly valid only for $\Omega \ll 1$, 
see also the discussion on focusing in the \MCS\ Manual \ref{s:focus}. 
The expression is easily generalized to a general number of Maxwellians.


\newpage
\section{Source\_gen: A general continuous source}
\label{source-gen}
\index{Sources!An All-in-One continuous source(Source\_gen)}

\component{Source\_gen}{System, E. Farhi}{$w$, $h$, $xw$, $yh$, $E_0$, $\Delta E$, $T_1$, $T_2$, $T_3$, $I_1$, $I_2$, $I_3$ }{$r$, $\lambda_0$, $d\lambda$, $E_{min}$, $E_{max}$, $\lambda_{min}$, $\lambda_{max}$}{Validated for Maxwellian expressions, position is the center of area}

This component is a continuous neutron source (rectangular or circular), which aims at
a square target centered at the beam (in order to improve MC-acceptance
rate). The angular divergence is then given by the dimensions of the
target. Size may be rectangular (dimension $h$ and $w$, or a disk of radius $r$. The wavelength/energy range to emit is specified either using center and half width, or using minimum and maximum boundaries, alternatively for energy and wavelength.
The flux spectrum is specified with the same Maxwellian parameters as in component Source\_Maxwell\_3 (refer to section \ref{source-maxwell}).

Maxwellian parameters for some sources are in the table given below. For some cases, a correction factor (multiply $I$ parameters) should be used to reach measured data.

\begin{table}
  \begin{center}
  {\let\my=\\
    \begin{tabular}{cccccccc}
    \hline
    Source Name & $T_1$ & $I_1$ & $T_2$ & $I_2$ & $T_3$ & $I_3$ & factor \\
    \hline
    PSI cold source & 150.42 & 3.67e11   & 38.74 & 3.64e11    & 14.84& 0.95e11    &\\
    ILL VCS (H1)    & 216.8  & 1.24e13  & 33.9  & 1.02e13   & 16.7 & 3.0423e12 &\\
    ILL HCS (H5)    & 413.5  & 10.22e12  & 145.8 & 3.44e13    & 40.1 & 2.78e13    & *2\\
    ILL Thermal(H2) & 683.7  & 0.5874e13& 257.7 & 2.5099e13 & 16.7 & 1.0343e12 & /2.25\\
    ILL Hot source  & 1695   & 1.74e13   & 708   & 3.9e12     &      &            &\\
    \end{tabular}
    \caption{Flux parameters for Source\_gen and Source\_Maxwell\_3. ILL cold sources data are obtained from \cite{Ageron89}.}
    \label{t:source-params}
  }
  \end{center}
\end{table}

\newpage
% Emacs settings: -*-mode: latex; TeX-master: "manual.tex"; -*-

\section{Source\_adapt: A neutron source with adaptive importance sampling}
\label{s:Source_adapt}
\label{s:source-adapt}
\index{s:source-adapt}
\index{Optimization}

The {\bf Source\_adapt} component is a neutron source that uses adaptive
importance sampling to improve the efficiency of the simulations. It
works by changing on-the-fly the probability distributions from which
the initial neutron state is sampled so that samples in regions that
contribute much to the accuracy of the overall result are preferred over
samples that contribute little. The method can achieve improvements of a
factor of ten or sometimes several hundred in simulations where only a
small part of the initial phase space contains useful neutrons.
On the other side, a warning is in place here regarding potential wrong results using optimization techniques (see section \ref{s:optim} of the \MCS\ user manual). It is highly recommanded in any case to benchmark 'optimized' simulations against non-optimized ones, checking that obtained results are the same, but hopefully with a much improved statistics.

The physical characteristics of the source are similar to those of
Source\_flat (see section~\ref{sourceaim}). The source is a thin
rectangle in the $X$-$Y$ plane with a flat energy spectrum in a
user-specified range. The flux per area per steradian per
{\AA}ngstr{\o}m per second is specified by the user; the total weight of
neutrons emitted from the source will then be the same irrespectively of
the number of neutron histories simulated, corresponding to one second
of measurements.

The initial neutron weight is given by (see
section~\ref{Source_flux_lambda} for details)
$$ p_0 = \frac{N_{\rm total}}{N_{\rm sim}} =
    \frac{\Phi}{N_{\rm sim}} A \Omega \Delta\lambda $$
Here $\Delta\lambda$ is the total wavelength range of the source; since
the spectrum is flat in energy (but not in wavelength), the flux
will actually be different for different energies. A later version of
this component will probably adapt (in a backward-compatible way) a more
sensible way to specify the flux. For now, an energy or wavelength
monitor (see sections~\ref{s:e_monitor} and~\ref{s:L_monitor}) placed
just after the source will show the actual energy-dependent flux.


\subsection{The adaption algorithm}

The adaptive importance sampling works by subdividing the initial
neutron phase space into a number of equal-sized bins. The division is
done on the three dimensions of energy, horizontal position, and
horizontal divergence, using $N_{\rm eng}$, $N_{\rm pos}$, and $N_{\rm
  div}$ number of bins in each dimension, respectively. The total number
of bins is therefore
$$
N_{\rm bin} = N_{\rm eng} N_{\rm pos} N_{\rm div}
$$
Each bin $i$ is assigned a sampling weight $w_i$; the probability of
emitting a neutron within bin $i$ is
$$
P(i) = \frac{w_i}{\sum_{j=1}^{N_{\rm bin}} w_j}
$$
In order to avoid false learning, the sampling weight of a bin is
kept larger than $w_{\rm min}$, defined as
$$
w_{\rm min} = \frac{\beta}{N_{\rm bin}}\sum_{j=1}^{N_{\rm bin}}w_j,\qquad
    0 \leq \beta \leq 1
$$
This way a (small) fraction $\beta$ of the neutrons are sampled
uniformly from all bins, while the fraction $(1 - \beta)$ are sampled in an adaptive way.

Compared to a uniform sampling of the phase space (where the probability
of each bin is $1/N_{\rm bin}$), the neutron weight
must be adjusted by the amount
$$
\pi_i = \frac{1/N_{\rm bin}}{P(i)} =
    \frac{\sum_{j=1}^{N_{\rm bin}} w_j}{N_{\rm bin} w_i}
$$

In order to set the criteria for adaption, the Adapt\_check component is
used (see section~\ref{s:adapt_check}). The source attemps to sample
only from bins from which neutrons are not absorbed prior to the
position in the instrument at which the Adapt\_check component is
placed. Among those bins, the algorithm attemps to minimize the variance
of the neutron weights at the Adapt\_check position. Thus bins that
would give high weights at the Adapt\_check position are sampled more
often (lowering the weights), while those with low weights are sampled
less often.

Let $\pi = p_1/p_0$ denote the ratio between the neutron weight $p_1$ at
the Adapt\_check position and the initial weight $p_0$ just after the
source. For each bin, the component keeps track of the sum $\psi$ of
$\pi$'s as well as of the total number of neutrons $n_i$ from that
bin. The average weight at the Adapt\_source position of bin $i$ is thus
$\psi_i/n_i$.

We now distribute a total sampling weight of $\beta$ uniformly
among all the bins, and a total weight of $(1 - \beta)$ among bins in
proportion to their average weight $\psi_i/n_i$ at the Adapt\_source
position:
$$
w_i = \frac{\beta}{N_{\rm bin}} +
    (1-\beta) \frac{\psi_i/n_i}{\sum_{j=1}^{N_{\rm bins}} \psi_j/n_j}
$$
After each neutron event originating from bin $i$, the sampling weight $w_i$
is updated.

This basic idea can be improved with a small modification. The problem
is that until the source has had the time to learn the right sampling
weights, neutrons may be emitted with high neutron weights (but low
probability). These low probability neutrons may account for a large fraction of
the total intensity in detectors, causing large variances in the
result. To avoid this, the component emits early neutrons with a lower
weight, and later neutrons with a higher weight to compensate. This way
the neutrons that are emitted with the best adaption contribute the most
to the result.

The factor with which the neutron weights are adjusted is given by a
logistic curve
\begin{equation}
  F(j) = C\frac{y_0}{y_0 + (1 - y_0) e^{-r_0 j}}
\end{equation}
where $j$ is the index of the particular neutron history, $1 \leq j
\leq N_{\rm hist}$. The constants $y_0$, $r_0$, and $C$ are given by
\begin{eqnarray}
  y_0 &=& \frac{2}{N_{\rm bin}} \\
  r_0 &=& \frac{1}{\alpha}\frac{1}{N_{\rm hist}}
     \log\left(\frac{1 - y_0}{y_0}\right) \\
  C &=& 1 + \log\left(y_0 + \frac{1 - y_0}{N_{\rm hist}}
     e^{-r_0 N_{\rm hist}}\right)
\end{eqnarray}
The number $\alpha$ is given by the user and specifies (as a fraction
between zero and one) the point at which the adaption is considered
good. The initial fraction $\alpha$ of neutron histories are emitted
with low weight; the rest are emitted with high weight:
$$ p_0(j) =
    \frac{\Phi}{N_{\rm sim}} A \Omega \Delta\lambda
    \frac{\sum_{j=1}^{N_{\rm bin}} w_j}{N_{\rm bin} w_i}
    F(j)
$$
The choice of the constants $y_0$, $r_0$, and $C$ ensure that
$$
\int_{t=0}^{N_{\rm hist}} F(j) = 1
$$
so that the total intensity over the whole simulation will be correct

Similarly, the adjustment of sampling weights is modified so that the
actual formula used is
$$
w_i(j) = \frac{\beta}{N_{\rm bin}} +
    (1-\beta) \frac{y_0}{y_0 + (1 - y_0) e^{-r_0 j}}
     \frac{\psi_i/n_i}{\sum_{j=1}^{N_{\rm bins}} \psi_j/n_j}
$$

\subsection{The implementation}

\component{Source\_adapt}{K. Nielsen}{$x_{min}$, $x_{max}$, $y_{min}$, $y_{max}$, $E0$, $dE$, dist, $xw$, $yh$, $Phi$}{$\alpha$, $\beta$ (plenty, default values are ok)}{not fully validated}

The heart of the algorithm is a discrete distribution $p$. The
distribution has $N$ \emph{bins}, $1\ldots N$. Each bin has a value
$v_i$; the probability of bin $i$ is then $v_i/(\sum_{j=1}^N v_j)$.

Two basic operations are possible on the distribution. An \emph{update}
adds a number $a$ to a bin, setting $v_i^{\rm new} = v_i^{\rm old} +
a$. A \emph{search} finds, for given input $b$, the minimum $i$ such
that
$$ b \leq \sum_{j=1}^{i} v_j. $$
The search operation is used to sample from the distribution p. If $r$
is a uniformly distributed random number on the interval
$[0;\sum_{j=1}^N v_j]$ then $i = {\rm search}(r)$ is a random number
distributed according to $p$. This is seen from the inequality
$$ \sum_{j=1}^{i-1} v_j < r \leq \sum_{j=1}^{i} v_j, $$
from which $r \in [\sum_{j=1}^{i-1} v_j; v_i + \sum_{j=1}^{i-1} v_j]$
which is an interval of length $v_i$. Hence the probability of $i$ is
$v_i/(\sum_{j=1}^N v_j)$.
The update operation is used to
adapt the distribution to the problem at hand during a simulation. Both
the update and the add operation can be performed very efficiently; how
this is achieved will be described elsewhere.

The input parameters for Source\_adapt are
\textit{xmin}, \textit{xmax}, \textit{ymin}, and
\textit{ymax} in meters to set the source dimensions;
\textit{dist}, \textit{xw}, and \textit{yh}
to set the focusing as for Source\_flat (section~\ref{sourceaim}); \textit{E0} and
\textit{dE} to set the range of energies emitted, in meV (the range
will be from $\textit{E0} - \textit{dE}$ to
$\textit{E0} + \textit{dE}$); flux to set the source flux $\Phi$ in ${\rm
  cm}^{-2} {\rm st}^{-1} \textit{\AA} {\rm s}^{-1}$;
$N_{\rm eng}$, $N_{\rm pos}$, and $N_{\rm
  div}$ to set the number of bins in each dimensions; \textit{alpha} and
\textit{beta} to set the parameters $\alpha$ and $\beta$ as described
above; and \textit{filename} to give the name of a file in which to
output the final sampling destribution.

A good general-purpose value for $\alpha$ and $\beta$ is $\alpha = \beta
= 0.25$. The number of bins to choose will depend on the
application. More bins will allow better adaption of the sampling, but
will require more neutron histories to be simulated before a good
adaption is obtained. The output of the sampling distribution is only
meant for debugging, and the units on the axis are not necessarily
meaningful. Setting the filename to \verb+NULL+ disables the output of
the sampling distribution.

As an alternative, you may use the Source\_Optimizer component (see section \ref{source-optimizer}).


\newpage
\section{Moderator: A time-of-flight source}
\label{s:moderator}
\index{Sources!Time of flight pulsed moderator}

\component{Moderator}{System}{$r_s$, $E_0$, $E_1$, $z_f$, $w$, $h$, $\tau_0$, $E_c$, $\gamma$}{}{}
The simple time-of-flight source component {\bf Moderator} resembles
the source component {\bf Source\_flat} described in \ref{sourceaim}.
Like {\bf Source\_flat}, {\bf Moderator} is circular and focuses
on a rectangular target. Further, the initial velocity is chosen
with a linear distribution within an interval, defined by the
minimum and maximum energies, $E_0$ and $E_1$,
respectively.

The initial time of the neutron is determined on basis of a
simple heuristical model for the time dependence of the
neutron intensity from a time-of-flight source.
For all neutron energies, the flux decay is assumed to be exponential,
\begin{equation}
\Psi(E,t) = \exp(-t/\tau(E)) ,
\end{equation}
where the decay constant is given by
\begin{equation}
\tau(E) = \left\{
\begin{array}{cc}
 \tau_0                               & ; E<E_c \\
 \tau_0 / [ 1 + (E-E_c)^2/\gamma^2 ]  & ; E \geq E_c
\end{array}
\right.
\end{equation}

The input parameters for {\bf Moderator} are the source radius, $r_{\rm s}$,
the minimum and maximum energies, $E_0$ and $E_1$ (in meV),
the distance to the target, $z_{\rm f}$, the dimensions of the target,
$w$ and $h$, and the decay parameters
$\tau_0$ (in $\mu$s), $E_c$, and $\gamma$ (both in meV).

\newpage
\input{Source_Optimizer}

\newpage
% Emacs settings: -*-mode: latex; TeX-master: "manual.tex"; -*-

\subsection{Monitor\_Optimizer: Optimization locations for the
  Sour\-ce\_Op\-ti\-miz\-er component}
\label{s:monitoroptimizer}

This component was contributed by Emmanuel Farhi, Institute
Laue-Langevin.

The {\bf Monitor\_Optimizer} component works with the {\bf
  Source\_Optimizer} component. See section~\ref{s:sourceoptimizer}
for usage.

The input parameters for {\bf Monitor\_Optimizer} are the rectangular
shaped opening coordinates $x_{\rm min}, x_{\rm max}, y_{\rm min}$,
$y_{\rm max}$ (in meters), and the name of the associated instance of
the {\bf
  Source\_Optimizer} component used in the instrument description file (one word,
without quotes).


\newpage
\section{Other sources components}
\label{sources-seealso}

There are other ways to define a source.

Pulsed source components are available for new facilities:\index{Sources!Pulsed sources}
\begin{enumerate}
\item SNS ({\bf contrib/SNS\_source})
\item ISIS ({\bf contrib/ISIS\_moderator})
\item ESS (project) ({\bf ESS\_moderator\_long} and {\bf  ESS\_moderator\_short}).
\end{enumerate}

When no model exists (e.g. not a Maxwellian distribution), you may have access to measurement, estimated flux distributions, event files, and - better - to MCNP/Triploli4 neutron event records. The following components are then for you.

\begin{enumerate}
\item{{\bf misc/Virtual\_input} can read a \MCS\ event file (in text or binary format), often bringing a factor 10 speed-up. See section \ref{virtual_input}.}
\item{{\bf contrib/Virtual\_tripoli4\_input} does the same, but from event files (text format) obtained from the \emph{Tripoli4} \cite{tripoli_webpage} reactor simulation program. Such files are usually huge.\index{Sources!Virtual source from Tripoli4}}
\item{{\bf misc/Vitess\_input} can read \emph{Vitess} \cite{vitess_webpage} neutron event binary files.\index{Sources!Virtual source from Vitess}}
\item{{\bf optics/Filter\_gen} reads a 1D distributionb from a file, and may either modify or set the flux according to it.\index{Sources!from 1D table input}}
\item{A component for reading MCNP "PTRAC" records is planed for a next release. Contact us if you wish to participate.}
\end{enumerate}
