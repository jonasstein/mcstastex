% Emacs settings: -*-mode: latex; TeX-master: "manual.tex"; -*-

\chapter{Monitors and detectors}

In real neutron experiments, detectors and monitors play quite
different roles. One wants the detectors to be as efficient as 
possible, counting all neutrons (absorbing them in the process), 
while the monitors measure the intensity of the incoming beam, and must
as such be almost transparent, interacting only with (roughly) 0.1-1\%
of the neutrons passing by. In computer simulations, it is 
of course possible to detect every neutron without 
absorbing it or disturbing any of its parameters. Hence, the two components
have very similar functions in the simulations, and we do
not distinguish between them. For simplicity, they are from here on
just called monitors, since they do not absorb neutrons.

Another important difference between computer simulations 
and real experiments is
that one may allow the monitor to be sensitive to any neutron property,
as {\em e.g.} direction, energy, and divergence, in addition to what
is found in advanced existing monitors (space and time). One may, in
fact, let the monitor have several of these properties at the same time,
as seen for example in the energy sensitive monitor in
section~\ref{s:e_monitor}.

When a monitor detects a neutron ray, 
a number counting variable is incremented: $n_i = n_{i-1}+1$
In addition, the neutron
weight $p_i$ is added to the weight counting variable:
$I_i = I_{i-1} + p_i$, 
and the second moment of the weight is
updated: $M_{2,i} = M_{2,i-1} + p_i^2$. 
As also discussed in the System Manual, after a simulation of $N$ rays
the detected intensity (in units of neutrons/sec.) is $I_N$,
while the estimated errorbar is $\sqrt{M_{2,i}^2}$.


Many different monitor components have been developed for
\MCS , but we have selected to support only the most important ones.
One example of the monitors we have omitted is the single monitor,
{\bf Monitor},
that measures just one number (with errorbars) per simulation.
This effect is mirrored by any of the 1- or 2-dimensional detectors
we support, e.g. the {\rm PSD\_monitor}. 
In case additional functionality of monitors is required,
the existing monitors can easily be modified.
However, the ultimate solution is the use of the 
``Swiss army knife'' of monitors, {\rm Monitor\_nD}, that can face
almost any simulation challenge. 

\newpage
\section{TOF\_monitor: The time-of-flight monitor}
\component{TOF\_monitor}{System}{$x_{\rm min}$, $x_{\rm max}$, $y_{\rm min}$, $y_{\rm max}$, $n_{\rm chan}$, $t_0$, $t_1$, filename}{$\Delta t$}{}
\index{Monitors!Time-of-flight monitor}

The component {\bf TOF\_monitor} has a rectangular opening
in the $(x,y)$ plane, given by the $x$ and $y$ parameters,
like for {\bf Slit}.
The neutron ray is propagated to the plane of the monitor
by the kernel call PROP\_Z0.
A neutron ray is counted if it passes within the rectangular opening
given by the $x$ and $y$ limits.

Special about {\bf TOF\_monitor} is that it is sensitive to
the arrival time, $t$, of the neutron ray.
Like in a real time-of-flight detector, the time dimension is
binned into small time intervals.
Hence this monitor maintains a one-dimensional histogram of counts.
The $n_{\rm chan}$ time intervals begin at $t_0$ and
end at $t_1$ (alternatively, the interval length is specified by $\Delta t$).
As usual in time-of-flight analysis, all times are given in units of $\mu$s.

The output parameters from {\bf TOF\_monitor} are the three count numbers,
$N, I$, and $M_2$ for the total counts in the monitor.
In addition, a file, \verb+filename+, is produced with a list of
the same three data divided in different TOF bins.
This file can be read and plotted by the {\rm mcplot} tool; see the
System Manual.



\newpage
% Emacs settings: -*-mode: latex; TeX-master: "manual.tex"; -*-

\section{E\_monitor: The energy-sensitive monitor}
\index{Monitors!Energy monitor}
\component{E\_monitor}{System}{$x_{\rm min}$, $x_{\rm max}$, $y_{\rm min}$, $y_{\rm max}$, $n_{\rm chan}$, $E_{\rm min}$, $E_{\rm max}$, filename}{}{}

The component {\bf E\_monitor} resembles {\bf TOF\_monitor}
to a very large extent. Only this monitor is sensitive to
the neutron energy, which in binned in \textit{nchan} bins between
$E_{\rm min}$ and $E_{\rm max}$.

The output parameters from {\bf E\_monitor} are the total counts,
and a file with 1-dimensional data: counts vs. $E$ as for {\bf TOF\_monitor}.




\newpage
% Emacs settings: -*-mode: latex; TeX-master: "manual.tex"; -*-


\section{L\_monitor: The wavelength sensitive monitor}
\label{s:L_monitor}

\component{L\_monitor}{System}{$x_{\rm min}$, $x_{\rm max}$, $y_{\rm min}$, $y_{\rm max}$, $n_{\rm chan}$, $\lambda_{\rm min}$, $\lambda_{\rm max}$, filename}{}{}


The component {\bf L\_monitor} is very similar to
{\bf TOF\_monitor} and {\bf E\_monitor}.
Only is this component sensitive to the neutron wavelength.
The wavelength spectrum is output in a one-dimensional histogram.
between $\lambda_{\rm min}$ and $\lambda_{\rm max}$ (measured in \AA ).

As for the two other 1-dimensional monitors, this component outputs
the total counts and a file with the histogram.



\newpage
% Emacs settings: -*-mode: latex; TeX-master: "manual.tex"; -*-

\section{PSD\_monitor: The PSD monitor}
\component{PSD\_monitor}{System}{$x_{\rm min}$, $x_{\rm max}$, $y_{\rm min}$, $y_{\rm max}$, $n_x$, $n_y$, filename}{}{}


The component {\bf PSD\_monitor} resembles other monitors, e.g.
{\bf E\_Monitor}, and also propagates the neutron to the detector
surface in the $(x,y)$-plane, where the detector window is set
by the $x$ and $y$ input coordinates.
The PSD monitor, though, is not sensitive to the neutron energy, but
rather its position. the rectangular monitor window is divided
into $n_x \times n_y$ pixels, each of which acts like a single
counter.

The output from {\bf PSD\_monitor} is the integrated counts, $n, I, M_2$,
as well as
three two-dimensional arrays of counts: $n(x,y), I(x,y), M_2(x,y)$.
The arrays are written to a file and can be read e.g. by the tool
{\bf MC\_plot}, see the system manual.

{\bf Burde man ikke kunne specificere en radius,
og s\aa\ blev den en 2D cylinder detektor??}

\newpage
% Emacs settings: -*-mode: latex; TeX-master: "manual.tex"; -*-

\section{Divergence\_monitor: A divergence sensitive monitor}
\label{s:div-monitor}
\index{Monitors!Divergence monitor}
\component{Divergence\_monitor}{System}{$x_{\rm min}$, $x_{\rm max}$, $y_{\rm min}$, $y_{\rm max}$, $n_{\rm v}$, $n_{\rm h}$, $\eta_{\rm v,max}$, $\eta_{\rm h,max}$, filename}{}{}


The component {\bf Divergence\_monitor} is a two-dimensional monitor,
which resembles {\bf PSD\_Monitor}. As for this component,
the detector window is set
by the $x$ and $y$ input coordinates.
{\bf Divergence\_ monitor} is sensitive to the neutron divergence,
defined by
$\eta_{\rm h} = \tan^{-1}(v_x/v_z)$ and $\eta_{\rm v} = \tan^{-1}(v_y/v_z)$.
The neutron counts are being histogrammed
into $n_{\rm v} \times n_{\rm h}$ pixels. 
The divergence range accepted is in the vertical direction
$[-\eta_{\rm v,max}; \eta_{\rm v,max}]$, and similar for the horizontal
direction.

The output from {\bf PSD\_monitor} is the integrated counts, $n, I, M_2$,
as well as
three two-dimensional arrays of counts: $n(\eta_{\rm v},\eta_{\rm h}),
I(\eta_{\rm v},\eta_{\rm h}), M_2(\eta_{\rm v},\eta_{\rm h})$.
The arrays are written to a file, \verb+filename+, 
and can be read e.g. by the tool {\bf MC\_plot}, see the system manual.


\newpage
\section{DivPos\_monitor: A divergence and position sensitive monitor}
\label{s:divpos-monitor}
\index{Monitors!Divergence/position monitor}
\component{DivPos\_monitor}{System}{$x_{\rm min}$, $x_{\rm max}$, $y_{\rm min}$, $y_{\rm max}$, $n_x$, $n_{\rm h}$, $\eta_{\rm h,max}$, filename}{}{}

{\bf DivPos\_monitor} is a two-dimensional monitor component,
which is sensitive to both horizontal position ($x$) and horizontal divergence
defined by $\eta_{\rm h} = \tan^{-1}(v_x/v_z)$.
The detector window is set
by the $x$ and $y$ input coordinates.

The neutron counts are being histogrammed
into $n_x \times n_{\rm h}$ pixels. The horizontal divergence range accepted is
$[-\eta_{\rm h,max}; \eta_{\rm h,max}]$, and the horizontal position
range is the size of the detector.

The output from {\bf PSD\_monitor} is the integrated counts, $n, I, M_2$,
as well as
three two-dimensional arrays of counts: $n(x,\eta_{\rm h}),
I(x,\eta_{\rm h}), M_2(x,\eta_{\rm h})$.
The arrays are written to a file and can be read e.g. by the tool
{\bf mcplot}, see the system manual.

This component can be used for measuring acceptance diagrams \cite{Cussen03}.
{\bf PSD\_monitor} can easily be changed into being sensitive
to $y$ and vertical divergence by a 90 degree rotation around the $z$-axis.


%\newpage
%\input{monitor_nd}