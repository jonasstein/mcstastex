\section{Filter: A transmission filter}
A neutron transmission filter act in much of the same way as two
identical slits, one after the other.
The only difference is that the transmission of the filter
varies with the neutron energy.

In the simple component {\bf Filter},
we have not tried to simulate the details of the transmission
process (which includes absorption, incoherent scattering,
and Bragg scattering in a polycrystalline sample, {\em e.g.} Be).
Instead, the transmission is parametrised to be
$\pi_i=T_0$ when $E \leq E_{\rm min}$, $\pi_i=T_1$ when $E \geq E_{\rm max}$,
and linearly interpolated between the two values
in the intermediate interval.
\begin{equation}
\pi_i = \left\{ \begin{array}{lc}
 T_0  & E \leq E_{\rm min} \\
 T_1 + (T_0-T_1) \frac{E_{\rm max}-E}{E_{\rm max}-E_{\rm min}}
 & E_{\rm min} < E < E_{\rm max} \\
 T_1  & E_\geq E_{\rm max} \\
\end{array} \right.
\end{equation}
If $T_1=0$, the neutrons with $E>E_{\rm max}$ are ABSORB'ed.

The input parameters are the four slit coordinates, 
$(x_{\rm min}, x_{\rm max}, y_{\rm min}, y_{\rm max})$,
the distance, $l$, between the slits, and the transmission parameters 
$T_0$, $T_1$, $E_{\rm min}$, and $E_{\rm max}$. 
The energies are given in meV.
