% Emacs settings: -*-mode: latex; TeX-master: "manual.tex"; -*-

\chapter{Libraries and conversion constants}
\label{c:kernelcalls}
\index{Library|textbf}
\index{Library!Shared|see{Library/Components/share}}
\index{Library!mcstas-r|see{Library/Run-time}}

The \MCS\ Library contains a number of built-in functions
and conversion constants which are useful when constructing
components. These are stored in the \verb+share+ directory of 
the \verb+MCSTAS+ library. \index{Library!Components!share} 
\index{Environment variable!MCSTAS}

Within these functions, the 'Run-time' part is available for all
component/instrument descriptions. The other parts (see
table~\ref{t:comp-share}) are dynamic, that is they are not
pre-loaded, but only imported once when a component requests it
using the \verb+%include+ \MCS\ keyword. For instance, within a
component C code block, (usually SHARE or DECLARE):
\index{Keyword!\%include}
\begin{verbatim}
    %include "read_table-lib"
\end{verbatim}
will include the 'read\_table-lib.h' file, and the 'read\_table-lib.c'
(unless the \verb+--no-runtime+ option is used with \verb+mcstas+).
Similarly,
\begin{verbatim}
    %include "read_table-lib.h"
\end{verbatim}
will \emph{only} include the 'read\_table-lib.h'.
The library embedding is done only once for all components (like the
 SHARE section). \index{Keyword!SHARE} For an example of implementation, see the Res\_monitor component.

Here, we present a short list of both each of the library contents and the run-time features.

\section{Run-time calls and functions}
\label{s:calls:run-time}
\index{Library!Run-time|textbf}
\index{Library!mcstas-r|see{Library/Run-time}}
Here we list a number of preprogrammed macros 
which may ease the task of writing component and instrument definitions.

\subsection{Neutron propagation}
\index{Library!Run-time!SCATTER}
\index{Library!Run-time!ABSORB}
\index{Library!Run-time!PROP\_Z0}
\index{Library!Run-time!PROP\_DT}
\index{Library!Run-time!PROP\_GRAV\_DT}
\begin{itemize}
\item {\bf ABSORB}. This macro issues an order to the overall
  \MCS\ simulator to interrupt the simulation of the current neutron
  history and to start a new one.
\item {\bf PROP\_Z0}. Propagates the neutron to the $z=0$ plane,
  by adjusting $(x,y,z)$ and $t$. If the neutron velocity 
  points away from the $z=0$ plane, the neutron is absorbed.
  If component is centered, in order to avoid the neutron to be propagated
  there, use the \verb+_intersect+ functions to determine intersection time(s),
  and then a \verb+PROP_DT+ call.
\item {\bf PROP\_DT}$(dt)$. Propagates the neutron through the
  time interval $dt$, adjusting $(x,y,z)$ and $t$.
\item {\bf PROP\_GRAV\_DT}$(dt,Ax,Ay,Az)$. Like {\bf PROP\_DT}, but it also
  includes gravity using the acceleration $(Ax,Ay,Az)$. In addition,
  to adjusting $(x,y,z)$ and $t$ also $(vx,vy,vz)$ is modified.
\item {\bf SCATTER}. This macro is used to denote a scattering event
  inside a component, see section~\ref{s:comp-trace}. It should be used e.g 
  to indicate that a component has 'done something' (sctattered or detected).
  This does not affect the simulation at all, and is mainly used by the 
  \verb+MCDISPLAY+ section and the \verb+GROUP+ modifier (see~\ref{s:trace} and \ref{s:comp-mcdisplay}). See also the SCATTERED variable (below).
  \index{Keyword!GROUP} \index{Keyword!MCDISPLAY} \index{Keyword!EXTEND}
\end{itemize}

\subsection{Coordinate and component variable retrieval}
\index{Library!Run-time!MC\_GETPAR}
\index{Library!Run-time!NAME\_CURRENT\_COMP}
\index{Library!Run-time!POS\_A\_CURRENT\_COMP}
\index{Library!Run-time!ROT\_A\_CURRENT\_COMP}
\index{Library!Run-time!POS\_A\_COMP}
\index{Library!Run-time!ROT\_A\_COMP}
\index{Library!Run-time!STORE\_NEUTRON}
\index{Library!Run-time!RESTORE\_NEUTRON}
\index{Library!Run-time!SCATTERED}
\begin{itemize}
\item {\bf MC\_GETPAR}(). This may be used in the finally section of an
  instrument definition to reference the output parameters of a
  component. See page~\pageref{mcgetpar} for details.
\item {\bf NAME\_CURRENT\_COMP} gives the name of the current component as a string.
\item {\bf POS\_A\_CURRENT\_COMP} gives the absolute position of the 
  current component. A component of the vector is referred to as
  POS\_A\_CURRENT\_COMP.$i$ where $i$ is $x$, $y$ or $z$.
\item {\bf ROT\_A\_CURRENT\_COMP} and 
  {\bf ROT\_R\_CURRENT\_COMP} give the orientation
  of the current component as rotation matrices
  (absolute orientation and the orientation relative to
  the previous component, respectively). A
  component of a rotation matrice is referred to as 
  ROT\_A\_CURRENT\_COMP$[m][n]$, where $m$ and
  $n$ are 0, 1, or 2.
\item {\bf POS\_A\_COMP}$(comp)$ gives the absolute position
  of the component with the name {\em comp}. Note that
  {\em comp} is not given as a string. A component of the
  vector is referred to as POS\_A\_COMP$(comp).i$
  where $i$ is $x$, $y$ or $z$.
\item {\bf ROT\_A\_COMP}$(comp)$ and
  {\bf ROT\_R\_COMP}$(comp)$ give the orientation of the
  component {\em comp} as rotation matrices (absolute
  orientation and the orientation relative to its
  previous component, respectively). Note that {\em comp}
  is not given as a string. A component of  a rotation
  matrice is referred to as 
  ROT\_A\_COMP$(comp)[m][n]$, where $m$ and $n$ are
  0, 1, or 2. 
\item {\bf INDEX\_CURRENT\_COMP} is the number (index) of the
       current component  (starting from 1).
\item {\bf POS\_A\_COMP\_INDEX}$(index)$ is the absolute position of
  component $index$. \\
  POS\_A\_COMP\_INDEX (INDEX\_CURRENT\_COMP) is the same as \\
  POS\_A\_CURRENT\_COMP. You may use
  POS\_A\_COMP\_INDEX \\ (INDEX\_CURRENT\_COMP+1) to make, for instance, your
  component access the position of the next component (this is usefull for
  automatic targeting).  A component of the vector is referred to as \\
  POS\_A\_COMP\_INDEX$(index).i$ where $i$ is $x$, $y$ or $z$. {\bf
  POS\_R\_COMP\_INDEX} works the same, but with relative coordinates. 
\item {\bf STORE\_NEUTRON}$(index, x, y, z, vx, vy, vz, t$, $sx, sy,
sz, p)$ stores the current neutron state in the trace-history table,
in local coordinate system. $index$ is usually INDEX\_CURRENT\_COMP.
This is automatically done when entering each component of an
instrument.
\item {\bf RESTORE\_NEUTRON}$(index, x, y, z, vx, vy, vz, t, sx, sy,
sz, p)$ restores the neutron state to the one at the input of the
component $index$. To ignore a component effect, use
RESTORE\_NEUTRON (INDEX\_CURRENT\_COMP, \\ 
$x, y, z, vx, vy, vz, t,
sx, sy, sz, p$) at the end of its TRACE section, or in its EXTEND
section. These neutron states are in the local component coordinate
systems.
\item {\bf SCATTERED} is a variable set to 0 when entering
  a component, which is incremented each time a SCATTER event occurs.
  This may be used in the \verb+EXTEND+ sections (always executed when
  existing) to branch
  action depending if the component acted or not on the current neutron.
\item {\bf extend\_list}($n$, \&\textit{arr}, \&\textit{len},
  \textit{elemsize}). Given an array \textit{arr} with \textit{len}
  elements each of size \textit{elemsize}, make sure that the array is
  big enough to hold at least $n$ elements, by extending \textit{arr}
  and \textit{len} if necessary. Typically used when reading a list of
  numbers from a data file when the length of the file is not known in advance.
\item {\bf mcset\_ncount}$(n)$. Sets the number of neutron histories to simulate to $n$.
\item {\bf mcget\_ncount}(). Returns the number of neutron histories to simulate (usually set by option \verb+-n+).
\item {\bf mcget\_run\_num}(). Returns the number of neutron histories that have been simulated until now.
\end{itemize}

\subsection{Coordinate transformations}
\begin{itemize}
\item {\bf coords\_set}$(x,y,z)$ returns a Coord structure (like POS\_A\_CURRENT\_COMP) with $x$, $y$ and $z$ members.
\item {\bf  coords\_get}$(P,$ \&$x$, \&$y$, \&$z)$ copies the $x$, $y$ and 
$z$ members of the Coord structure $P$ into $x,y,z$ variables.
\item {\bf coords\_add}$(a,b)$, {\bf coords\_sub}$(a,b)$, {\bf
coords\_neg}$(a)$ enable to  operate on coordinates, and return the
resulting Coord structure.
\item {\bf rot\_set\_rotation}({\it Rotation t}, $\phi_x, \phi_y, \phi_z$) Get transformation for rotation first $\phi_x$ around x axis, then $\phi_y$ around y, then $\phi_z$ around z. $t$ should be a 'Rotation' ([3][3] 'double' matrix).
\item {\bf rot\_mul}{\it (Rotation t1, Rotation t2, Rotation t3)} performs $t3 = t1 . t2$.
\item {\bf rot\_copy}{\it (Rotation dest, Rotation src)} performs $dest = src$ for Rotation arrays.
\item {\bf rot\_transpose}{\it (Rotation src, Rotation dest)} performs $dest = src^t$.
\item {\bf rot\_apply}{\it (Rotation t, Coords a)} returns a Coord structure which is $t.a$
\end{itemize}

\subsection{Mathematical routines}
\begin{itemize}
\item {\bf NORM}$(x,y,z)$. Normalizes the vector $(x,y,z)$ to have
  length 1.
\item {\bf scalar\_prod}$(a_x,a_y,a_z,b_x,b_y,b_z)$. Returns the scalar
  product of the two vectors $(a_x,a_y,a_z)$ and $(b_x,b_y,b_z)$.
\item {\bf vecprod}$(a_x,a_y,a_z,b_x,b_y,b_z, c_x,c_y,c_z)$. Sets
  $(a_x,a_y,a_z)$ equal to the vector product $(b_x,b_y,b_z) \times (c_x,c_y,c_z)$.
\item {\bf rotate}$(x,y,z,v_x,v_y,v_z,\varphi,a_x,a_y,a_z)$. Set
  $(x,y,z)$ to the result of rotating the vector $(v_x,v_y,v_z)$
  the angle $\varphi$ (in radians) around the vector $(a_x,a_y,a_z)$.
\item {\bf normal\_vec}(\&$n_x$, \&$n_y$, \&$n_z$, $x$, $y$, $z$).
  Computes a unit vector $(n_x, n_y, n_z)$ normal to the vector
  $(x,y,z)$.
\end{itemize}

\subsection{Output from detectors}
\begin{itemize}
\item {\bf DETECTOR\_OUT\_0D}$(...)$. Used to output the results from a
  single detector. The name of the detector is output together
  with the simulated intensity and estimated statistical error. The
  output is produced in a format that can be read by \MCS\ front-end
  programs. See section~\ref{s:comp-finally} for details.
\item {\bf DETECTOR\_OUT\_1D}$(...)$. Used to output the results from a
  one-dimentional detector. See section~\ref{s:comp-finally} for details.
\item {\bf DETECTOR\_OUT\_2D}$(...)$. Used to output the results from a
  two-dimentional detector. See section~\ref{s:comp-finally} for details.
\item {\bf DETECTOR\_OUT\_3D}$(...)$. Used to output
  the results from a three-dimentional detector. Arguments are the same as
  in DETECTOR\_OUT\_2D, but with the additional $z$ axis (the signal).
  Resulting data files are treated as 2D data, but the 3rd dimension is
  specified in the $type$ field.
\item {\bf mcheader\_out}{\it (FILE *f,char *parent,
  int m, int n, int p,
  char *xlabel, char *ylabel, char *zlabel, char *title,
  char *xvar, char *yvar, char *zvar,
  double x1, double x2, double y1,  double y2, double z1, double z2,
  char *filename)} appends a header file using the current data format setting. Signification of parameters may be found in section~\ref{s:comp-finally}. Please contact the authors in case of perplexity.
\item {\bf mcinfo\_simulation}{\it (FILE *f, mcformat, 
  char *pre, char *name)} is used to append the simulation parameters into file $f$ (see for instance the Res\_monitor component). Internal variable $mcformat$ should be used as specified. Please contact the authors in case of perplexity.
\end{itemize}

\subsection{Ray-geometry intersections}
\begin{itemize}
\item {\bf box\_intersect}(\&$t_1$, \&$t_2$, $x$, $y$, $z$, $v_x$, $v_y$, $v_z$,
  $d_x$, $d_y$, $d_z$). Calculates the (0, 1, or 2) intersections between
  the neutron path and a box of dimensions $d_x$, $d_y$, and $d_z$,
  centered at the origin for a neutron with the parameters
  $(x,y,z,v_x,v_y,v_z)$. The times of intersection are returned
  in the variables $t_1$ and $t_2$, with $t_1 < t_2$. In the case
  of less than two intersections, $t_1$ (and possibly $t_2$) are set to
  zero. The function returns true if the neutron intersects the box,
  false otherwise.
\item {\bf cylinder\_intersect}(\&$t_1$, \&$t_2$, $x$, $y$, $z$, $v_x$, $v_y$, $v_z$,
  $r$, $h$).  Similar to {\bf box\_intersect}, but using a cylinder of height $h$ and radius $r$,
  centered at the origin.
\item {\bf sphere\_intersect}(\&$t_1$, \&$t_2$, $x$, $y$, $z$, $v_x$, $v_y$, $v_z$,
  $r$). Similar to {\bf box\_intersect}, but using a sphere
  of radius $r$. 
\end{itemize}

\subsection{Random numbers}
\begin{itemize}
\item {\bf rand01}(). Returns a random number distributed uniformly between 0 and 1.
\item {\bf randnorm}(). Returns a random number from a normal
  distribution centered around 0 and with $\sigma=1$. The algorithm used to
  get the normal distribution is explained in~\cite{num_rep}, chapter~7.
\item {\bf randpm1}(). Returns a random number distributed uniformly between -1 and 1.
\item {\bf randvec\_target\_circle}(\&$v_x$, \&$v_y$, \&$v_z$, \&$d\Omega$,
  aim$_x$, aim$_y$, aim$_z$, $r_f$). Generates a random vector $(v_x, v_y,
  v_z)$, of the same length as (aim$_x$, aim$_y$, aim$_z$), which is
  targeted at a \emph{disk} centered at (aim$_x$, aim$_y$, aim$_z$) with
  radius $r_f$ (in meters), and perpendicular to the \emph{aim} vector.. All directions
  that intersect the sphere are chosen with equal probability. The solid
  angle of the sphere as seen from the position of the neutron is returned
  in $d\Omega$. This routine was previously called {\bf randvec\_target\_sphere}
  (which still works).
\item {\bf randvec\_target\_rect\_angular}(\&$v_x$, \&$v_y$, \&$v_z$, 
  \&$d\Omega$, aim$_x$, aim$_y$, aim$_z$,$height, width, Rot$) does the same as
  randvec\_target\_circle but targetting at a rectangle with angular dimensions
  $height$ and $width$ (in {\bf radians}, not in degrees as other angles). The
  rotation matrix $Rot$ is the coordinate system orientation in the absolute
  frame, usually ROT\_A\_CURRENT\_COMP.
\item {\bf randvec\_target\_rect}(\&$v_x$, \&$v_y$, \&$v_z$, 
  \&$d\Omega$, aim$_x$, aim$_y$, aim$_z$,$height, width, Rot$) is the same as 
  randvec\_target\_rect\_angular but $height$ and $width$ dimensions are given
  in meters. This function is useful to target at a guide entry window.
\end{itemize}

\section{Reading a data file into a vector/matrix (Table input)}
\index{Library!read\_table-lib (Read\_Table)|textbf}
The \verb+read_table-lib+ provides functionalities for reading text (and binary) data files. To use this library, add a \verb+%include "read_table-lib"+ in your component definition DECLARE or SHARE section. Available functions are:

\begin{itemize}
\item {\bf Table\_Init}(\&$Table$) and {\bf Table\_Free}(\&$Table$) initialize and free allocated memory blocks
\item {\bf Table\_Read}(\&$Table$, $filename$, $block$) reads numerical block number $block$ (0 for all) data from \emph{text} file $filename$ into $Table$. The block number changes when the numerical data changes its size, or a comment is encoutered (lines starting by '\verb+# ; % /+'). If the data could not be read, then $Table.data$ is NULL and $Table.rows = 0$. You may then try to read it using Table\_Read\_Offset\_Binary.
\item {\bf Table\_Rebin}(\&$Table$) rebins $Table$ rows with increasing, evenly spaced first column (index 0), e.g. before using Table\_Value.
\item {\bf Table\_Read\_Offset}(\&$Table$, $filename$, $block$, \&$offset$, $n_{rows}$) does the same as Table\_Read except that it starts at offset $offset$ (0 means begining of file) and reads $n_{rows}$ lines (0 for all). The $offset$ is returned as the final offset reached after reading the $n_{rows}$ lines.
\item {\bf Table\_Read\_Offset\_Binary}(\&$Table$, $filename$, $type$,
  $block$, \&$offset$, $n_{rows}$, $n_{columns}$) does the same as
  Table\_Read\_Offset, but also specifies the $type$ of the file (may
  be "float" or "double"), the number $n_{rows}$ of rows to read, each
  of them having $n_{columns}$ elements. No text header should be present
  in the file.
\item {\bf Table\_Info}$(Table)$ print information about the table $Table$.
\item {\bf Table\_Index}($Table, m, n$) reads the $Table[m][n]$ element.
\item {\bf Table\_Value}($Table, x, n$) looks for the closest $x$
  value in the first column (index 0), and extracts in this row the
  $n$-th element (starting from 0). The first column is thus the 'x' axis for the data.
\end{itemize}

The format of text files is free. Lines starting by '\verb+# ; % /+' characters are considered to be comments. Data blocks are vectors and matrices. Block numbers are counted starting from 1, and changing when a comment is found, or the column number changes. For instance, the file 'MCSTAS/data/BeO.trm' (Transmission of a Berylium filter) looks like:
\begin{verbatim}
  # BeO transmission, as measured on IN12
  # Thickness: 0.05 [m]
  # [ k(Angs-1) Transmission (0-1) ]
  # wavevector multiply
  1.0500  0.74441
  1.0750  0.76727
  1.1000  0.80680
  ...
\end{verbatim}
Binary files should be of type "float" (i.e. REAL*32) and "double" (i.e. REAL*64), and should \emph{not} contain text header lines. These files are plateform dependent (little or big endian).

The $filename$ is first searched into the current directory (and all user additional locations specified using the \verb+-I+ option, see section~\ref{s:files}), and if not found, in the \verb+data+ sub-directory of the \verb+MCSTAS+ library location. \index{Library!Components!data}
\index{Environment variable!MCSTAS} This way, you do not need to have local copies of the \MCS\ Library Data files (see table~\ref{t:comp-data}).

A usage example for this library part may be:
\begin{verbatim}
    t_Table rTable;       % declares a t_Table structure
    char file="BeO.trm";  % a file name
    double x,y;
    
    Table_Init(&rTable);  % initialize the table to empty state
    Table_Read(&rTable, file, 1); % reads the first numerical block
    Table_Info(rTable);   % display table informations
    ...
    x = Table_Index(rTable, 2,5);  % reads the 3rd row, 6th column element 
                                   % of the table. Indexes start at zero in C.
    y = Table_Value(rTable, 1.45,1);  % looks for value 1.45 in 1st column (x axis)
                                      % and extract 2nd column value of that row
    Table_Free(&rTable);  % free allocated memory for table
    
\end{verbatim}
Additionally, if the block number (3rd) argument of  {\bf Table\_Read} is 0, all blocks will be catenated.
The {\bf Table\_Value} function assumes that the 'x' axis is the first column (index 0).
Other functions are used the same way with a few additional parameters, e.g. specifying an offset for reading files, or reading binary data.

You may look into, for instance, the Monochromator\_curved component, or the Virtual\_input component for other implementation examples.

\section{Monitor\_nD Library}
\index{Library!monitor\_nd-lib}

This library gathers a few functions used by a set of monitors e.g. Monitor\_nD, Res\_monitor, Virtual\_output, \ldots.
It may monitor any kind of data, create the data files, and may display many geometries (for \verb+mcdisplay+).
Refer to these components for implementation examples, and ask the authors for more details.

\section{Adaptative importance sampling Library}
\index{Library!adapt\_tree-lib}

This library is currently only used by the components Source\_adapt
and Adapt\_check. It performs adaptative importance sampling of neutrons for simulation efficiency optimization.
Refer to these components for implementation examples, and ask the authors for more details.

\section{Vitess import/export Library}
\index{Library!vitess-lib}

This library is used by Vitess\_input, Vitess\_output components, as well as the \verb+mcstas2vitess+ utility (see section~\ref{s:mcstas2vitess}). \index{Tools!mcstas2vitess}
Refer to these components for implementation examples, and ask the authors for more details.

\section{Constants for unit conversion etc.}
The following predefined constants are useful for conversion
between units
\def\textvb{\textbf}
\begin{center}
\begin{tabular}{|l|c|p{0.29\textwidth}|p{0.252\textwidth}|}
\hline
Name & Value & Conversion from & Conversion to \\ \hline
\textvb{DEG2RAD} & $2 \pi / 360$ & Degrees & radians \\
\textvb{RAD2DEG} & $360 / (2 \pi)$ & Radians & degrees \\
\textvb{MIN2RAD} & $2 \pi / (360 \cdot 60)$ 
  & Minutes of arc & radians \\
\textvb{RAD2MIN} & $(360\cdot 60) / (2 \pi)$ 
  & Radians & minutes of arc \\
\textvb{V2K} & $10^{10} \cdot m_{\rm N}/\hbar$ 
  & Velocity (m/s) & {\bf k}-vector (\AA$^{-1}$) \\ 
\textvb{K2V} & $10^{-10} \cdot \hbar / m_{\rm N}$ 
  & {\bf k}-vector (\AA$^{-1}$) & Velocity (m/s) \\
\textvb{VS2E} & $m_{\rm N} / (2 e)$
  & Velocity squared (m$^2$ s$^{-2}$) & Neutron energy (meV) \\
\textvb{SE2V} & $\sqrt{2 e/m_{\rm N}}$ 
  & Square root of neutron energy (meV$^{1/2}$) & Velocity (m/s) \\
\textvb{FWHM2RMS} & $1/\sqrt{8\log(2)}$ 
  & Full width half maximum & Root mean square (standard deviation) \\
\textvb{RMS2FWHM} & $\sqrt{8\log(2)}$ 
  & Root mean square (standard deviation) & Full width half maximum \\
\textvb{MNEUTRON} & $1.67492E-27 kg$ 
  & Neutron mass & \\
\textvb{HBAR} & $1.05459E-34 Js$ 
  & Planck constant & \\
\textvb{PI} & $3.14159265358979323846$ 
  & $\pi$ & \\
\hline
\end{tabular}
\end{center}

Further, we have defined the constants \textvb{PI}$=\pi$ and \textvb{HBAR}$=\hbar$.
