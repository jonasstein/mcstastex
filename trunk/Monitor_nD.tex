\section{Monitor\_nD: A general Monitor for 0D/1D/2D records}
\label{s:monitornd}

\component{Monitor\_nD}{System}{$x_{\rm min}$, $x_{\rm max}$, $y_{\rm min}$, $y_{\rm max}$, options}{$file$, $x_{width}, y_{height}, z_{thick}$, $bins$, $min$, $max$}{}

The component {\bf Monitor\_nD} is a general Monitor that may output any
set of physical parameters regarding the passing neutrons. The
generated files are either a set of 1D signals ([Intensity] {\it vs.}
[Variable]), or a single 2D signal ([Intensity] {\it vs.} [Variable 1]
{\it vs.} [Variable 1]), and possibly a simple long list of the selected
physical parameters for each neutron.

The input parameters for {\bf Monitor\_nD} are its dimensions $x_{\rm
  min}, x_{\rm max}, y_{\rm min}$, $y_{\rm max}$ (in meters) and an {\it
  options} string describing what to detect, and what to do with the
signals, in clear language. The $x_{width}, y_{height}, z_{thick}$ may also be used to enter dimensions.

The formatting of the {\it options}
parameter is free, as long as it contains some specific keywords, that
can be sometimes followed by values. The {\it no} or {\it not} option
modifier will revert next option. The {\it all} option can also affect a
set of monitor configuration parameters (see below).

As the usage of this component ebnables to monitor virtually anything, and thus the combinations of options and parameters is infinite, we shall only present the most basic configuration. The reader should refer to the on-line component help, using e.g. \verb+mcdoc Monitor_nD.comp+.

\subsubsection{The Monitor\_nD geometry}

The monitor shape can be selected among six geometries:
\begin{enumerate}
\item{({\it square}) The default geometry is flat rectangular in ($xy$)
    plane with dimensions $x_{\rm min}, x_{\rm max}, y_{\rm min}$,
    $y_{\rm max}$.}
\item{({\it box})A rectangular box with dimensions $x_{width}, y_{height}, z_{thick}$.}
\item{({\it disk}) When choosing this geometry, the detector is a flat
    disk in ($xy$) plane. The radius is then
    \begin{equation}
      \mbox{radius} = \max ( \mbox{abs } [ x_{\rm min}, x_{\rm max}, y_{\rm
        min}, y_{\rm max} ] ).
    \end{equation}
    }
\item{({\it sphere}) The detector is a sphere with the same radius as
    for the {\it disk} geometry.}
\item{({\it cylinder}) The detector is a cylinder with revolution axis
    along $y$ (vertical). The radius in ($xz$) plane is
    \begin{equation}
      \mbox{radius} =  \max ( \mbox{abs } [ x_{\rm min}, x_{\rm max} ] ),
    \end{equation}
    and the height along $y$ is
    \begin{equation}
      \mbox{height} =  | y_{\rm max} - y_{\rm max} |.
    \end{equation}
    }
\item{({\it banana}) The same as the cylinder, but without the top/bottom caps, and on a restricted angular range. The angular range is specified using a \verb+theta+ variable limit specification in the \verb+options+.}
\end{enumerate}

By default, the monitor is flat, rectangular. Of course, you can choose
the orientation of the {\bf Monitor\_nD} in the instrument description
file with the usual \texttt{ROTATED} modifier.

For the {\it box}, {\it sphere} and {\it cylinder}, the outgoing neutrons are
monitored by default, but you can choose to monitor incoming neutron
with the {\it incoming} option.

At last, the {\it slit} or {\it absorb} option will ask the component to
absorb the neutrons that do not intersect the monitor. The {\it exclusive} option word removes neutrons which are similarly outside the monitor limits (that may be other than geometrical).

\subsubsection{The neutron parameters that can be monitored}

There are 25 different variables that can be monitored at the same time
and position. Some can have more than one name (e.g. \texttt{energy} or
\texttt{omega}).


\begin{verbatim}
    kx ky kz k wavevector [Angs-1] (    usually axis are
    vx vy vz v            [m/s]         x=horz., y=vert., z=on axis)
    x y z                 [m]      Distance, Position
    kxy vxy xy radius     [m]      Radial wavevector, velocity and position
    t time                [s]      Time of Flight
    energy omega          [meV]
    lambda wavelength     [Angs]
    p intensity flux      [n/s] or [n/cm^2/s]
    ncounts               [1]
    sx sy sz              [1]      Spin
    vdiv ydiv dy          [deg]    vertical divergence (y)
    hdiv divergence xdiv  [deg]    horizontal divergence (x)
    angle                 [deg]    divergence from  direction
    theta longitude       [deg]    longitude (x/z) [for sphere and cylinder]
    phi   lattitude       [deg]    lattitude (y/z) [for sphere and cylinder]
\end{verbatim}
as well as two other special variables
\begin{verbatim}
    user user1            will monitor the [Mon_Name]_Vars.UserVariable{1|2}
    user2                 to be assigned in an other component (see below)
\end{verbatim}

To tell the component what you want to monitor, just add the variable
names in the {\it options} parameter. The data will be sorted into {\it
  bins} cells (default is 20), between some default {\it limits}, that
can also be set by user. The {\it auto} option will automatically
determine what limits should be used to have a good sampling of signals.

The {\it borders} option will monitor variables that are outside
the limits. These values are then accumulated on the 'borders' of the
signal.

Each monitoring will record the flux (sum of weights $p$) versus the
given variables. The {\it cm2} option will ask to normalize the flux
to the monitor section surface.

Some examples ?
\begin{enumerate}
\item{\texttt{options="x bins=30 limits=[-0.05 0.05] ; y"} \\
    will set the monitor to look at $x$ and $y$. For $y$, default bins
    and limits values (monitor dimensions) are used.}
\item{\texttt{options="x y, all bins=30, all limits=[-0.05 0.05]"} \\
    will do the same, but set limits and bins for $x$ and $y$.}
\item{\texttt{options="x y, auto limits"} \\
    will determine itself the required limits for $x$ and $y$ to monitor
    passing neutrons with default {\it bins}=20.}
\end{enumerate}

\subsubsection{The output files}

By default, the file names will be the component name, followed by
automatic extensions showing what was monitored (such as
\texttt{MyMonitor.x}). You can also set the filename in {\it options}
with the {\it file} keyword followed by the file name that you want. The
extension will then be added if the name does not contain a dot (.).
Finally, the $filename$ parameter may also be used.

The output files format are standard 1D or 2D McStas detector files.
The {\it no file} option will {\it unactivate} monitor, and make it a
single 0D monitor detecting integrated flux and counts.
The {\it verbose} option will display the nature of the monitor, and the
names of the generated files.

\subsubsection{The 2D output}

When you ask the {\bf Monitor\_nD} to monitor only two variables (e.g.
{\it options} = "x y"), a single 2D file of intensity versus these two
correlated variables will be created.

\subsubsection{The 1D output}

The {\bf Monitor\_nD} can produce a set of 1D files, one for each
monitored variable, when using 1 or more than 2 variables, or when
specifying the {\it multiple} keyword option.

\subsubsection{The List output}

The {\bf Monitor\_nD} can additionally produce a {\it list} of variable
values for neutrons that pass into the monitor. This feature is additive
to the 1D or 2D output. By default only 1000 events will be recorded in
the file, but you can specify for instance "{\it list} 3000 neutrons" or
"{\it list all} neutrons". This last option might require a lot of
memory and generate huge files.

\subsubsection{Monitor equivalences}

In the following table, we show how the Monitor\_nD may substitute any other \MCS\ monitor.

\begin{table}
  \begin{center}
    {\let\my=\\
    \begin{tabular}{|p{0.24\textwidth}|p{0.7\textwidth}|}
\hline
\MCS\ monitor & Monitor\_nD equivalent \\
\hline
Divergence\_monitor & {\it options}="dx bins=$ndiv$ limits=[$-\alpha/2 \alpha/2$],
                                lambda bins=$nlam$ limits=[$\lambda_0$ $\lambda_1$] file=$file$"\\
DivLambda\_monitor  & {\it options}="dx bins=$nh$   limits=[$-h_{max}/2 h_{max}/2$],
                                    dy bins=$nv$   limits=[$-v_{max}/2 v_{max}/2$]" {\it filename}=$file$\\
DivPos\_monitor     & {\it options}="dx bins=$ndiv$ limits=[$-\alpha/2 \alpha/2$],
                                     x bins=$npos$" {\it xmin}=$x_{min}$ {\it xmax}=$x_{max}$ \\
E\_monitor          & {\it options}="energy bins=$nchan$ limits=[$E_{min} E_{max}$]" \\
EPSD\_monitor       & {\it options}="energy bins=$n_E$ limits=[$E_{min} E_{max}$], x bins=$nx$"
                              {\it xmin}=$x_{min}$ {\it xmax}=$x_{max}$ \\
Hdiv\_monitor       & {\it options}="dx bins=$nh$ limits=[$-h_{max}/2 h_{max}/2$]" {\it filename}=$file$ \\
L\_monitor          & {\it options}="lambda bins=$nh$ limits=[$-\lambda_{max}/2 \lambda_{max}/2$]" {\it filename}=$file$ \\
Monitor\_4PI        & {\it options}="sphere" \\
Monitor            & {\it options}="unactivate" \\
PSDcyl\_monitor     & {\it options}="theta bins=$nr$,y bins=$ny$, cylinder"
{\it filename}=$file$ {\it yheight}=$height$ {\it xwidth}=2*radius\\
PSDlin\_monitor     & {\it options}="x bins=$nx$" {\it xmin}=$x_{min}$ {\it xmax}=$x_{max}$ {\it ymin}=$y_{min}$ {\it ymax}=$y_{max}$ {\it filename}=$file$\\
PSD\_monitor\_4PI    & {\it options}="x y, sphere" \\
PSD\_monitor        & {\it options}="x bins=$nx$, y bins=$ny$" {\it xmin}=$x_{min}$ {\it xmax}=$x_{max}$ {\it ymin}=$y_{min}$ {\it ymax}=$y_{max}$ {\it filename}=$file$\\
TOF\_cylPSD\_monitor & {\it options}="theta bins=$n_\phi$, time bins=$nt$ limits=[$t_0, t_1$], cylinder" {\it filename}=$file$ {\it yheight}=$height$ {\it xwidth}=2*radius\\
TOFLambda\_monitor  & {\it options}="lambda bins=$n_\lambda$ limits=[$\lambda_0$ $\lambda_1$], time bins=$nt$ limits=[$t_0, t_1$]" {\it filename}=$file$\\
TOFlog\_mon         & {\it options}="log time bins=$nt$ limits=[$t_0, t_1$]" \\
TOF\_monitor        & {\it options}="time bins=$nt$ limits=[$t_0, t_1$]" \\
\hline
    \end{tabular}
    \caption{Using Monitor\_nD in place of other components. All limits specifications may be advantageously replaced by an {\it auto} word preceeding each monitored variable. Not all file and dimension specifications are indicated (e.g. filename, xmin, xmax, ymin, ymax).}
    \label{t:monitor-nd-equiv}
    }
  \end{center}
\end{table}

\subsubsection{Other usage examples}

\begin{itemize}
\item{
\begin{verbatim}
COMPONENT MyMonitor = Monitor_nD(
    xmin = -0.1, xmax = 0.1,
    ymin = -0.1, ymax = 0.1,
    options = "energy auto limits")
\end{verbatim}
will monitor the neutron energy in a single 1D file (a kind of E\_monitor)}
\item{\texttt{{\it options}="x y, all bins=50, all auto"} \\
will monitor the neutron $x$ and $y$ in a single 2D file (same as PSD\_monitor)}

\item{\texttt{{\it options}="multiple x bins=30, y limits=[-0.05 0.05], all auto"} \\
will monitor the neutron $x$ and $y$ in two 1D files}
\item{\texttt{{\it options}="x y z kx ky kz, all auto"} \\
will monitor theses variables in six 1D files}
\item{\texttt{{\it options}="x y z kx ky kz, list all, all auto"} \\
will monitor all theses neutron variables in one long list}
\item{\texttt{{\it options}="multiple x y z kx ky kz, and list 2000, all auto"} \\
    will monitor all theses neutron variables in one list of 2000 events
    and in six 1D files}
\end{itemize}
