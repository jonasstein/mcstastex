\chapter{Special-purpose components}
\index{Library!Components!misc}

The chapter deals with components that are not easily included
in any of the other chapters because of their special nature,
but which still are considered to be of significant importance
to \MCS .

One part of these components deals with splitting simulations
into two (or more) stages. For example, a guide system is often
not changed much, and a long simulation of neutron rays
``surviving'' through the guide system could be reused
for several simulations of the instrument back-end, speeding up
the simulations by (typically) one or two orders of magnitude.
The components for doing this trick is {\bf Virtual\_input} and
{\bf Virtual\_output}, which stores and reads neutron rays, respectively.

Another pair of components performs the simulation of the instrument
resolution functions. This is {\bf Res\_sample}, that is to be
placed at the sample position, and {\bf Res\_monitor}, that should
be localized at the position of the instrument detector.

\newpage
\section{Vitual\_input: Starting the second part of a split simulation}
\label{virtual_input}

\component{Virtual\_input}{System}{filename}{buffer size, repeat count, type}

The component {\bf Virtual\_input} resumes a split simulation where the 
first part has been performed by another instrument and the neutron ray
parameters have been stored by the component {\bf Virtual\_output}.

All neutron ray parameters are read from the input file, which is by default
of ``text'' type, but can also assume the binary formats 
``float'' or ``double''. The reading of neutron rays continue until the 
specified number of rays have been simulated or 
till the file has been exhausted. If desirable, the input file
can be reused a number of times, determined by the optional parameter
``repeat count''. However, care should be taken when dealing with 
absolute intensities, which are automatically correct only
when the input file has been exhausted once.

For performance reasons, it is possible to specify the size of 
the buffer to use in reading the neutron rays from file. Default 
is to buffer the whole file, but this may be fatal for very large
input files.

\section{Vitual\_output: Saving the first part of a split simulation}
\label{virtual_output}

\component{Virtual\_output}{System}{filename}{buffer size, type}

The component {\bf Virtual\_output} stores the neutron ray parameters
from a split simulation. It is then possible to let the 
next part of the simulation be performed by another instrument,
which reads the stored neutron ray
parameters by the component {\bf Virtual\_input}.

All neutron ray parameters are saved to the output file, which is by default
of ``text'' type, but can also assume the binary formats 
``float'' or ``double''. The storing of neutron rays continue until the 
specified number of simulations have been performed.

For performance reasons, it is possible to specify the size of 
the buffer to use in writing the neutron rays to file. Default 
is to buffer the whole file, but this may be fatal for very large
simulations.


\newpage
% Emacs settings: -*-mode: latex; TeX-master: "manual.tex"; -*-

\section{Res\_sample: A uniform scatterer for resolution calculation}
\label{s:res_sample}
\index{Samples!Resolution function, sample for}

\component{Res\_sample}{System, Alan Tennant}{$r_{\rm i}$, $r_{\rm o}$, $h$, $r_{\rm focus}$, $x_{\rm target}$, $y_{\rm target}$, $z_{\rm target}$, $E_0$, $\Delta E$ }{$x_w$, $y_h$, $z_t$, $x_{\rm focus}$, $y_{\rm focus}$, $a_{\rm v, focus}$, $a_{\rm h, focus}$, target index}{}

The component \textbf{Res\_sample} models an inelastic sample that
scatters completely homogeneous in position and energy,
within specified intervals. Regardless of
the state of the incoming neutron, all directions and energies for the
scattered neutron have the same probability. This is clearly
not a physical sample! Rather, the component is meant
for computation of the resolution function, but it may also be used
for test and debugging purposes. For calculations of the resolution
function, {\bf Res\_sample} should be used
together with the \textbf{Res\_monitor} component, described in
section~\ref{s:res_monitor}.

The shape of {\rm Res\_sample} is either a hollow cylinder
or a rectangular box, as specified for {\rm Vanadium}
(see section~\ref{s:v_sample}).The hollow cylinder shape is
specified with the inner and outer radius, $r_{\rm i}$ and $r_{\rm o}$,
respectively, and the height, $h$.
If these parameters are unspecified,
the shape is instead a box of dimensions $x_w$, $y_h$, and $z_t$.
See figure~\ref{f:res_sample}.\par
%
\begin{figure}[htbp]
  \begin{center}
        \psfrag{ri}[c][c]{\textit{radius\_i}}
        \psfrag{ro}[c][c]{\textit{radius\_o}}
        \psfrag{h}[c][c]{\textit{h}}
        \psfrag{bri}[c][c]{\textit{radius\_i}}
        \psfrag{bro}[c][c]{$-\textit{radius\_o}$}
        \psfrag{bh}[c][c]{\textit{h}}
        \psfrag{X}[c][c]{\textit{X}}
        \psfrag{Y}[c][c]{\textit{Y}}
        \psfrag{Z}[c][c]{\textit{Z}}
        \includegraphics[width=0.9\textwidth]{figures/res_sample.eps}
    \caption{The two possible shapes of the \textbf{Res\_sample} component.}
    \label{f:res_sample}
  \end{center}
\end{figure}
%
The component only propagates the neutrons that are scattered; neutrons
that would pass through or miss {\bf Res\_sample} are absorbed. There is no
modeling of the cross section of the sample, secondary extinction
\textit{etc.}; the scattering probability is proportional to the neutron
flight path length inside the sample, with the constant of
proportionality arbitrarily set to $1/(2|\textit{radius\_o}|)$. The
reason for this is that the resolution
function of an instrument (although including the sample size)
is independent of any sample properties such as scattering
and absorbtion cross sections.does not depend on the scattering
cross section.

The point of scattering in the sample is chosen uniformly
along the neutron flight path inside the sample, and the scattered
neutron is given a random energy and direction. The energy is selected in
the interval $[E_0-\Delta E; E_0+\Delta E]$ which hence must be
chosen large enough to cover all interesting neutrons. Similarly, the
direction is chosen in a user-specified range,
either within a sphere of radius $r_{\rm foc}$, within a rectangular
target with measures $(x_{\rm focus}, y_{\rm focus})$
or in the specified angular range. This target is positioned at the $x_{target}$, $y_{target}$, $z_{target}$ point in space, or using the target\_index for wich e.g. 1 is the further component, -1 is the previous, etc...

A special feature, used when computing resolution functions, is that the
component stores complete information about the scattering event in the
output parameter \textit{res\_struct}. The information includes initial
and final wave vectors, the coordinates of the scattering point, and the
neutron weight after the scattering event. From this information the
scattering parameters $({\bf Q}, \omega)$ can be recorded
for every scattering event and used to compute the resolution function.
For an example of using the
information in the output parameter, see the description of the
\textbf{Res\_monitor} component in section~\ref{s:res_monitor}.

\subsection{Background on resolution functions}

In an experiment, as well as in the simulation, the expected intensity
is by definition of the resolution function given by
%
$$
  I = \int R({\bf Q}, \omega) \sigma({\bf Q}, \omega) d{\bf Q}d\omega
$$
%
Here $I({\bf Q}_0, \omega_0)$ is the measured or simulated intensity in
the detector, $R$ is the resolution function for the instrument in a
given setup, $\sigma$ is the scattering cross section of the sample, and
$({\bf Q}, \omega)$ denote the scattering vector and energy transfer in
the sample. For the uniform scatterer, $\sigma({\bf Q}, \omega) = 1/V_0$
is a constant, so we have
%
$$
  I = 1/V_0 \int R({\bf Q}, \omega) d{\bf Q}d\omega
$$
%
If we instead consider only the intensity contributed by scattering with
parameters $({\bf Q}, \omega)$ that lie within a small part $\Delta\Omega$ of
the total phase space and has volume $\Delta V$,
%
$$
  I_{\Delta\Omega} = 1/V_0 \int_{\Delta\Omega} R({\bf Q}, \omega) d{\bf Q}d\omega
  = \frac{\Delta V}{V_0} R(\Delta\Omega)
$$
%
(where $R(\Delta\Omega)$ denotes the average value of $R$ over
$\Delta\Omega$), we get a good approximation of the value of $R$
provided that $\Delta\Omega$ is sufficiently small. This is useful with
the output from the simulations, since $I_{\Delta\Omega}$ is
approximated by

$$ I_{\Delta\Omega} \approx \sum_{({\bf Q_i},\omega_i) \in \Delta\Omega} p_i $$


This can be used to
histogram the resolution function or visualize it in different ways. The
3D visualization of the resolution function produced by the
\verb+mcresplot+ program for example uses this by displaying a cloud of
dots, the local density of which is proportional to the resolution
function.

The \verb+mcresplot+ program also computes the covariance and resolution
matrices. Letting $(x^1_i,x^2_i,x^3_i,x^4_i)$ denote the $({\bf
  Q_i},\omega_i)$ values obtained from the scattering events in the
simulation and $\mu^j = (\sum_i p_i x^j_i) / (\sum_i p_i)$ the mean
value of $x^j_i$, the covariance matrix is computed as
$$ {\bf C}_{jk} = \Big(\sum_i p_i (x^j_i - \mu_j) (x^k_i - \mu_k)\Big) /
   \Big(\sum_i p_i\Big) $$
This covariance matrix is given in the local coordinate system of the
sample component. The \verb+mcresplot+ program actually outputs the
covariance matrix in another coordinate system which is rotated around
the Y axis so that the projection to the X-Z plane of the average
scattering vector ${\bf Q}_{\rm avg} = (\sum_i p_i {\bf Q}_i) / (\sum_i
p_i)$ is parallel to the X axis.

The resolution matrix ${\bf M}$ is the inverse of the covariance matrix
and is also output in the rotated coordinate system by \verb+mcresplot+.
The 4-dimensional gaussian distribution, defined by
\begin{equation}
  \label{eq:gauss-res}
  f({\bf X}) = e^{-\frac{1}{2}{\bf X}^T {\bf M} {\bf X}}
\end{equation}
where ${\bf X} = ({\bf Q},\omega)$, has covariance matrix ${\bf C}$ and
thus defines the gaussian resolution function with the same covariance
as the resolution computed by the simulation.

The \verb+mcresplot+ program provides for the simultaneous visualization
of the computed and the gaussian resolution function by obtaining an
appropriate number of random points with the statistical
distribution~(\ref{eq:gauss-res}). Each point ${\bf X}$ is obtained as
follows: A vector ${\bf Y}$ is generated of four individually gaussian
distributed random numbers with mean zero and variance one. Using the
Cholesky decomposition of ${\bf C}$, ${\bf C} = {\bf L}{\bf L}^T$, we
have
$$ {\bf X} = {\bf L} {\bf Y}.$$




% Emacs settings: -*-mode: latex; TeX-master: "manual.tex"; -*-

\section{Res\_monitor: The monitor for resolution calculation}
\label{s:res_monitor}
\index{Monitors!Resolution monitor|see{Samples/Resolution function}}

\component{Res\_monitor}{(System); Alan Tennant, HMI}{$x_{\rm min}$, $x_{\rm max}$, $y_{\rm min}$, $y_{\rm max}$, filename, res\_sample, buffer size}{$x_w$, $y_h$, $z_t$, options}{}

The component {\bf Res\_monitor} is used for calculating the
resolution function of a particular instrument with detector of the
shape, size, and position as {\bf Res\_monitor}.
The shape of {\rm Res\_monitor} is by default rectangular,
but can be a box, a sphere, a disk, or a cylinder,
depending on the parameter ``options''.
The component works like a normal single detector, but
also records all scattering events and stores
them to a file that can later be read by 
the \MCS\ frontend tool \verb+mcresplot+.

For time-of-flight (TOF) instruments, {Res\_monitor} should be understood 
as giving the resolution of one time bin of the TOF-detector only; 
the bin properties being specified in the preceding {\bf TOF\_Res\_sample}.

As described in section~\ref{s:res_sample},
the {\bf Res\_monitor} should be used in connection with one of the
components {\bf Res\_sample} or {\bf TOF\_Res\_sample}, 
the name of which should be passed as an
input parameter to \textbf{Res\_monitor}. For example
\begin{verbatim}
    COMPONENT mysample = Res_sample( ... )
    ...
    COMPONENT det = Res_monitor(res_sample_comp = mysample, ...)
    ...
\end{verbatim}

The output file is in ASCII format, one line per scattering event, with
the following columns:
\begin{enumerate}
\item ${\bf k}_{\rm i}$, the three components of the initial wave vector.
\item ${\bf k}_{\rm f}$, the three components of the final wave vector.
\item ${\bf r}$, the three components of the position of the scattering
  event in the sample.
\item $p_{\rm i}$, the neutron weight just after the scattering event.
\item $p_{\rm f}$, the relative neutron weight adjustment from sample to
  detector (so the total weight in the detector is $p_{\rm i}p_{\rm f}$).
\end{enumerate}
From ${\bf k}_{\rm i}$ and ${\bf k}_{\rm f}$, we may compute ${\bf Q} =
{\bf k}_{\rm i} - {\bf k}_{\rm f}$ and $\omega 
= \hbar^2/(2 m_{\rm n})({\bf k}_{\rm i}^2 - {\bf k}_{\rm f}^2)$.
The vectors are given in the local coordinate system of the resolution
sample component. The wave vectors are in units of $\mbox{\AA}^{-1}$, the
energy transfer in meV.

The output parameters from {\bf Res\_monitor}
are the three count numbers, \textit{Nsum}, \textit{psum},
and \textit{p2sum}, and the handle \textit{file} of the output file.


\newpage
\section{Progress\_bar: Simulation progress and automatic saving}
\component{Progress\_bar}{System}{percent, flag\_save}{}{}
\label{s:progress-bar}
\index{Optics!Point in space (Arm, Optical bench)}
\index{Simulation progress bar}

This component is an Arm (see section \ref{explain:arm}) which additionally displays the simulation progress and status. The default is to work with a 10 \% step, but it may be changed using the \verb+percent+ parameter (from 0 to 100). The estimated computation time is displayed at the begining and actual simulation time is shown at the end.

Additionally, setting the \verb+flag_save+ to 1 results in a regular save of the data files during the simulation course. This means that is is possible to view the data before the end of the computation, and have also a trace of it in case of computer crash. The advancement level is stored in these temporary data files. Technically, this save is equivalent to sending regularly a USR2 signal to the running simulation.
