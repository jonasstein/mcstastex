\section{Source\_Maxwell\_3: A continuous source
with a Maxwellian spectrum}
\label{source-maxwell}
\index{Sources!Source\_Maxwell\_3}

\component{Source\_Maxwell\_3}{System}{ $h$, $w$, $d_{\rm foc}$, $xw$, $yh$, $\lambda_{\rm low}$, $\lambda_{\rm high}$, $I_1$, $T_1$}{$I_2$, $T_2$, $I_3$, $T_3$}{Validated}

This component is a source with a Maxwellian energy/wavelength distribution
sampled in the range $\lambda_{\rm low}$ to $\lambda_{\rm high}$.
The initial neutron ray position is chosen randomly from within a
rectangle of area $h \times w$ in the $z=0$ plane.
The initial neutron ray direction is focused within
a solid angle, defined by a rectangular target of width
$xw$, height $yh$, parallel to
the $xy$ plane placed at $(0,0,d_{\rm foc})$.
The energy distribution used is a sum of 1, 2, or 3 Maxwellians with
temperatures $T_1$ to $T_3$ and integrated intensities $I_1$ to $I_3$.

The initial weight of the created neutron ray, $\pi_1$, is
calculated in the following way for one single Maxwellian:
The intensity in a small wavelength interval $[\lambda, \lambda+d\lambda]$ is
$ I_1 M(\lambda,T_1) d\lambda $
where
$M(\lambda,T_1) = 2 \alpha^2 \exp(-\alpha/\lambda^2) / \lambda^5 $ 
is the normalized Maxwell distribution ($\alpha=949.0$~K \AA$^2/T_1$).
The number of neutrons per second through a focusing window
of solid angle $\Omega$
from a source of area $A$ within the wavelength interval $\lambda_1$ to
$\lambda_2$ is thus
\begin{equation}
I_{\rm tot} = \Omega A \int_{\lambda_1}^{\lambda_2} I_1 M(\lambda,T_1) d\lambda.
\end{equation}
In a Monte Carlo integration, the observed intensity becomes
\begin{equation}
I_{\rm MC} \approx N_{\rm MC} \int p(\lambda) \pi_1(\lambda) d\lambda ,
\end{equation}
where $N_{\rm MC}$ is the number of Monte Carlo steps.
We here choose the wavelength from a uniform distribution between the two
limits, giving $p(\lambda)=1/(\lambda_2-\lambda_1)$.
To fulfill $I_{\rm tot} = I_{\rm MC}$ we need to have
\begin{equation}
\pi_1(\lambda) = \Omega A (\lambda_2-\lambda_1) I_1 M(\lambda,T_1) / N_{\rm MC} .
\end{equation}

%This expression is strictly valid only for $\Omega \ll 1$,
%see also the discussion on focusing in section \ref{s:focus}.
The expression is easily generalized to a general number of Maxwellians.

The component {\bf Source\_gen} (see section \ref{source-gen}) 
works on the same principle, but provides more options concerning 
wavelength/energy range specifications, shape, etc.

