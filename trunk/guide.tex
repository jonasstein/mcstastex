% Emacs settings: -*-mode: latex; TeX-master: "manual.tex"; -*-

\chapter{Advanced optical components: mirrors, guides, and choppers}
\index{Optics|textbf}

This section describes advanced neutron optical
components such as supermirrors and guides as well as various rotating choppers.
A description of the reflectivity of a supermirror is found
in section~\ref{s:mirror}.

\section{Mirror: The single mirror}
\label{s:mirror}
\index{Optics!Mirror plane}
\component{Mirror}{System}{$l$, $h$, $m$}{$R_0, Q_c, W, \alpha$}{validated, no gravitation support}

The component {\bf Mirror}
models a single rectangular neutron mirror plate. It can
be used to \textit{e.g.}~assemble a complete neutron guide by putting multiple
mirror components at appropriate locations and orientations in the
instrument definition, much like a real guide is build from individual
mirrors.

The mirror is assumed to lie in the first quadrant of the
$x$-$y$ plane, with one corner at $(0,0,0)$.
If the neutron trajectory intersects the mirror plate, it is
reflected, otherwise it is left untouched. Since the mirror lies in the
$x$-$y$ plane, an incoming neutron with velocity
${\bf v}_{\rm i} = (v_x,v_y,v_z)$
is reflected with velocity ${\bf v}_{\rm f} = (v_x,v_y,-v_z)$.
The computation of the reflectivity is handled as detailed in
section~\ref{ss:mirrorreflect}.

The input parameters of this component are
the rectangular mirror dimensions $(l, h)$
and the values of $R_0, m, Q_c, W$, and $\alpha$ for the mirror.
As a special case, if $m=0$ then the reflectivity is zero for all $Q$,
\textit{i.e.}\ the surface is completely absorbing.

This component may produce wrong results with gravitation support.

\subsection{Mirror reflectivity}
\label{ss:mirrorreflect}
To compute the reflectivity of the supermirrors, we use an empirical
formula derived from experimental data \cite{pb_241_50},
see Fig.~\ref{f:reflectivity}. The reflectivity is given by
\begin{equation} \label{e:Rmirror}
  R = \left\{
    \begin{array}{ll}
      R_0 & \textrm{if $Q \leq Q_{\rm c}$} \\
      \frac{1}{2}R_0(1 - \tanh[(Q - m Q_{\rm c})/W])(1-\alpha(Q-Q_{\rm c}))
         & \textrm{if $Q > Q_{\rm c}$}
    \end{array}
  \right.
\end{equation}

Here $Q$ is the length of the scattering vector (in \AA$^{-1}$)
defined by
\begin{equation} \label{e:reflectivity}
Q = |{\bf k}_{\bf i} - {\bf k}_{\bf f}|
  = \frac{m_{\rm n}}{\hbar} |{\bf v}_{\bf i} - {\bf v}_{\bf f}|,
\end{equation}
$m_{\rm n}$ being the neutron mass.
The number $m$ in (\ref{e:Rmirror}) is a parameter determined by
the mirror materials,
the bilayer sequence, and the number of bilayers.
As can be seen, $R=R_0$ for $Q < Q_{\rm c}$, where $Q_{\rm c}$ is the
critical scattering wave vector for a single layer of the mirror
material. At higher values of $Q$, the reflectivity starts falling
linearly with a slope $\alpha$ until a "soft cut-off" at $Q = m Q_{\rm c}$.
The width of this cut-off is denoted $W$. See the example reflection curve in
figure~\ref{f:reflectivity}.

\subsection{Algorithm}
In the components, the neutron weight is adjusted with the amount $\pi_i = R$.
To avoid spending large amounts of computation time on very low-weight
neutrons, neutrons for which the reflectivity is lower than about
$10^{-10}$ are ABSORB'ed.

\begin{figure}
  \begin{center}
    \includegraphics[width=0.6\textwidth]{figures/supermirror.eps}
  \end{center}
\caption{A typical reflectivity curve for a supermirror,
Eq.~(\protect\ref{e:reflectivity}). The used values are
$ m=4$, $R_0=1$, $Q_{\rm c} = 0.02$~\AA$^{-1}$, $\alpha = 6.49$~\AA,
$ W=1/300$~\AA$^{-1}$.
}
\label{f:reflectivity}
\end{figure}

\newpage

\section{Guide: The guide section}
\index{Optics!Straight guide}

\component{Guide}{System}{$w_1, h_1$, $w_2, h_2$, $l$, $m$}{$R_0, Q_c, W, \alpha$}{validated, no gravitation support}

The component {\bf Guide}
models a guide tube consisting of four flat mirrors. The
guide is centered on the $z$ axis with rectangular entrance and exit
openings parallel to the $x$-$y$ plane. The entrance has the dimensions
$(w_1,h_1)$ and placed at $z=0$. The exit is of dimensions $(w_2,h_2)$
and is placed at $z=l$ where $l$ is the guide length. See
figure~\ref{f:guide}. Neutrons not clearing the guide entrance are
ABSORB'ed. For a more general guide simulation, see {\bf Channeled\_guide}
in section~\ref{s:channeled_guide}.

Further input parameters are the values of $R_0, m, Q_c, W$, and $\alpha$
as for {\bf Mirror}.

{\bf Guide} may produce wrong results with gravitation support.
Use {\bf Guide\_gravity} (section \ref{s:guide_gravity}) in this case.

\begin{figure}
  \begin{center}
    \includegraphics[width=0.7\textwidth]{figures/guide1.eps}
  \end{center}
\caption{The geometry used for the guide component.}
\label{f:guide}
\end{figure}

\subsection{Guide geometry and reflection}
For computations on the guide geometry, we define the planes of the four
guide sides by giving their normal vectors (pointing into the guide)
and a point lying in the plane:
$$
\begin{array}{rclcrcl}
{\bf n}^v_1 &=& (l, 0, {(w_2 - w_1) / 2})
     & & {\bf O}^v_1 &=& (- w_1 / 2, 0, 0) \\
{\bf n}^v_2 &=& (-l, 0, {(w_2 - w_1) / 2})
     & & {\bf O}^v_2 &=& (w_1 / 2, 0, 0) \\
{\bf n}^h_1 &=& (0, l, {(h_2 - h_1) / 2})
     & & {\bf O}^h_1 &=& (0, - h_1 / 2, 0) \\
{\bf n}^h_2 &=& (0, -l, {(h_2 - h_1) / 2})
     & & {\bf O}^h_2 &=& (0, h_1 / 2, 0) \\
\end{array}
$$
In the following, we refer to an arbitrary guide side by its origin
{\bf O} and normal {\bf n}.

With these definitions, the time of intersection of the neutron with a
guide side can be computed by considering the projection onto the
normal:
\begin{equation}
t^\alpha_\beta = \frac{({\bf O}^\alpha_\beta - {\bf r}_0) \cdot {\bf n}^\alpha_\beta}
  {{\bf v} \cdot {\bf n}^\alpha_\beta}  ,
\end{equation}
where $\alpha$ and $\beta$ are indices for the different guide walls,
assuming the values (h,v) and (1,2), respectively.
For a neutron that leaves the guide directly through the guide exit we have
\begin{equation}
t_{\rm exit} = \frac{l - z_0}{v_z}
\end{equation}

\subsection{Algorithm}
To compute the interaction of the neutron
with the guide, the neutron is initially propagated to the $z = 0$ plane of the
guide entrance. If it misses the entrance, it is ABSORB'ed. Otherwise,
we repeatedly compute the time of intersection with the
four mirror sides and the guide exit. The smallest positive $t$ thus
found gives the time of the next intersection with the guide (or in the
case of the guide exit, the time when the neutron leaves the guide). The
neutron is propagated to this point, the reflection from the side is
computed and the process is repeated until the neutron leaves the guide.

The reflected velocity ${\bf v}_{\rm f}$ of the neutron with incoming velocity
${\bf v}_{\rm i}$ is computed by the formula
\begin{equation}
 {\bf v}_{\rm f} =
  {\bf v}_{\rm i}
   - 2{{\bf n} \cdot {\bf v}_{\rm i} \over {|{\bf n}|^2}} {\bf n}
\end{equation}
This expression is arrived at by again considering the projection onto
the mirror normal (see figure~\ref{f:guidereflect}). The reflectivity of the
mirror is taken into account as explained in section~\ref{s:mirrorreflect}.

\begin{figure}
  \begin{center}
    \includegraphics[width=0.5\textwidth]{figures/guide2.eps}
  \end{center}
\caption{Neutron reflecting from mirror. ${\bf v}_{\rm i}$ and
${\bf v}_{\rm f}$ are the initial and final velocities, respectively,
and {\bf n} is a vector normal to the mirror surface.}
\label{f:guidereflect}
\end{figure}

There are a few optimizations possible here to avoid redundant
computations. Since the neutron is always inside the guide during the
computations, we always have
$({\bf O} - {\bf r}_0) \cdot {\bf n} \leq 0$.
Thus $t \leq 0$ if ${\bf v} \cdot {\bf n} \geq 0$, so in this case
there is no need to actually compute $t$. Some redundant computations
are also avoided by utilizing symmetry and the fact that many
components of {\bf n} and {\bf O} are zero.

\newpage

\section{Guide\_channeled: A guide section component with multiple channels}
\label{s:channeled_guide}
\index{Optics!Guide with channels (non focusing)}

\component{Guide\_channeled}{System}{$w_1, h_1$, $w_2, h_2$, $l$, $k$, $m_x, m_y$}{$d, R_0, Q_{cx}, Q_{cy}, W, \alpha_x, \alpha_y$}{partly validated, no gravitation support}

The component {\bf Guide\_channeled} is a more complex variation of {\bf Guide}
described in the previous section. It allows the specification
of different supermirror parameters for the horizontal and vertical
mirrors, and also implements guides with multiple channels as used in
neutron bender devices. By setting the $m$ value of the supermirror
coatings to zero, nonreflecting walls are
simulated; this may be used to simulate a Soller collimator.

The input parameters are $w_1$, $h_1$, $w_2$, $h_2$, and $l$
to set the guide dimensions as for {\bf Guide}
(entry window, exit window, and length);
$k$ to set the number of channels; $d$ to set the thickness of the
channel walls; and $R_0$, $W$, $Q_{cx}$, $Q_{cy}$, $\alpha_x$, $\alpha_y$,
$m_x$, and $m_y$ to set the supermirror parameters as described under {\bf Guide}
(the names with \textit{x} denote the vertical mirrors,
and those with \textit{y} denote the horizontal ones).

\subsection{Algorithm}
The implementation is based on that of {\bf Guide}. Initially, the
channel which the neutron will enter is computed. The $x$ coordinate is
then shifted so that the channel can be simulated as a single instance
of the Guide component. Finally the coordinates are restored when the
neutron exits the guide or is absorbed.

\subsection{Known problems}
This component may produce wrong results with gravitation support. Use Guide\_gravity (section \ref{s:guide_gravity}) in this case.
Additionally, the focusing channeled geometry (for $k > 1$ and different
values of $w_1$ and $w_2$) is buggy
(wall slopes are not computed correctly, and the component 'leaks' neutrons).

\newpage

\section{Guide\_gravity: A guide with multiple channels and gravitation handling}
\label{s:guide_gravity}
\index{Optics!Guide with channels and gravitation handling}

\component{Guide\_gravity}{System}{$w_1, h_1$, $w_2, h_2$, $l$, $k$, $m$}{$d, R_0, Q_c, W, \alpha$, wavy, chamfers, $k_h$, $n$, $G$}{fully validated, {\bf with} gravitation support}

This component is a variation of {\bf Guide\_channeled}
(section \ref{s:channeled_guide}) with the ability to handle
gravitation effects and functional channeled focusing geometry.
Channels can be specified in two dimensions,
producing a 2D array ($k, k_h$) on smaller guide channels.

Waviness effects, supposed to be randomly distributed
(\emph{i.e.} non-periodic waviness)
can be specified globally, or for each part of the guide section.
Additionally, chamfers (originating from the substrate manufacturing)
may be defined the same way.

The straight section of length $l$ may be divided into $n$ bits of same length
within which chamfers are taken into account.

To activate gravitation support, either select the \MCS\ gravitation support,
or set the gravitation field strength $G$ (e.g. -9.81 on Earth).

