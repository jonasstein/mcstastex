% Emacs settings: -*-mode: latex; TeX-master: "manual.tex"; -*-

\chapter{The component library}
\label{s:components}
\index{Library!Components|textbf}

This chapter has been removed from the manual and will instead be published 
in a separate manual describing the McStas components. The McStas component manual will be edited by the McStas authors and it will include contributions from users writing components. Until the \MCS\ component manual is published we refer to the McStas web-page~\cite{mcstas_webpage} where all components are documented using the McDoc system or to previous manual for 
release 1.4.

\section{A short overview of the \MCS\ component library}
\label{s:comp-overview}

This section gives a quick overview of available \MCS\ components
provided with the distribution, in the \verb+MCSTAS+ library. The
location of this library is detailed in section~\ref{s:files}. All of them are thought to be reliable, eventhough no absolute guaranty may be given concerning their accuracy.\index{Environment variable!MCSTAS}

The \verb+contrib+ directory of the library contains components that were given by \MCS\ users, but are not validated yet. \index{Library!Components!contrib}

Additionally the \verb+obsolete+ directory of the library gathers components that were renamed, or considered to be outdated. Anyway, they still all work as before.
\index{Library!Components!obsolete}

The \verb+mcdoc+ front-end (section~\ref{s:mcdoc-run}) enables to display both the 
catalog of the \MCS\ library, e.g using: \index{Tools!mcdoc}
\begin{quote}
  \verb|mcdoc --show|
\end{quote}
as well as the documentation of specific components, e.g with:
\begin{quote}
  \verb|mcdoc --text| {\it name} \\
  \verb|mcdoc --show| {\it file.comp}
\end{quote}
The first line will search for all components matching the {\it name}, and display their help section as text, where as the second example will display the help corresponding to the {\it file.comp} component, using your BROWSER\index{Environment variable!BROWSER} setting, or as text if unset. The \verb+--help+ option will display the command help, as usual.

\begin{table}
  \begin{center}
    {\let\my=\\
    \begin{tabular}{|p{0.24\textwidth}|p{0.7\textwidth}|}
      \hline
       MCSTAS/sources & Description \\
       \hline
Adapt\_check & Optimization specifier for the Source\_adapt component. \\
ESS\_moderator\_long & Parametrised pulsed source for modelling ESS long pulses. \\
ESS\_moderator\_short & A parametrised pulsed source for modelling ESS short pulses. \\
Moderator  & A simple pulsed source for time-of-flight. \\
Monitor\_Optimizer &  To be used after the Source\_Optimizer component. \\
Source\_Maxwell\_3 & Source with up to three Maxwellian distributions \\
Source\_Optimizer & A component that optimizes the neutron flux passing through the Source\_Optimizer in order to have the maximum flux at the Monitor\_Optimizer position. \\
Source\_adapt  &       Neutron source with adaptive importance sampling. \\
Source\_div &          Neutron source with Gaussian divergence. \\
Source\_flat &  A circular neutron source with flat energy spectrum and arbitrary flux.\\
Source\_flat\_lambda & Neutron source with flat wavelength spectrum and
arbitrary flux. \\
Source\_flux         & An old variant of the official
                      Source\_flux\_lambda component. \\
Source\_flux\_lambda  & Neutron source with flat wavelength
                      spectrum and 
                      user-specified flux. \\
Source\_gen     &    Circular/squared neutron source with flat or Maxwellian
                      energy/wavelength spectrum (possibly spatially
                      gaussian). \\
Virtual\_input &  Source-like component that generates neutron events from an ascii/binary 'virtual source' file (for Virtual\_output). \\
Virtual\_output &  Detector-like component that writes neutron state (for Virtual\_input). \\
      \hline
    \end{tabular}
    \caption{Source components of the \MCS\ library.}
    \label{t:comp-sources}
    \index{Library!Components!sources}
    }
  \end{center}
\end{table}


\begin{table}
  \begin{center}
    {\let\my=\\
    \begin{tabular}{|p{0.24\textwidth}|p{0.7\textwidth}|}
      \hline
       MCSTAS/optics & Description \\
       \hline
Arm                &  Arm/optical bench \\
 Beamstop          &   Rectangular/circular
                      beam stop. \\
 Bender            &   Models a curved
                      neutron guide. \\
 Chopper           &   Disk chopper. \\
 Chopper\_Fermi     &   Fermi Chopper with
                      curved slits. \\
 Collimator\_straight &  A simple analytical Soller collimator
                      (with triangular
                      transmission).  \\
 Filter\_gen        &   This components may
                      either set the flux
                      or change it (filter-like), using
                      an external data
                      file. \\
 Guide             &   Neutron guide. \\
 Guide\_channeled   &   Neutron guide with
                      channels (bender
                      section). \\
 Guide\_gravity     &  Neutron guide with gravity. Can be
                      channeled and focusing. \\

 Guide\_wavy        &   Neutron guide with
                      gaussian waviness. \\

 Mirror             &  Single mirror plate. \\

                      
 Monochromator\_curved & Double bent multiple crystal
                      slabs with anisotropic gaussian
                      mosaic. \\

 Monochromator\_flat &  Flat Monochromator
                      crystal with
                      anisotropic mosaic. \\

 Selector            & A velocity selector
                      (helical lamella
                      type) such as
                      V\_selector component. \\

 Slit                & Rectangular/circular
                      slit. \\

 V\_selector          & Velocity selector. \\
      \hline
    \end{tabular}
    \caption{Optics components of the \MCS\ library.}
    \label{t:comp-optics}
    \index{Library!Components!optics}
    }
  \end{center}
\end{table}

\begin{table}
  \begin{center}
    {\let\my=\\
    \begin{tabular}{|p{0.24\textwidth}|p{0.7\textwidth}|}
      \hline
       MCSTAS/samples & Description \\ 
       \hline
       Powder1      &  General powder sample with a single
                scattering vector. \\

 Res\_sample   & Sample for resolution function
                calculation. \\

 Single\_crystal & Mosaic single crystal with multiple
                scattering vectors. \\

 V\_sample      & Vanadium sample.\\
      \hline
    \end{tabular}
    \caption{Sample components of the \MCS\ library.}
    \label{t:comp-samples}
    \index{Library!Components!samples}
    }
  \end{center}
\end{table}

\begin{table}
  \begin{center}
    {\let\my=\\
    \begin{tabular}{|p{0.24\textwidth}|p{0.7\textwidth}|}
      \hline
       MCSTAS/monitors & Description \\ 
       \hline
DivLambda\_monitor &  Divergence/wavelength monitor. \\
DivPos\_monitor  &    Divergence/position
                    monitor (acceptance
                    diagram). \\
Divergence\_monitor &  Horizontal+vertical
                    divergence monitor. \\
EPSD\_monitor    &    A monitor measuring neutron
                    intensity vs. position, x,
                    and neutron energy, E. \\
E\_monitor       &    Energy-sensitive monitor. \\
Hdiv\_monitor    &    Horizontal divergence monitor
L\_monitor           Wavelength-sensitive monitor. \\
Monitor          &   Simple
                    single detector/monitor. \\
                    
Monitor\_4PI     &    Monitor that detects ALL
                    non-absorbed neutrons. \\
Monitor\_nD      &    This
                    component is a general
          Monitor that can output
                    0/1/2D signals (Intensity
                    or signal vs. [something]
                    and vs. [something] ...). \\
PSD\_monitor     &    Position-sensitive
                    monitor. \\
PSD\_monitor\_4PI  &   Spherical
                    position-sensitive
                    detector. \\
PSDcyl\_monitor  &    A 2D
                    Position-sensitive
                    monitor. The shape is
                    cylindrical with the axis
                    vertical. The monitor
                    covers the whole cylinder
                    (360 degrees). \\
PSDlin\_monitor   &   Rectangular 1D PSD,
                    measuring intensity vs.
                    vertical position, x. \\
PreMonitor\_nD    &   This component is a
                    PreMonitor that is to be
                    used with one Monitor\_nD,
                    in order to record some
                    neutron parameter correlations. \\
Res\_monitor      &   Monitor for resolution
                    calculations. \\
TOFLambda\_monitor &  Time-of-flight/wavelength
                    monitor. \\
TOF\_cylPSD\_monitor & Cylindrical (2pi) PSD
                    Time-of-flight monitor. \\
TOF\_monitor     &    Rectangular Time-of-flight
                    monitor. \\
TOFlog\_mon      &    Rectangular Time-of-flight
                    monitor with logarithmic
                    time binning. \\
      \hline
    \end{tabular}
    \caption{Monitor components of the \MCS\ library.}
    \label{t:comp-monitors}
    \index{Library!Components!monitors}
    }
  \end{center}
\end{table}

\begin{table}
  \begin{center}
    {\let\my=\\
    \begin{tabular}{|p{0.24\textwidth}|p{0.7\textwidth}|}
      \hline
       MCSTAS/misc & Description \\ 
       \hline
       Progress\_bar  &       A simulation
                      progress bar. May also trigger intermediate SAVE.\\ 
 Vitess\_input      &   Read neutron state
                      parameters from
                      VITESS neutron file.\\ 
 Vitess\_output    &  Write neutron state
     parameters to VITESS
                      neutron file.\\
      \hline
    \end{tabular}
    \caption{Miscellaneous components of the \MCS\ library.}
    \label{t:comp-misc}
    \index{Library!Components!misc}
    }
  \end{center}
\end{table}

\begin{table}
  \begin{center}
    {\let\my=\\
    \begin{tabular}{|p{0.24\textwidth}|p{0.7\textwidth}|}
      \hline
       MCSTAS/contrib & Description \\ 
       \hline
       Al\_window     &         Aluminium
                        window in the beam. \\
 Collimator\_ROC   &      Radial
                        Oscillationg
                        Collimator (ROC). \\
 FermiChopper    &       Fermi Chopper with
                        rotating frame. \\
 Filter\_graphite  &      Pyrolytic
                        graphite filter
                        (analytical model). \\
 Filter\_powder   &       Box-shaped powder
                        filter based on Single\_crystal (unstable). \\
 Guide\_honeycomb & Neutron guide
                        with gravity and
                        honeycomb geometry. Can be
                        channeled and
                        focusing. \\
 He3\_cell    &           Polarised 3He cell. \\

 Monochromator\_2foc   &  Double bent
                        monochromator with
                        multiple slabs. \\

 SiC           &         SiC multilayer sample for reflectivity simulations. \\
      \hline
    \end{tabular}
    \caption{Contributed components of the \MCS\ library.}
    \index{Library!Components!contrib}
    \label{t:comp-contrib}
    }
  \end{center}
\end{table}

\begin{table}
  \begin{center}
    {\let\my=\\
    \begin{tabular}{|p{0.24\textwidth}|p{0.7\textwidth}|}
      \hline
       MCSTAS/share & Description \\ 
       \hline
       adapt\_tree-lib  & Handles a simulation optimisation space for
       adatative importance sampling.
                          Used by the Source\_adapt component. \\
       {\bf mcstas-r}      &   Main Run-time library (always included). \\
       monitor\_nd-lib & Handles multiple monitor types. 
                        Used by Monitor\_nD, Res\_monitor, \ldots \\
       read\_table-lib  & Enables to read a data table (text/binary) to be used within
                          an instrument or a component. \\
       vitess-lib &     Enables to read/write Vitess event binary files. 
                        Used by Vitess\_input and Vitess\_output \\             
      \hline
    \end{tabular}
    \caption{Shared libraries of the \MCS\ library. See Appendix~\ref{c:kernelcalls} for details.}
    \label{t:comp-share}
    \index{Library!Components!share}
    \index{Library!Run-time}
    }
  \end{center}
\end{table}

\begin{table}
  \begin{center}
    {\let\my=\\
    \begin{tabular}{|p{0.24\textwidth}|p{0.7\textwidth}|}
      \hline
       MCSTAS/data & Description \\ 
       \hline
 .lau & Laue pattern file, as issued from Crystallographica or FullProf
       data: [ h   k   l Mult. d-space 2Theta   F-squared ] \\
 .trm & transmission file, typically for monochromator crystals and filters
       data: [ k (Angs-1) , Transmission (0-1) ] \\
 .rfl & reflectivity file, typically for mirrors and monochromator crystals
       data: [ k (Angs-1) , Reflectivity (0-1) ] \\       
      \hline
    \end{tabular}
    \caption{Data files of the \MCS\ library.}
    \label{t:comp-data}
    \index{Library!Components!data}
    }
  \end{center}
\end{table}
