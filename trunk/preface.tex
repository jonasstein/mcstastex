% Emacs settings: -*-mode: latex; TeX-master: "manual.tex"; -*-

\addcontentsline{toc}{chapter}{\protect\numberline{}{Preface and acknowledgements}}
\chapter*{Preface and acknowledgements}
This document contains information on the Monte Carlo neutron
ray-tracing program \MCS\ version \version, building on the initial
release in October 1998 of version 1.0 as presented in Ref.~\cite{nn_10_20}. The reader of this
document is supposed to have some knowledge of neutron scattering,
whereas only little knowledge about simulation techniques is
required. In a few places, we also assume familiarity with the
use of the C programming language and UNIX/Linux.

It is a pleasure to thank Prof.~Kurt N.~Clausen, PSI, for his continuous
support to this project and for having initiated \MCS\ in the first
place. Essential support has also been given by Prof.~Robert McGreevy, ISIS.
Apart from the authors of this manual, also Prof. Per-Olof \AA strand, NTNU Trondheim,
has contributed to the development of the \MCS\ system.
%Both he and our other collaborators, Henrik M.\ R\o nnow and Mark
%Hagen have made major contributions to the project.  Also the
%contributions from our test users, the students Asger Abrahamsen, Niels
%Bech Christensen, and Erik Lauridsen, are gratefully acknowledged; they
%gave us an excellent opportunity to pinpoint a vast amount of serious
%errors in the test version.  Useful comments to this document itself
%have been given by Bella Lake and Alan Tennant.
We have also benefited
from discussions with many other people in the neutron scattering
community, too numerous to mention here.

%Philipp Bernhardt contributed the two chopper components in
%sections~\ref{s:chopper} and~\ref{s:first_chopper}. Emmanuel Farhi
%contributed the components in sections~\ref{s:sourceoptimizer},
%\ref{s:monitornd}, and~\ref{s:monitoroptimizer}. We encourage other
%users to contribute components with manual entries for inclusion in
%future versions of \MCS.

In case of errors, questions, or suggestions,
%or other need for support should arise,
do not hesitate to
contact the authors at \verb+mcstas@risoe.dk+
or consult the \MCS\ home page~\cite{mcstas_webpage}.
A special bug/request reporting service is available \cite{mczilla_webpage}.

If you {\bf appreciate} this software, please subscribe to the \verb+neutron-mc@risoe.dk+ email list, send us a smiley message, and contribute to the package. We also encourage you to refer to this software when publishing results, with the following citations:
\begin{itemize}
\item{K. Lefmann and K. Nielsen, Neutron News {\bf 10/3}, 20, (1999).}
\item{P. Willendrup, E. Farhi and K. Lefmann, Physica B, {\bf 350} (2004) 735.}
\end{itemize}


\section*{\MCS\ \version\ contributors}
Several people outside the core developer team have been contributing
to \MCS\ \version:
\begin{itemize}
\item Thorwald van Vuure, ILL contributed his \verb+PSD_Detector.comp+
  component, a physical detector model.
\item Chama Hennae, ENSIMAG worked with Emmanuel Farhi on the new
  \verb+Virtual_mcnp_input.comp+ and \verb+Virtual_mcnp_output.comp+
  for handeling MCNP event files.
\item Ludovic Giller and Uwe Filges, PSI contributed
  \verb+Source_multi_surfaces.comp+, a source component with multiple
  surface areas with individual spectrums.
\end{itemize}
Thank you guys! This is what \MCS\ is all about!

Third party software included in \MCS\ are:
\begin{itemize}
\item perl Math::Amoeba from John A.R. Williams \verb+J.A.R.Williams@aston.ac.uk+.
\item perl Tk::Codetext from Hans Jeuken \verb+haje@toneel.demon.nl+.
\item scilab Plotlib from St\'ephane Mottelet \verb+mottelet@utc.fr+.
\item and optionally PGPLOT from Tim Pearson \verb+tjp@astro.caltech.edu+.
\end{itemize}

The \MCS\ project has been supported by the European Union, initially
through the XENNI program and the RTD ``Cool Neutrons'' program in FP4,
In FP5, \MCS\ was supported strongly through the
``SCANS'' program.
Currently, in FP6, \MCS\ is supported through the Joint Research Activity
``MCNSI'' under the Integrated Infrastructure Initiative ``NMI3'', see
the home pages~\cite{mcnsi_webpage,nmi3_webpage}. \MCS\ is also
supported directly from the construction project for the ISIS second
target station (TS2), see~\cite{ts2_webpage}.
