% Emacs settings: -*-mode: latex; TeX-master: "manual.tex"; -*-

\addcontentsline{toc}{chapter}{\protect\numberline{}{Preface and acknowledgements}}
\chapter*{Preface and acknowledgements}
This document contains information on the Monte Carlo neutron
ray-tracing program \MCS\ version \version, an update to the initial
release in October 1998 of version 1.0 as presented in Ref.~\cite{nn_10_20}. The reader of this
document is supposed to have some knowledge of neutron scattering,
whereas only little knowledge about simulation techniques is
required. In a few places, we also assume familiarity with the
use of C, UNIX and of the world wide web (WWW).

It is a pleasure to thank Prof.~Kurt N.~Clausen for his continuous
support to this project and for having initiated the work in the first
place. 
%Both he and our other collaborators, Henrik M.\ R\o nnow and Mark
%Hagen have made major contributions to the project.  Also the
%contributions from our test users, the students Asger Abrahamsen, Niels
%Bech Christensen, and Erik Lauridsen, are gratefully acknowledged; they
%gave us an excellent opportunity to pinpoint a vast amount of serious
%errors in the test version.  Useful comments to this document itself
%have been given by Bella Lake and Alan Tennant.  
We have also benefited
from discussions with many other people in the neutron scattering
community, too numerous to mention here.

%Philipp Bernhardt contributed the two chopper components in
%sections~\ref{s:chopper} and~\ref{s:first_chopper}. Emmanuel Farhi
%contributed the components in sections~\ref{s:sourceoptimizer},
%\ref{s:monitornd}, and~\ref{s:monitoroptimizer}. We encourage other
%users to contribute components with manual entries for inclusion in
%future versions of \MCS.

This project has been supported by the European Union
through the XENNI program and the RTD ``Cool Neutrons'' and ``SCANS'' programs.

In case of any errors, questions, suggestions, %or other need for support should arise,
%do not hesitate to 
contact the authors at \verb+mcstas@risoe.dk+
or consult the \MCS\ WWW home page~\cite{mcstas_webpage}.
