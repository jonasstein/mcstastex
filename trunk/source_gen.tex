\section{Source\_gen: A general continuous source}
\label{source-gen}
\index{Sources!Source\_gen}

\component{Source\_gen}{System, E. Farhi}{$w$, $h$, $xw$, $yh$, $E_0$, $\Delta E$, $T_1$, $T_2$, $T_3$, $I_1$, $I_2$, $I_3$ }{$r$, $\lambda_0$, $d\lambda$, $E_{min}$, $E_{max}$, $\lambda_{min}$, $\lambda_{max}$}{Validated for Maxwellian expressions, position is the center of area}

This component is a continuous neutron source (rectangular or circular), which aims at
a square target centered at the beam (in order to improve MC-acceptance
rate). The angular divergence is then given by the dimensions of the
target. Size may be rectangular (dimension $h$ and $w$, or a disk of radius $r$. The wavelength/energy range to emit is specified either using center and half width, or using minimum and maximum boundaries, alternatively for energy and wavelength.
The flux spectrum is specified with the same Maxwellian parameters as in component Source\_Maxwell\_3 (refer to section \ref{source-maxwell}).

Maxwellian parameters for some sources are in the table given below. For some cases, a correction factor (multiply $I$ parameters) should be used to reach measured data.

\begin{table}
  \begin{center}
  {\let\my=\\
    \begin{tabular}{cccccccc}
    \hline
    Source Name & $T_1$ & $I_1$ & $T_2$ & $I_2$ & $T_3$ & $I_3$ & factor \\
    \hline
    PSI cold source & 150.42 & 3.67e11   & 38.74 & 3.64e11    & 14.84& 0.95e11    &\\
    ILL VCS (H1)    & 216.8  & 1.24e13  & 33.9  & 1.02e13   & 16.7 & 3.0423e12 &\\
    ILL HCS (H5)    & 413.5  & 10.22e12  & 145.8 & 3.44e13    & 40.1 & 2.78e13    & *2\\
    ILL Thermal(H2) & 683.7  & 0.5874e13& 257.7 & 2.5099e13 & 16.7 & 1.0343e12 & /2.25\\
    ILL Hot source  & 1695   & 1.74e13   & 708   & 3.9e12     &      &            &\\
    \end{tabular}
    \caption{Flux parameters for Source\_gen and Source\_Maxwell\_3}
    \label{t:source-params}
  }
  \end{center}
\end{table}