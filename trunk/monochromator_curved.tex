% Emacs settings: -*-mode: latex; TeX-master: "manual.tex"; -*-

\section{Monochromator\_curved: An infinitely thin, curved mosaic crystal with 
a single scattering vector}
\label{s:monochromator_curved}

\component{Monochromator\_curved}{System, Peter Link}{$z_{\rm w}$, $y_{\rm h}$, gap, $\eta_{\rm h}$, $\eta_{\rm v}$, $n_{\rm h}$, $n_{\rm v}$, $R_0$, $Q$, $r_{\rm h}$, $r_{\rm v}$}{$d_{\rm m}$, $\eta$, $h$, $w$, verbose, transmit, reflect}


This component simulates an array of infinitely thin single
crystals with a single scattering vector perpendicular to the
surface and a mosaic spread. 
This component is used to simulate a singly or doubly
curved monochromator or analyzer.

The component uses  rectangular pieces of monochromator material
as described in {\bf Monochromator\_curved}. 
The important parameters are the piece height and width,
$y_{\rm h}$ and $z_{\rm w}$, respectively, the
horizontal and vertical mosaicities, $\eta_{\rm h}$ and $\eta_{\rm v}$,
respectively. The number of pieces vertically and horizontally are called
$n_{\rm v}$ and $n_{\rm h}$, respectively, and the vertical and horizontal
radii of curvature are named $r_{\rm v}$ and $r_{\rm h}$, respectively.
The scattering vector is named $Q$, and as described in 
{\bf Monochromator\_flat}, multiples of $Q$ can be applied.

If just one mosaicity, $\eta$, is specified, this the same for 
both directions.

The constant monochromator reflectivity, $R_0$ can be replaced by
a file of tabulated reflectivities. In the same sense, the transmission
can be modeled by a tabulated file. 

As for {\bf Monochromator\_flat}, the crystal is assumed to be infinitely
thin, and the varition in lattice spacing, ($\Delta d/d$), 
is assumed to be zero. Hence, this
component is not suitable for simulating backscattering instruments or to
investigate multiple scattering effects.

The algorithm for scattering and calculation of the neutron weight for
the individual blades is described under {\bf Monochromator\_flat}.