\section{Collimator\_radial: A radial Soller blade collimator}
\index{Optics!Radial collimator}

\component{Collimator\_radial}{(System) E.Farhi, ILL}{$w_1$, $h_1$, $w_2$, $h_2$, $len$, $\theta_{min}$, $\theta_{max}$, $nchan$, $radius$}{$divergence$, $nblades$, $roc$ and others}{Validated}

This radial collimator works either using an analytical approximation
like {\bf Collimator\_linear} (see section \ref{collimator-linear}), 
or with an exact model.

The input parameters are the inner radius $radius$, the radial length $len$, 
the input and output window dimensions $w_1$, $h_1$, $w_2$, $h_2$, 
the number of Soller channels $nchan$ 
(each of then being a single linear collimator) covering the angular interval
[$\theta_{min}$, $\theta_{max}$] angle with respect to the $z$-axis.

If the $divergence$ parameter is defined, 
the approximation level is used as in {\rm Collimator\_linear} 
(see section \ref{collimator-linear}). 
On the other hand, if you perfer to describe exactly the number of blades 
$nblades$ assembled to build a single collimator channel, 
then the model is exact, and traces the neutron trajectory inside each Soller. 
The computing efficiency is then lowered by a factor 2.

The component can be made oscillating with an amplitude of $roc$ times 
$\pm w_1$.

\begin{figure}
  \begin{center}
    \includegraphics[width=0.8\textwidth]{figures/radial.eps}
  \end{center}
\caption{A radial collimator}
\label{f:coll-radial}
\end{figure}
