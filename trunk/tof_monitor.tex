% Emacs settings: -*-mode: latex; TeX-master: "manual.tex"; -*-

\section{TOF\_monitor: The time-of-flight monitor}
\component{TOF\_monitor}{System}{$x_{\rm min}$, $x_{\rm max}$, $y_{\rm min}$, $y_{\rm max}$, $n_{\rm chan}$, $t_0$, $t_1$, filename}{$\Delta t$}{}

The component {\bf TOF\_monitor} has a rectangular opening
in the $x-y$ plane, given by the $x$ and $y$ parameters,
like for {\rm Slit}.
The neutron is propagated to the plane of the monitor
by the kernel call PROP\_Z0.
Any neutron ray that passes within the opening is counted, and
the total neutron counts are updated:

Special about {\bf TOF\_monitor} is that it is sensitive to
the time, $t$, where the neutron ray is hits the component.
Like in a real time-of-flight detector, the time dimension is
binned into small time intervals.
Hence this monitor updates a one-dimensional array of counts.
The $n_{\rm chan}$ time intervals begin at $t_0$ and
end at $t_1$ (or have each length $dt$ if this parameter is specified).
As usual in time-of-flight analysis, all times are given in units of $\mu$s.

The output parameters from {\bf TOF\_monitor} are the three count numbers,
$N, I$, and $M_2$ for the total counts in the monitor.
In addition a file is produced with a list of the same three data divided in
different TOF bins.
This file can be read and plotted by the {\rm MCplot} tool; see the
System Manual.

