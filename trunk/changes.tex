% Emacs settings: -*-mode: latex; TeX-master: "manual.tex"; -*-

\chapter{New features in \MCS\ \version\ }
\label{c:changes}

This version of \MCS\ implements both new features, as well as many bug corrections. Bugs are reported and traced using the \MCS\ Bugzilla system \cite{mczilla_webpage}. We will not present here an extensive list of corrections, and we let the reader refer to this bug reporting service for details. Only important changes are indicated here.

Of course, we can not guarantee that the software is bullet proof, but we do our best to correct bugs, when they are reported.\index{Bugs}

%\section{General}
%\label{s:new-features:general}
%\begin{itemize}
%\end{itemize}

\section{Kernel}
\label{s:new-features:kernel}
\index{Kernel}

The following changes concern the 'Kernel' (i.e. the \MCS\ meta-language and program). See the dedicated chapter in the {\it User manual} for more details.

\begin{itemize}
\item New \verb+SPLIT+ keyword, method for improvement of statistics. Can be considered equivalent to splitting
  simulations in part by use of the \verb+Virtual_output+ and \verb+Virtual_input+ components. Please consult the
  relevant documentation in Section \ref{s:instrdefs-extend-enhance} before using this feature!
\end{itemize}

\section{Run-time}
\label{s:new-features:run-time}
\index{Library!Run-time}

\begin{itemize}
\item The threading mechanism for parallelisation has been removed from McStas since it caused too many problems. For 
  parallelisation on single machines (e.g. modern dual-core processors) or clusters, MPI (MPICH) is the recommended solution.
  The McStas team members routinely run developer machines and clusters using MPI.
\end{itemize}


\section{Components and Library}
\label{s:new-features:components}
\index{Components} \index{Library!Components}
We here list the new and updated components (found in the \MCS\ \verb+lib+ directory)
which are detailed in the {\it Component manual}, also mentioned in
the {\it Component Overview} of the {\it User Manual}.
\subsection{New components}
\begin{itemize}
\item \verb+Tunneling_sample.comp+ Double-cylinder shaped all-incoherent scatterer with elastic, quasielastic (Lorentzian)
  and tunneling (sharp) components. No multiple scattering. Absorbtion included. By Kim Lefmann
\item \verb+TOF2E_monitor.comp+ TOF-sensitive monitor, converting to energy. By Kim Lefmann
\end{itemize}
\subsection{Updated components}
\begin{itemize}
\item \verb+Source_adapt.comp+, additions by Aaron Percival which allows to specify a flat wavlength distibution.
\item \verb+Single_crystal.comp+, warning NOT to use this component as a monochromator (bug fix/validation under way).
\item \verb+Powder_N.comp+, can now be used in concentric mode, i.e. for modelling sample surroundings (cryostat, container..).
\item Various minor updates to other components, including \verb+Monochromator_curved.comp+ and \verb+Progress_bar.comp+
\end{itemize}
\subsection{New example instruments}
\begin{itemize}
  \item \verb+ILL_H15_IN6+ by Emmanuel Farhi
  \item \verb+ILL_H142_IN12+ by Emmanuel Farhi
  \item \verb+Histogrammer.instr+ by Peter Willendrup - see the 'Tools' section
  \item \verb+ESS_IN5_reprate.instr+ Instrument for simulating IN5-TYPE (cold chopper) multi-frame spectrometer at ESS LPT.
    (example instrument for \verb+Tunneling_sample.comp+.) By  Kim Lefmann
\end{itemize}
\section{Documentation}
\label{s:new-features:documentation}
\begin{itemize}
\item Manual and component manual slightly updated according to
  adding/modification of components and functionality.
\item New appendix on the polarisation features. {\bf(p)}
\end{itemize}

\section{Tools, installation}
\label{s:new-features:tools}
\index{Tools}
\index{Installing}
\subsection{New tools}
\begin{itemize}
  \item New mcdaemon for visualisation of intermediate simulation results (obtained by sending USR2 signal to a 
    running simulation or by using the \verb+Progress_bar+ component with \verb+flag_save=1+).
  \item Improvements to mcgui:
    \begin{itemize}
      \item New tool menu with hooks to mcformat, mcdaemon and mcplot
      \item Possibility for auto-setup of MPI ssh keys
      \item Possibility to run the McStas editor in 'detached' mode, hence available whilst a simulation is running
      \item Histogrammer.instr: Special histogramming instrument for visualisation of virtual source files (\verb+Virtual_input+,
	VitESS, MCNP and Tripoli formats)
    \end{itemize}
  \item PGPLOT output format (original McStas format) is now possible on Windows. A pgplot/pgperl installation
    is included in a standard McStas Win32 installation.
  \item  NeXus output format possible. To use this feature, HDF and NeXus libraries must be available
    and functional on your system before installing McStas from source. (In case of a binary package, 
    you MUST recompile the McStas software.) To enable a McStas build with NeXus, run 
    \verb+./configure --with-nexus.+
\end{itemize}
\subsection{Installation}
\begin{itemize} 
\item Mac OS X is now considered a supported platform. For now, no actuall installer program
  is given, but all needed software has been packed together with easy to follow instructions.
  Test of the instructions have been performed on Mac OS X 10.4 Tiger on both Intel and PPC
  hardware.
\item McStas now comes in a Debian binary package (\verb+.deb+), tested to work on Debian and Ubuntu systems. The debian
    package provides McStas, pgplot and pgperl and have dependencies for the perl, perl-tk, gcc, libg2c0, pdl and libc6-dev 
    packages.See \verb+http://www.mcstas.org+ or  Chapter \ref{installing} for details.
\item As also mentioned in the list of new tools, PGPLOT output is available on Windows with a default \MCS\ installation.
\item  Intel C compiler - The documentation now includes instructions to run McStas with the Intel C compiler (available on Windows,
    Linux x86 and Mac OS x86 systems). Typically, a performance gain of 2 is found relative to gcc -O2. Relevant
    compiler flags are:
    \begin{itemize}
      \item \verb+MCSTAS_CFLAGS="-g -O2 -wd177,266,1011,181"+
    \end{itemize}
\item To run McStas with MPI and the Intel C compiler, you may have to edit your mpicc shell script to set \verb+CC="icc"+
\end{itemize}
Details about the installation and the available tools are given in chapter \ref{installing}.

\subsection{Warnings}
{\bf WARNING:} The 'dash' shell which is used as /bin/sh on some Linux system (Including Ubuntu 7.04) makes the 'Cancel' and 'Update' 
buttons fail in mcgui. Solutions are:
\begin{itemize}
\item[a)] If your system is a Debian or Ubuntu, please install our Debian package which requests automatic removal of 'dash'.
\item[b)] If your /bin/sh is dash, please install bash and manually change the /bin/sh link to point at bash.
\end{itemize}

\section{Future extensions}
\label{s:future}
The following features are planned for the oncoming releases of \MCS\
(not an ordered list):
\begin{itemize}
\item Increased validation and testing.
\item Extend test cases to all (most) components. One instrument
  pr. component. (Probably not in \verb+examples/+.
\item Better support for heterogeneous grid computing.
\item Updates to mcresplot to support the Matlab and Scilab backends.
\item Compatibility issues with recent Perl releases on Windows.
\item Compatibility issues with the next release of Scilab.
\item Global changes of components relating to polarisation
  visualisation.
\item Visualisation of neutron spins in magnetic fields for all
  graphical backends.
\item \emph{Array} \verb+AT+ specifiers for components, i.e. \\
  \verb+COMPONENT MyComp=Comp(...)+\\\verb+AT([Xarray],[Yarray],[Zarray])+ and\\
  \verb+AT Positions('filename')+
\item Gui support for array AT specifiers.
\item More complete polarisation support including numerically defined
  magnetic fields and advanced sample components.
\item Perl plotting alternative to PGPLOT.
\item Larger variety of sample components.
\end{itemize}








