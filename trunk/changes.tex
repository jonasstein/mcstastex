% Emacs settings: -*-mode: latex; TeX-master: "manual.tex"; -*-

\chapter{New features in \MCS\ \version\ }
\label{c:changes}

This version of \MCS\ implements both new features, as well as many bug corrections, especially for the Windows platform. Bugs are reported and traced using the \MCS\ Bugzilla system \cite{mczilla_webpage}. We will not present here an extensive list of corrections, and we let the reader refer to this bug reporting service for details. Only important changes are indicated here.

Of course, we can not garranty that the software is bullet proof, but we do our best to correct bugs, when they are reported.\index{Bugs}

\section{General}
\label{s:new-features:general}
\begin{itemize}
\item  This release is a major step forward for McStas in terms of
  support for polarisation. Peter Christiansen who worked at Ris\o\
  March-October 2006 was the main workforce behind this new
  functionality, backed by funding from ISIS, work by Rob Dalgliesh at
  ISIS plus functionality of the VitESS and NSIP packages. As this is
  the first release with polarisation support, the methods and
  algorithms implemented are probably not the final solution. Hence,
  documentation of the functionality has been placed in Appendix
  \ref{c:polarization} to this manual. Polarisation users: Please give
  us feedback for further development! Items marked by {\bf(p)} below
  relate to the new polarisation support. 
\item New keywords for the meta language improves support for e.g. description of sample enviroments.
\item A method for automatic optimisation has been implemented, e.g.~for achieving maximum flux at the sample position (any quantity measured by a McStas monitor can be optimised) as a function of simulation parameters.
\item Bugfixes in many different areas.
\end{itemize}
\section{Kernel}
\label{s:new-features:kernel}
\index{Kernel}

The following changes concern the 'Kernel' (i.e. the \MCS\ meta-language and program). See the dedicated chapter in the {\it User manual} for more details.

\begin{itemize}
\item New \verb+WHEN+ keyword, conditional use of components,
  \\\verb+COMPONENT MyComp=Component(...) WHEN (condition) AT+
\item New \verb+JUMP+ keyword, possibility to iterate a given component a
  number of times (multiple scattering) or for 'teleportation' to a
  given component. (USE WITH CAUTION)
\item Improved \verb+COPY+ keyword with parameter substitution (Make a copy
  of an other component instance with a few parameters changed)
\item Spin propagation algorithm (When magnetic field is set 'on', the
  central propagation routines also handle Larmor precession in the field) {\bf(p)}
\item Handeling of analytical B-fields {\bf(p)}
\end{itemize}

\section{Run-time}
\label{s:new-features:run-time}
\index{Library!Run-time}

\begin{itemize}
\item Optmisation support in mcrun/mcgui using Perl::Amoeba (se description above)
\item \verb+POSIX+ threading on multi-core processors (BEWARE, performance is generally better using MPI)
\end{itemize}


\section{Components and Library}
\label{s:new-features:components}
\index{Components} \index{Library!Components}

We here list some of the new components (found in the \MCS\ \verb+lib+ directory)
which are detailed in the {\it Component manual}, also mentioned in
the {\it Component Overview} of the {\it User Manual}.
\subsection{New components}
\begin{itemize}
\item Monochromator\_pol.comp - Polarising monochromator/analyzer {\bf(p)}
\item Pol\_bender.comp - Polarising bender {\bf(p)}
\item Pol\_mirror.comp - Polarising mirror {\bf(p)}
\item Pol\_guide\_vmirror.comp - Guide with semi-transparent, polarising mirror {\bf(p)}
\item Pol\_simpleBfield.comp - Numerical precession in analytical B-fields {\bf(p)}
\item 3 Polarisation monitors - MeanPolLambda\_monitor.comp, PolLambda\_monitor.comp
    Pol\_monitor.comp {\bf(p)}
\item PSD\_Detector.comp - Physical detector - comes with many gas lookup tables 
    (Contrib: Thorwald van Vuure, ILL)
\item Virtual\_mcnp\_*.comp - MCNP event file handeling
    (Contrib: Chama Hennane, ENSIMAG and Emmanuel Farhi, ILL) 
\item Source\_multi\_surfaces.comp - Source comp with multiple surface areas with individual
    spectrums (Contrib: Ludovic Giller, Uwe Filges)
\item V\_sample enriched with quasi-elastic features (Kim Lefmann)
\end{itemize}
\subsection{New example instruments}
\begin{itemize}
\item TAS frontend with reciprocal space (hkl) calculator (Emmanuel Farhi)
\item Example instruments for polarisation comps {\bf(p)}
\item QUENS\_test.instr (Test of new quasi-elastic features V\_sample)
\end{itemize}
\section{Documentation}
\label{s:new-features:documentation}
\begin{itemize}
\item Manual and component manual slightly updated according to
  adding/modification of components and functionality. 
\item New appendix on the polarisation features. {\bf(p)}
\end{itemize}

\section{Tools, installation}
\label{s:new-features:tools}
\index{Tools}
\index{Installing}
\subsection{New tools} 
\begin{itemize}
\item New 'merge/convert' tool mcformat: Convert between McStas output formats and merge cluster
    node datasets into one dataset.
\item TOF mode for mcdisplay (statistical chopper acceptance diagrams generated from the simulated
    neutron rays)
\end{itemize}
\subsection{Installation}
\begin{itemize}
\item Much easier installation on Windows, single executable file with all support applications
    is now available. Simply click 'next' alle the way and you are done.
\item PGPLOT and Scilab installation help tools on Linux systems
\end{itemize}
Details about the installation and the available tools are given in chapter \ref{installing}.

\section{Future extensions}
\label{s:future}
The following features are planned for the oncoming releases of \MCS\ 
(not an ordered list):
\begin{itemize}
\item Increased validation and testing.
\item Extend test cases to all (most) components. One instrument
  pr. component. (Probably not in \verb+examples/+.
\item Better support for heterogeneous grid computing.
\item Improvements to the NeXus data format (initial support is
  already included).
\item Updates to mcresplot to support the Matlab and Scilab backends.
\item Compatibility issues with recent Perl releases on Windows.
\item Compatibility issues with the next release of Scilab.
\item Global changes of components relating to polarisation
  visualisation.
\item Visualisation of neutron spins in magnetic fields for all
  graphical backends.
\item \emph{Array} \verb+AT+ specifiers for components, i.e. \\
  \verb+COMPONENT MyComp=Comp(...)+\\\verb+AT([Xarray],[Yarray],[Zarray])+ and\\
  \verb+AT Positions('filename')+
\item Gui support for array AT specifiers.
\item More complete polarisation support including numerically defined
  magnetic fields and advanced sample components.
\item Perl plotting alternative to PGPLOT.
\item Larger variety of sample components.
\end{itemize}








