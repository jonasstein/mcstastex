% Emacs settings: -*-mode: latex; TeX-master: "manual.tex"; -*-

\section{Res\_monitor: The detector for resolution calculation}
\label{s:res_monitor}

\component{Res\_monitor}{System, Alan Tennant}{$x_{\rm min}$, $x_{\rm max}$, $y_{\rm min}$, $y_{\rm max}$, filename, res\_sample, buffer size}{$x_w$, $y_h$, $z_t$, options}{}

The component {\bf Res\_monitor} is used for calculating the
resolution function of a particular instrument with detector of the
shape, size, and position as {\bf Res\_monitor}.
The shape of {\rm Res\_sample} is by default rectangular,
but can be a box, a sphere, a disk, or a cylinder,
depending on the parameter ``options''.
The component works like a normal single detector, but
also records all scattering events in the resolution sample and writes
them to a file that can later be read by \verb+mcresplot+.

As described in section~\ref{s:res_sample},
the instrument definition should contain an instance of the
\textbf{Res\_sample} component, the name of which should be passed as an
input parameter to \textbf{Res\_monitor}. For example
\begin{verbatim}
    COMPONENT mysample = Res_sample( ... )
    ...
    COMPONENT det = Res_monitor(res_sample_comp = mysample, ...)
    ...
\end{verbatim}

The output file is in ASCII format, one line per scattering event, with
the following columns:
\begin{itemize}
\item ${\bf k}_{\rm i}$, the three components of the initial wave vector.
\item ${\bf k}_{\rm f}$, the three components of the final wave vector.
\item ${\bf r}$, the three components of the position of the scattering
  event in the sample.
\item $p_{\rm i}$, the neutron weight just after the scattering event.
\item $p_{\rm f}$, the relative neutron weight adjustment from sample to
  detector (so the total weight in the detector is $p_{\rm i}p_{\rm f}$).
\end{itemize}
From ${\bf k}_{\rm i}$ and ${\bf k}_{\rm f}$, we may compute ${\bf Q} =
{\bf k}_{\rm i} - {\bf k}_{\rm f}$ and $\omega = (\mbox{2.072
  meV$\cdot$\AA$^2$})({\bf k}_{\rm i}^2 - {\bf k}_{\rm f}^2)$.

The vectors are given in the local coordinate system of the resolution
sample component. The wave vectors are in units of $\mbox{\AA}^{-1}$, the
scattering position in units of meters.

 The output parameters from {\bf Res\_monitor}
are the three count numbers, \textit{Nsum}, \textit{psum},
and \textit{p2sum}, and the handle \textit{file} of the output file.
