This component models a simple phonon signal from a single crystal of 
one element in an {\em fcc} crystal structure.
Only one isotropic acoustic phonon branch is modeled, and the longitudinal
and transverse dispersions are identical with the velocity of sound being $c$.
Other physical parameters are the atomic mass, $M$, the lattice parameter, $a$,
the scattering length, $b$,
the Debye-Waller factor, \verb+DW+, and the temperature, $T$.

Incoherent scattering and absorption are also taken into account by the cross
sections $\sigma_{\rm abs}$ and $\sigma_{\rm inc}$.

The sample can have the form of a cylinder with height $h$ and radius
$r_0$, or a box with dimensions $w_x, h_y, t_z$. 

Phonons are emitted into a specific range of solid angles, specified 
by the location $(x_t, y_t, z_t)$ and the focusing radius, $r_0$.
Alternatively, the focusing is given by a rectangle, 
$w_{\rm focus}$ and $h_{\rm focus}$, and the focus point is given by the
indeox of a down-stream optical element, \verb+target_index+.

Multiple scattering is not included in this component. The scattering
cross section is given by the detailed calculations below.

\subsubsection{The phonon cross section} % This is taken directly from the paper %
The inelastic phonon cross section for a Bravais crystal of a pure element
is given by Ref.~\cite[ch.3~]{squires} 
\begin{eqnarray}
\frac{d^2\sigma'}{d\Omega dE_{\rm f}} &=&
  b^2 \frac{k_{\rm f}}{k_{\rm i}} \frac{(2\pi)^3}{V_0}\frac{1}{2M} \exp(-2W) \nonumber \\
&\times&
  \sum_{\tau,q,p} \frac{(\mbox{\boldmath $\kappa$} \cdot {\bf e}_{q,p})^2}
                       {\omega_{q,p}} 
  \left\langle n_{q,p} + \frac{1}{2} \mp \frac{1}{2} \right\rangle
  \delta(\omega\pm\omega_{q,p}) \delta(\kappa\pm{\bf q}-\tau) ,
\end{eqnarray}
where both annihilation and creation of one phonon is considered
(represented by the plus and minus sign in the dispersion relation,
respectively).
In the equation, 
$\exp(-2W)$ is the Debye-Waller factor, \verb+DW+ and
$V_{\rm c} = \rho_{\rm c}^{-1}$ is the volume of the unit cell.
The sum runs over the reciprocal lattice vectors, $\tau$, 
over the polarisation index, $p$,
and the $N$ allowed wave vectors {\bf q} within the Brillouin zone
(where $N$ is the number of unit cells in the crystal).
Further, ${\bf e}_{q,p}$ is the
polarization unit vectors, $\omega_{q,p}$ the phonon dispersion,
and $\langle n_{q,p} \rangle$ is the Bose factor at the given value of
$\hbar |\omega_{q,p}|/(k_{\rm B}T)$.

We have simplified this expression by assuming no polarization
dependence of the dispersion, giving 
$\sum_{p} (\mbox{\boldmath $\kappa$} \cdot {\bf e}_{q,p})^2 = \kappa^2$.
We assume that the inter-atomic interaction is nearest-neighbour-only
so that the phonon dispersion becomes: 
\begin{equation}
d_1({\bf q}) = c_1/a \sqrt(z-s_q) ,
\end{equation}
where $z=12$ is the number of nearest neighbours and 
$s_q=\sum_{\rm nn} \cos({\bf q} \cdot {\bf r}_{\rm nn})$,
where in turn ${\bf r}_{\rm nn}$ is the lattice positions of the 
nearest neighbours.

This dispersion relation may be modified with a small effort,
since it is given as a separate c-function attatched to the component.

To calculate $d\sigma/d\Omega$ we need to transform the 
{\bf q} sum into an integral over the Brillouin zone by 
$\sum_q \rightarrow N V_{\rm c} (2\pi)^{-3} \int_{\rm BZ} d^3{\bf q}$.
The $\mbox{\boldmath $\kappa$}$ sum can now be removed by
expanding the {\bf q} integral to infinity. 
All in all, the partial differential cross section reads
\begin{eqnarray}
\frac{d^2\sigma'}{d\Omega dE_{\rm f}}
  (\mbox{\boldmath $\kappa$},\omega) &=&
  b^2 \frac{k_{\rm f}}{k_{\rm i}} N \frac{1}{2M} 
  \int \frac{\hbar \kappa^2}{c_1 |{\bf q}-{\bf Q}|} 
  \left\langle n_{q}+\frac{1}{2}\mp\frac{1}{2} \right\rangle
  \delta(\omega\pm\omega_{q}) \delta(\mbox{\boldmath $\kappa$}\pm{\bf q}) 
   d^3{\bf q} \nonumber \\
 &=& b^2 \frac{k_{\rm f}}{k_{\rm i}} N 
          \frac{\hbar^2 \kappa^2}{2M c_1|{\bf q}-{\bf Q}|} 
  \left\langle n_{\kappa}+\frac12\pm\frac12 \right\rangle 
  \delta(\hbar\omega\pm d_1(\kappa)) .
\end{eqnarray}
Using the integration sketched in eq.~(\ref{eq:dcsdisp}), we reach
*** CHECK THIS!! ***
\begin{equation} \label{eq:phononcross}
\left(\frac{d\sigma'}{d\Omega}\right)_j =
b^2 \frac{k_{\rm f}^2}{k_{\rm i}} N 
\frac{\hbar^4 \kappa^2}{2Mm c_1 |{\bf q}-{\bf Q}| J(k_{{\rm f},j})} 
\left\langle n_{\kappa}+\frac12\pm\frac12 \right\rangle . 
\end{equation}
A rough order-of-magnitude consideration gives
$\frac{\hbar^2 k_{{\rm f},j}}{mJ(k_{{\rm f},j})} \approx 1$,
$\frac{k_{{\rm f},j}}{k_{{\rm f},i}}\approx 1$,
$\langle n_{\kappa}+\frac12\pm\frac12 \rangle \approx 1$,
$\frac{\hbar^2\kappa^2}{2M\hbar\omega_\kappa}
\approx \frac{m}{M}$ (for large disperison values?????).
Hence, $\left(\frac{d\sigma}{d\Omega}\right)_j \approx N b^2 \frac{m}{M}$, and
$\sigma_{\rm inel}$ becomes a fraction of $4 \pi N b^2$, as one
would expect.
The differential cross section per unit cell is found from 
(\ref{eq:phononcross}) by letting $N=1$. 

\subsubsection{The algorithm}
(*** SEE IF THIS IS WRITTEN ELSEWHERE ***)

\subsubsection{The weight transformation}
The total weight transformation becomes
\begin{equation} \label{eq:phonon_mult}
\pi_i = a_{\rm lin} \rho_c l_{\rm max} n_{\rm s} \Delta \Omega
 b^2 \frac{k_{\rm f}^2}{k_{\rm i}}
 \frac{\hbar^4 \kappa}{2Mm c_1 |{\bf q}-{\bf Q}| J(k_{{\rm f},j})} 
 \left\langle n_{\kappa}+\frac12\pm\frac12 \right\rangle ,
\end{equation}
where the Jacobian reads
\begin{equation}
J = -\frac{\hbar^2}{m} k_{\rm f} 
    - c \frac{\partial}{\partial k_{\rm f}} 
        \left| {\bf k}_{\rm i} - k_{\rm f} \hat{k}_{\rm f} \right| .
\end{equation}

