\section{Vitual\_input: Starting the second part of a split simulation}
\label{virtual_input}
\index{Sources!Virtual source from stored neutron events}

\component{Virtual\_input}{System}{filename}{repeat count, type}{}

The component {\bf Virtual\_input} resumes a split simulation where the
first part has been performed by another instrument and the neutron ray
parameters have been stored by the component {\bf Virtual\_output}.

All neutron ray parameters are read from the input file, which is by default
of ``text'' type, but can also assume the binary formats
``float'' or ``double''. The reading of neutron rays continue until the
specified number of rays have been simulated or
till the file has been exhausted. If desirable, the input file
can be reused a number of times, determined by the optional parameter
``repeat count''. This is only relevant if you know that further part of the suimulation makes use of MC choices, else you will repeat exactly the same events all the way long. However, care should be taken when dealing with
absolute intensities, which are automatically corrected only
when the input file has been exhausted once.

The simulation ends either with the end of the repeated file counts, or with the normal end with $ncount$ \MCS\ simulation events. We recommand to use a very large value, e.g. $ncount=1e10$ when launching simulations.
