% Emacs settings: -*-mode: latex; TeX-master: "manual.tex"; -*-

\chapter{Monte Carlo Techniques and simulation strategy}
\label{s:MCtechniques}

This chapter explains the simulation strategy and the Monte Carlo
techniques used in \MCS. We first explain the concept of the neutron
weight factor, and discuss the statistical errors in dealing with sums
of neutron weights.  Secondly, we give an expression for how the weight
factor transforms under a Monte Carlo choice and specialize this
to the concept of direction focusing.  Finally, we present a way of
generating random numbers with arbitrary distributions.

\section{Introduction}
What makes physicists happy? Probably collect good quality data, pushing the instruments to their limits, and fit that data to physical models.
Among available measurement techniques, neutron scattering provides a large variety of spectrometers to probe structure and dynamics of all kinds of materials.

Unfortunately the neutron flux is often a limitation in the experiments. This then motivates instrument responsibles to improve the flux and the overall efficiency at the spectrometer positions, and even to design new machines. Using both analytical and numerical methods, optimal configurations may be found.

But achieving a satisfactory experiment on the best neutron spectrometer is not all. Once collected, the data analysis process raises some questions concerning the signal: what is the background signal ? What proportion of coherent and incoherent scattering has been measured ? Is is possible to identify clearly the purely elastic (structure) contribution from the quasi-elastic and inelastic one (dynamics) ? What are the contributions from the sample geometry, the container, the sample environment, and generally the instrument itself ? And last but not least, how does multiple scattering affect the signal ? Most of the time, the physicist will elude these questions using rough approximations, or applying analytical corrections \cite{Copley86}. Monte-Carlo techniques provide a mean to evaluate some of these quantities.

\section{Neutron spectrometer simulations}

Neutron scattering instruments are built as a series of neutron optics elements. 
Each of these elements modifies the beam characteristics 
(e.g. divergence, wavelength spread, spatial and time distributions) 
in a way which may be modeled through analytical methods, 
for simplified neutron beam configurations.
This stands for individual elements such as guides \cite{Leibnitz63,Mildner90}, 
choppers \cite{Lowde60,Copley03}, Fermi choppers \cite{Fermi47,Peters05}, 
velocity selectors \cite{Clark66}, 
monochromators \cite{Freund83,Sears97,Shirane02,Alianelli04}, 
and detectors \cite{Radeka74,Charpak89,Manzin04}.
In the case of selected neutron instrument parts, one may use efficiently 
the so-called acceptance diagram theory \cite{Mildner90,Copley93,Cussen03} 
within which the neutron beam distributions are considered to be homogeneous 
or gaussian.
However, the concatenation of a high number of neutron optical elements, 
which indeed constitute real instruments, brings additional complexity 
by introducing strong correlations between neutron beam parameters: 
divergence and position - which is the basis of the acceptance diagram method - 
but also wavelength and time. 
The usual analytical methods (phase-space theory...) then reach their limit 
of validity in the description of the resulting fine effects.

In principle, computing individual neutron event propagation 
at each instrument part, using analytical and numerical models, 
is not such a hard task. 
The use of probablilities is common to describe microscopic physical processes. 
Integrating all these events over the propagation path will result 
in an estimation of measurable quantities characterizing the neutron instrument. 
Moreover, using variance reduction ({\it e.g.} importance sampling), 
whenever possible, will both speed-up the computation and achieve 
a better accuracy. What we just sketched is nothing else than the basis 
of the Monte-Carlo (MC) method \cite{James80}, 
applied to neutron ray-tracing instrumentation.

\subsection{Monte Carlo ray tracing simulations}
Mathematically, the Monte-Carlo method is an application of the law of large numbers \cite{James80,Grimmett92}. Let $f(u)$ be a finite continuous integrable function of parameter $u$ for which an integral estimate is desirable. The discrete statistical mean value of $f$ (computed as a series) in the uniformly sampled interval $a < u < b$ converges to the mathematical mean value of $f$ over the same interval.

\begin{equation}
\lim_{n \rightarrow \infty} \frac{1}{n} \sum_{i=1, a \leq u_i \leq b}^n f(u_i) = \frac{1}{b-a}\int_a^b f(u) du
\end{equation}

In the case were the $u_i$ values are regularly sampled, we come to the well known midpoint integration rule. In the case were the $u_i$ values are randomly (but regularly) sampled, this is the Monte-Carlo integration technique. As random generators are not perfect, we rather talk about \emph{quasi}-Monte-Carlo technique. We encourage the reader to refer to James \cite{James80} for a detailed review on the Monte-Carlo method.

Although early implementations of the method for neutron instruments used \emph{home-made} computer programs  (see e.g. papers by J.R.D. Copley, D.F.R. Mildner, J.M. Carpenter, J. Cook), more general packages have been designed, providing models for most parts of the simulations.
These present existing packages are: NISP \cite{NISP}, ResTrax \cite{Restrax}, McStas \cite{nn_10_20,mcstas_pb,mcstas_webpage}, Vitess \cite{Vitess,vitess_webpage}, and IDEAS \cite{IDEAS}.
Their usage usually covers all types of neutron spectrometers, most of the time through a user-friendly graphical interface, without requiring programming skills.

The neutron ray-tracing Monte-Carlo method has been used widely for 
{\em e.g.}\ guide studies \cite{Copley93,Farhi02,Schanzer04}, 
instrument optimization and design \cite{Zsigmond04,Lieutenant05}. 
Most of the time, the conclusions and general behaviour of such studies 
may be obtained using the classical analytical approaches, 
but accurate estimates for the flux, the resolutions, 
and generally the optimum parameter set, benefit advantageously from MC methods.

Recently, the concept of virtual experiments, {\em i.e.}\ full simulations
of a complete neutron experiment, has been suggested as the main 
goal for neutron ray-tracing simulations \cite{lefmann05}. The goal is that
simulations should be of benefit to not only instrument builders, but also
to users for training, experiment planning, diagnostics, and data analysis.

\section{The neutron weight}
\label{s:probweight}
A totally realistic semi-classical simulation will require that
each neutron is at any time either present or lost.
In many instruments, only a very
small fraction of the initial neutrons will ever be detected, and
simulations of this kind will therefore waste much time in dealing
with neutrons that never hit the detector.

An important way of speeding up calculations is to introduce
a neutron "weight factor" for each simulated neutron ray and to
adjust this weight according to the path of the ray.
If {\em e.g.}\ the reflectivity of a certain
optical component is 10\%, and only reflected neutrons ray are
considered in the simulations, the neutron
weight will be multiplied by 0.10 when passing this component,
but every neutron is allowed to reflect in the component.
In contrast, the totally realistic simulation of the component
would require in average ten incoming neutrons for each reflected one.

Let the initial neutron weight be $p_0$ and let us denote the weight
multiplication factor in the $j$'th component by $\pi_j$.  The resulting
weight factor for the neutron ray after passage of the whole instrument 
becomes the product of all contributions
\begin{equation}
\label{e:probprod}
p = p_0 \prod_{j=1}^n \pi_j .
\end{equation}
For convenience, the value of $p$ is updated (within each component) 
during the simulation.

Simulation by weight adjustment is performed
whenever possible. This includes
\begin{itemize}
\item Transmission through filters.
\item Transmission through Soller blade collimator
 (in the approximation
 which does not take each blade into account).
\item Reflection from monochromator (and analyser) crystals
 with finite reflectivity and mosaicity.
\item Passage of a continuous beam through a chopper.
\item Scattering from samples.
\end{itemize}

\subsection{Statistical errors of non-integer counts}
\label{s:staterror}

In a typical simulation, the result will consist of a
count of neutrons histories ("rays") with different weights. The
sum of these weights is an estimate of the mean number of neutrons 
hitting the monitor (or detector) per second in a ``real'' experiment.
One may write the counting result as
\begin{equation}
\label{psum}
I = \sum_i p_i = N \overline{p} ,
\end{equation}
where $N$ is the number of neutrons in the detector and the vertical bar 
denote averaging.
By performing the weight transformations, the (statistical)
mean value of $I$ is unchanged. However, $N$ will in general be enhanced,
and this will improve the accuracy of the simulation.

To give an estimate of the statistical error, we proceed as follows:
Let us first for simplicity assume that all the counted neutron weights are
almost equal, $p_i \approx \overline{p}$,
and that we observe a large number of neutrons, $N \geq 10$.
Then $N$ almost follows a normal distribution
with the uncertainty $\sigma(N) = \sqrt{N}$
\footnote{This is not correct in a
situation where the detector counts a large fraction of the
neutrons in the simulation, but we will neglect that for now.}.
Hence, the statistical uncertainty of the observed intensity becomes
\begin{equation} \label{e:sigI1}
 \sigma(I) = \sqrt{N} \overline{p} = I / \sqrt{N} ,
\end{equation}
as is used in real neutron experiments (where $\overline{p} \equiv 1$).
For a better approximation we return to Eq.~(\ref{psum}).
Allowing variations in both $N$ and $\overline{p}$,
we calculate the variance of the resulting intensity,
assuming that the two variables are independent:
\begin{equation}
\sigma^2(I) = \sigma^2(N) \overline{p}^2 + N^2 \sigma^2(\overline{p}) .
\end{equation}
Assuming that $N$ follows a normal distribution, we reach 
$\sigma^2(N) \overline{p}^2 = N \overline{p}^2$.
Further, assuming that the individual weights, $p_i$, 
follow a Gaussian distribution (which in many cases is far from the truth)
we have 
$N^2 \sigma^2(\overline{p}) = \sigma^2(\sum_i p_i) = N \sigma^2(p_i)$
and reach
\begin{equation}
\sigma^2(I) = N \left( \overline{p}^2 + \sigma^2(p_i) \right).
\end{equation}
The statistical variance of the $p_i$'s is estimated by
$\sigma^2(p_i) \approx  (\sum_i p_i^2 - N \overline{p}^2) / (N-1)$.
The resulting variance then reads
\begin{equation}
\sigma^2(I) = \frac{N}{N-1} \left( \sum_i p_i^2 - \overline{p}^2  \right) .
\end{equation}
For almost any positive value of $N$, this is very well approximated
by the simple expression
\begin{equation}
\sigma^2(I) \approx \sum_i p_i^2 .
\end{equation}
As a consistency check, we note that for all $p_i$ equal, this reduces to
eq.~(\ref{e:sigI1})

In order to compute the intensities and uncertainties, the detector components
in \MCS\ thus must keep track of
$N=\sum_i p_i^0, I=\sum_i p_i^1$, and $M_2 = \sum_i p_i^2$.

\section{Weight factor transformations during a Monte Carlo
 choice}
When a Monte Carlo choice must be performed, {\em e.g.} when the
initial energy and direction of the neutron ray is decided at the source,
it is important to adjust the neutron weight so that the combined
effect of neutron weight change and Monte Carlo probability
of making this particular choice
equals the actual physical properties we like to model.

Let us follow up on the example of a source.
In the ``real'' semi-classical world, there is a distribution
(probability density) for the neutrons in the six dimensional
(energy, direction, position) space of
$\Pi(E,\Ombold,{\bf r}) = dP/(dE d\Ombold d^3{\bf r})$ depending upon
the source temperature, geometry {\em etc.}\ In the
Monte Carlo simulations, the six coordinates are for efficiency reasons
in general picked from another distribution:
$f_{\rm MC}(E,\Ombold,{\bf r}) \neq \Pi(E, \Ombold,{\bf r})$,
since one would {\em e.g.} often generate
only neutrons within a certain parameter interval.
However, we must then require that the weights are adjusted
by a factor $\pi_j$ (in this case: $j=1$) so that
\begin{equation} \label{probrule}
f_{\rm MC}(E,\Ombold,{\bf r}) \pi_j(E,\Ombold,{\bf r})
 = \Pi(E,\Ombold,{\bf r}) .
\end{equation}
%For the sources present in version \version,
%only the $(\Ombold, {\bf r})$ dependence of the correction factors
%are taken into account.

The weight factor transformation rule Eq.~(\ref{probrule})
is equally valid for other types of Monte Carlo choices,
although the probability distributions may depend upon
different parameters. An important example
is elastic scattering from a powder sample,
where the Monte-Carlo choices are the particular powder line to scatter from,
the scattering position within the sample and the final neutron direction
within the Debye-Scherrer cone.

It should be noted that the $\pi_j$'s found in the weight factor
transformation are multiplied by the $\pi_j$'s found by the
weight adjustments described in
subsection \ref{s:probweight} to yield the final neutron
weight given by Eq.~(\ref{e:probprod}).

\subsection{Direction focusing}
\label{s:focus}
An important application of weight transformation is direction focusing.
Assume that the sample scatters the neutron rays in many directions.
In general, only neutron rays in some of these directions will
stand any chance of being detected. These directions we call
the {\em interesting directions}.
The idea in focusing is to avoid wasting computation time on
neutrons scattered in the other directions.
This trick is an instance of what in Monte Carlo terminology
is known as {\em importance sampling}. % \cite{importance}.

If {\em e.g.} a sample scatters isotropically
over the whole $4\pi$ solid angle, and all interesting
directions are known to be contained within a certain
solid angle interval $\Delta \Ombold$, only these solid angles
are used for the Monte Carlo choice of scattering direction.
According to Eq.~(\ref{probrule}), the weight factor will then have
to be changed by the amount
$\pi_j = |\Delta \Ombold| / (4 \pi)$.
One thus ensures that the mean simulated intensity is unchanged
during a "correct" direction focusing, while a too narrow focusing will
result in a lower (\textit{i.e.} wrong) intensity, since 
we cut neutrons rays that should have counted.

\subsection{Adaptive sampling}
Another strategy to improve sampling in simulations
is adaptive importance sampling, % \cite{importance},
where \MCS\ during the simulations will determine
the most interesting directions and gradually change
the focusing according to that. 
Implementation of this idea is
found in the {\bf Source\_adapt} and {\bf Source\_optimizer} components.
%, described in section~\ref{s:Source_adapt}.
