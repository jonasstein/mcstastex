% Emacs settings: -*-mode: latex; TeX-master: "manual.tex"; -*-

\chapter{About the component library}
\label{c:components}
This \MCS\ Component Manual consists of the following major parts:
\begin{itemize}
\item A thorough description of system components,
with one chapter per major category: Sources, optics (two chapters),
monochromators, samples, monitors, and other components.
\item The \MCS\ library functions and definitions
  that aid in the writing of simulations and components in
  Appendix~\ref{c:kernelcalls}.
\item An explanation of the \MCS\ terminology in Appendix~\ref{s:terminology}.
\end{itemize}
Additionally, you may refer to the list of example instruments
from the library in the \MCS\ User Manual.

\section{Authorship}
The component library is
maintained by the \MCS\ system group. A number of basic components
``belongs'' the \MCS\ system, and are supported and tested by the \MCS\
team.

Other components are contributed
by specific authors, who are listed in the code for each component
they contribute as well as in this manual.
\MCS\ users are encouraged to send their
contributions to us for inclusion in future releases.

Some contributed components have later been taken over by the \MCS\ system
group, with permission from the original authors.
The original author will still appear both in the component code and in the
\MCS\ manual.

\subsection{Symbols for neutron scattering and simulation}
In the description of the theory behind the component functionality
we will use the usual symbols {\bf r} for the position
$(x,y,z)$ of the particle (unit m), and {\bf v} for
the particle velocity $(v_x, v_y, v_z)$ (unit m/s).
Another essential quantity is the neutron wave vector
${\bf k} = m_{\rm n} {\bf v}/\hbar$ , where
$m_{\rm n}$ is the neutron mass. {\bf k} is usually given in
\AA$^{-1}$, while neutron energies are given in meV.
The neutron wavelength is the reciprocal wave vector,
$\lambda=2 \pi / k$.
In general, vectors are denoted by boldface symbols.

Subscripts "i" and "f" denotes ``initial'' and ``final'', respectively,
and are used in connection with the neutron state before and after
an interaction with the component in question.
This is of particular importance in sample components, where the
wave vector change is denoted the {\em scattering vector}
\begin{equation}
{\bf q} \equiv {\bf k}_{\rm i} - {\bf k}_{\rm f} .
\end{equation}
In analogy, the {\em energy transfer} is given by
\begin{equation}
\hbar \omega \equiv E_{\rm i}-E_{\rm f} =
\frac{\hbar^2}{2 m_{\rm n}} \left( k_{\rm i}^2 - k_{\rm f}^2 \right).
\end{equation}

\section{Component coordinate system}
All mentioning of component geometry refer to
the local coordinate system of the individual component.
The axis convention is so that the $z$ axis is along
the neutron propagation axis, the $y$ axis is vertical up,
and the $x$ axis points left when looking along the $z$-axis,
completing a right-handed coordinate system.
Most components 'position' (as specified in the instrument description
with the \verb+AT+ keyword) corresponds to their input side.
Some components are centered (e.g. radial components).
\index{Symbols}\index{Coordinates system}

Components are usually not designed to overlap.
Warnings will be issued during simulation if sections of the instrument
are not reached by any neutron rays, or if neutrons are removed.
This is usually the sign of either overlapping components
or a very low intensity.\index{Removed neutron events}

\section{Component source code}
Source code for all components may be found in the \verb+MCSTAS+ library
subdirectory of the McStas installation;
the default is \verb+/usr/local/lib/mcstas/+
on Unix-like systems and \verb+C:\mcstas\lib+ on Windows systems, but it may be
changed using the \verb+MCSTAS+ environment variable.
\index{Environment variable!MCSTAS}

For temporary modification of a component, it is advised to make a copy
of the component file into the working directory.
A component file in the local directory will in \MCS\ take precedence over
a library component of the same name. In case users only require to add new features, preserving the existing features of a component, it is recommanded to make use of the \verb+EXTEND+ keyword\index{Keyword!EXTEND}, as documented in the User Manual.

\section{Documentation}
As a complement to this Component Manual, we encourage users to use
the \verb+mcdoc+ front-end which enables to display both the
catalog of the \MCS\ library, e.g using: \index{Tools!mcdoc}
\begin{quote}
  \verb|mcdoc|
\end{quote}
as well as the documentation of specific components, e.g with:
\begin{quote}
  \verb|mcdoc --text| {\it name} \\
  \verb|mcdoc| {\it file.comp}
\end{quote}
The first line will search for all components matching the {\it name},
and display their help section as text,
whereas the second example will display the help corresponding to
the {\it file.comp} component, using your
BROWSER\index{Environment variable!BROWSER} setting, or as text if unset.
The \verb+--help+ option will display the command help, as usual.

An overview of the component library is also given
in chapter \ref{s:components} and at the \MCS\ home page \cite{mcstas_webpage}.

\subsubsection{About data files}\index{Data files}\index{Library!read\_table-lib (Read\_Table)}
Some components require external data files,
e.g. lattice crystallographic definitions for Laue and powder pattern diffraction,
$S(q,\omega)$ tables for inelastic scattering,
neutron events files for virtual sources,
transmission and reflectivity files, etc.

Such files distributed with \MCS\ are located in the
\verb+data+ sub-directory of the \verb+MCSTAS+ library.
Components that make use of the \MCS\ file system,
including the \verb+read-table+ library (see section \ref{s:read-table})
may access all \MCS\ data files without making local copies.
Of course, you are welcome to define your own data files,
and eventually contribute to \MCS\ if you find them useful.


\section{Component validation}

