\section{Vitual\_output: Saving the first part of a split simulation}
\label{virtual_output}

\component{Virtual\_output}{System}{filename}{buffer size, type}{}

The component {\bf Virtual\_output} stores the neutron ray parameters
from a split simulation. It is then possible to let the
next part of the simulation be performed by another instrument,
which reads the stored neutron ray
parameters by the component {\bf Virtual\_input}.

All neutron ray parameters are saved to the output file, which is by default
of ``text'' type, but can also assume the binary formats
``float'' or ``double''. The storing of neutron rays continue until the
specified number of simulations have been performed.

The buffer size may be used to limit the size of the output file, but intentities may be then wrong. The best is to let the default value to zero,m and save all neutrons. You may control then the size of the file with the general $ncounts$ parameter.
